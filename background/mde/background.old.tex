
\subsection{Overview of the approach}


\subsection{Model, meta-model: what is the difference?}

\subsection{Models@run.time}
The models@run.time paradigm~\cite{DBLP:journals/computer/MorinBJFS09,DBLP:journals/computer/BlairBF09} is a well-known realisation of the MAPE-K loop to handle the challenges faced in self-adaptive systems development.
It extends model-driven engineering (MDE) concepts by spanning the use of models from design time to runtime (\ie during the execution of a system).
The model, as an abstraction of a real system, can be used during runtime to reason about the state of the actual system. 
A conceptual link between the model and the real system allows modifying the actual system through the model and vice versa.
Models in MDE provide semantically rich ways to define contexts that can be used in reasoning activities. 
The models@run.time paradigm uses meta-models to define the domain concepts of a real system together with its surrounding environment. 
In other words, it specifies which elements can be used by the model.
Consequently, the runtime model depicts an abstract and yet rich representation of the system context that conforms to (is an instance of) its meta-model.
In this work, we seek to use the models@run.time paradigm to represent, reason, and update the context of a system.

\subsection{Framework used in this thesis: GreyCat Modelling}

 
In the \gls{mde} community, the most famous framework is the \gls{emf}\footnote{\url{https://www.eclipse.org/modeling/emf/}}\cite{steinberg2008emf}.



In this work, we will use the GreyCat Modelling Framework, publicly available on GitHub\footnote{\url{https://github.com/datathings/greycat/tree/master/modeling}}.
This framework is the result of the thesis of Hartmann~\cite{}.
As we do in this work, he argued that modern modelling framework should introduce time as a first class concept.
Using this framework, we will define the model used to trace back decisions, their expected effects, and their circumstances.
