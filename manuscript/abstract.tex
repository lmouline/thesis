\chapter*{Abstract}

\highlightbox[Vision]
{
	As state-of-the-art techniques fail to model efficiently the evolution and the uncertainty existing in dynamically adaptive systems, the adaptation process makes suboptimal decisions.
	To tackle this challenge, modern modelling frameworks should efficiently encapsulate time and uncertainty as first-class concepts.
}
\bigskip

\paragraphit{Context}
Smart grid approach introduces information and communication technologies into traditional power grid to cope with new challenges of electricity distribution.
Among them, one challenge is the resiliency of the grid: how to automatically recover from any incident such as overload?
These systems therefore need a deep understanding of the ongoing situation which enables reasoning tasks for healing operations.
\textbf{Abstraction} is a key technique that provided a description of systems, their behaviours, and/or their environments alleviating their complexity.
\textbf{Adaptation} is a cornerstone feature that enables reconfiguration at runtime for optimising software to the current and/or future situation.


The model-driven engineering (MDE) methodology promotes the use of abstraction in software engineering.
However, information concerning the grid, such as loads, is not always known with absolute confidence.
Through the thesis, this lack of confidence about data is referred to as \textbf{data uncertainty}.
They are approximated from the measured consumption and the grid topology.
This topology is inferred from fuse states, which are set by technicians after their services on the grid.
As humans are not error-free, the topology is therefore not known with absolute confidence.
This data uncertainty is propagated to the load through the computation made.
If it is neither present in the model nor not considered by the adaptation process, then the adaptation process may make suboptimal reconfiguration decision.

The literature refers to systems which provide adaptation capabilities as dynamically adaptive systems (DAS).
One challenge in the grid is the phase difference between the monitoring frequency and the time for actions to have measurable effects.
Action with no immediate measurable effects are named \textbf{long-term action}.
On the one hand, an incident should be detected in the next minutes.
On the other hand, a reconfiguration action can take up to several hours.
For example, when a tree falls on a cable and cuts it during a storm, the grid manager should be noticed in real time.
The reconfiguration of the grid, to reconnect as many people as possible before replacing the cable, is done by technicians who need to use their cars to go on the reconfiguration places.
In a fully autonomous adaptive system, the reasoning process should consider the ongoing actions to avoid repeating decisions.

\bigskip
\paragraphit{Problematic}
\textbf{Data uncertainty and long-term actions are not specific to smart grids.}

First, data are, almost by definition, uncertain and developers work with estimates in most cases.
Hardware sensors have by construction a precision that can vary according to the current environment in which they are deployed.
A simple example is the temperature sensor that provides a temperature with precision to the nearest degree.
Software sensors approximate also values from these physical sensors, which increases the uncertainty.
For example, CPU usage is computed counting the cycle used by a program.
As stated by Intel, this counter is not error-prone\footnote{\url{https://software.intel.com/en-us/itc-user-and-reference-guide-cpu-cycle-counter}}.

Second, it always exists a delay between the moment where a suboptimal state is detected by the adaptation process and the moment where the effects of decisions taken are measured.
This delayed is due to the time needed by a computer to process data and, eventually, to send orders or data through networks.
For example, migrating a virtual machine from a server to another one can take several minutes.

\textbf{Through this thesis, we argue that this data uncertainty and this delay cannot be ignored for all dynamic adaptive systems.}
To know if the data uncertainty should be considered, stakeholders should wonder \textbf{if this data uncertainty affects the result of their reasoning process, like adaptation}.
Regarding long-term actions, they should verify \textbf{if the frequency of the monitoring stage is lower than the time of action effects to be measurable}.
These characteristics are common to smart grids, cloud infrastructure or cyber-physical systems in general.

\bigskip
\paragraphit{Challenge}
These problematics come with different challenges concerning the representation of the knowledge for DAS.
The global challenge address by this thesis is: \textbf{how to represent the uncertain knowledge allowing to efficiently query it and to represent ongoing actions in order to improve adaptation processes?}

\bigskip
\paragraphit{Vision}
\textbf{This thesis defends the need for a unified modelling framework which includes, despite all traditional elements, temporal and uncertainty as first-class concepts.}
Therefore, a developer will be able to abstract information related to the adaptation process, the environment as well as the system itself.

Concerning the adaptation process, the framework should enable abstraction of the actions, their context, their impact, and the specification of this process (requirements and constraints).
It should also enable the abstraction of the system environment and its behaviour.
Finally, the framework should represent the structure, behaviour and specification of the system itself as well as the actuators and sensors.
All these representations should integrate the data uncertainty existing.

\bigskip
\paragraphit{Contributions}
Towards this vision, this document presents two contributions: a temporal context model and a language for uncertain data.

The temporal context model allows abstracting past, ongoing and future actions with their impacts and context.
First, a developer can use this model to know what the ongoing actions, with their expect future impacts on the system, are.
Second, she/he can navigate through past decisions to understand why they have been made when they have led to a sub-optimal state.

The language, named \langName, integrates data uncertainty as a first-class citizen.
It allows developers to attach data with a probability distribution which represents their uncertainty.
Plus, it mapped all arithmetic and boolean operators to uncertainty propagation operations. 
And so, developers will automatically propagate the uncertainty of data without additional effort, compared to an algorithm which manipulates certain data.

\bigskip
\paragraphit{Validation}
Each contribution has been evaluated separately. 
First, the context model has been evaluated through the performance axis.
The dissertation shows that it can be used to represent the Luxembourg smart grid.
The model also provides an API which enables the execution of query for diagnosis purpose.
In order to show the feasibility of the solution, it has also been applied to the use case provided by the industrial partner.

Second, the language has been evaluated through two axes: its ability to detect errors at development time and its expressiveness.
\langName{} can detect errors in the combination of uncertain data earlier than state-of-the-art approaches.
The language is also as expressive as current approaches found in the literature.
Moreover, we use this language to implement the load approximation of a smart grid furnished by an industrial partner, Creos S.A.\footnote{Creos S.A. is the power grid manager of Luxembourg. \url{https://www.creos-net.lu}}.

\bigskip
\bigskip
\bigskip
\highlightbox[Keywords]{dynamically adaptive systems, knowledge representation, model-driven engineering, uncertainty modelling, time modelling}


