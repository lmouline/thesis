\chapter*{Abstract}

\boxed{Vision}
{
	%% Context & Problem
	As state-of-the-art techniques fail to model efficiently the evolution and the uncertainty existing in dynamically adaptive systems, the adaptation process makes suboptimal decisions.
	%% Vision
	To tackle this challenge, modern modeling frameworks should efficiently encapsulate time and uncertainty as first-class concepts.
}
\bigskip

%% Context
\paragraphit{Context}
Smart grid approach introduces information and communication technologies into traditional power grid to cope with new challenges of electricity distribution.
Among them, one challenge is the resiliency of the grid: how to automatically recover from any incident such as overload?
These systems therefore need a deep understanding of the ongoing situation which enables reasoning tasks for healing operations.
\textbf{Abstraction} is a key technique that provided an illuminating description of systems, their behaviors, and/or their environments alleviating their complexity.
\textbf{Adaptation} is a cornerstone feature that enables reconfiguration at runtime for optimizing software to the current and/or future situation.

Abstraction technique is pushed to its paramountcy by the model-driven engineering (MDE) methodology.
However, information concerning the grid, such as loads, is not always known with absolute confidence.
Through the thesis, this lack of confidence about data is referred to as \textbf{data uncertainty}.
They are approximated from the measured consumption and the grid topology.
This topology is inferred from fuse states, which are set by technicians after their services on the grid.
As humans are not error-free, the topology is therefore not known with absolute confidence.
This data uncertainty is propagated to the load through the computation made.
If it is neither present in the model nor not considered by the adaptation process, then the adaptation process may make suboptimal reconfiguration decision.

The literature refers to systems which provide adaptation capabilities as dynamically adaptive systems (DAS).
One challenge in the grid is the phase difference between the monitoring frequency and the time for actions to have measurable effects.
Action with no immediate measurable effects are named \textbf{delayed action}.
On the one hand, an incident should be detected in the next minutes.
On the other hand, a reconfiguration action can take up to several hours.
For example, when a tree falls on a cable and cuts it during a storm, the grid manager should be noticed in real time.
The reconfiguration of the grid, to reconnect as many people as possible before replacing the cable, is done by technicians who need to use their cars to go on the reconfiguration places.
In a fully autonomous adaptive system, the reasoning process should be considered the ongoing actions to avoid repeating decisions.

%% Problematic 
\bigskip
\paragraphit{Problematic}
\textbf{Data uncertainty and delayed actions are not specific to smart grids.}

First, data are, almost by definition, uncertain and developers always work with estimates.
Hardware sensors have by construction a precision that can vary according to the current environment in which they are deployed.
A simple example is the temperature sensor which provides a temperature with precision of one Celsius degree.
Software sensors approximate also values from these physical sensors and accentuate and have their uncertainty.
For example, CPU usage is computed counting the cycle used by a program.
As stated by Intel, this counter is not error-free~\footnote{https://software.intel.com/en-us/itc-user-and-reference-guide-cpu-cycle-counter}.

Second, it always exists a delay between the moment where a suboptimal state is detected by the adaptation process and the moment where the effects of decisions taken are measured.
This delayed is due to the time needed by a computer to process a send and, eventually, to send orders or data through networks.
For example, migrating a virtual machine from a server to another one can take several minutes.

\textbf{Through this thesis, I argue that this data uncertainty and this delay cannot be ignored for all dynamic adaptive systems.}
To know if the data uncertainty should be considered, stakeholders should wonder \textbf{if this data uncertainty affects the result of their reasoning process, like adaptation}.
Regarding delayed action, they should verify \textbf{if the frequency of the monitoring stage is lower than the time of action effects to be measurable}.
These characteristics are common to smart grids, cloud infrastructure or cyber-physical systems in general.

%% Challenge
\bigskip
\paragraphit{Challenge}
These problematics come with different challenges concerning the representation of the knowledge for DAS.
Solutions should be bring to developers to ease their definition and manipulation of uncertain and temporal data.
In order to respect the real-time constraint of DAS, the read, write and process of this knowledge should be efficient in term of resources (memory, CPU) used and execution time.

%% Vision 
\bigskip
\paragraphit{Vision}
\textbf{This thesis defends the need of a unified modeling framework which includes, despite all traditional elements, temporal and uncertainty as first-class concepts.}
Therefore, a developer will be able to easily abstract information related to the adaptation process, the environment as well as the system itself.
This frameworks should enable efficient read and write operations and should provide data structures efficient to process in order to enable a real-time reasoning.

Concerning the adaption process, the framework should enable easy abstraction of the actions with their context and impact as well as the specification of this process (requirements and constraints).
Concerning the environment of the system, the framework should enable easy abstraction of its behavior and its structure.
Finally, the framework should represent the structure, behavior and specification of the system itself as well as the actuators and sensors.

%% Contributions which support the vision
\bigskip
\paragraphit{Contributions}
Towards this vision, two contributions have been proposed: a temporal context model and a language for uncertain data.

The temporal context model allows to abstract past, on going and future actions with their impacts and context.
First, a developer can use this model to know what are the ongoing actions with their expect future impacts on the system.
Second, she/he can navigate through past decisions to understand why the system has taken some decisions which lead to a sub-optimal state.

The language, named \langName, integrates data uncertainty as a first-class concept.
It allows developers to attach data with a probability distribution which represents their uncertainty.
Plus, it mapped all arithmetic and boolean operators to uncertainty propagation operation. 
And so, developers will automatically propagate the uncertainty of data without additional effort, compared to an algorithm which manipulates certain data.

%% Validation
\bigskip
\paragraphit{Validation}
Each contribution have been evaluated separately. 
The language has been evaluated through two axis: its ability to detect errors at development time and its expressiveness.
Ain'tea can detect [...] and it is as expressive as any state of the art solution.
Moreover, we use this language to implement the load approximation of a smart grid furnished by an industrial partner, Creos SA.

The context model, has been evaluated through the performance access.
I show that it can be used to [....]


%% Perspective FW
%- Push the contrib. into GreyCat
\bigskip
\bigskip
\bigskip
\boxed{Keywords}{dynamically adaptive systems, knowledge representation, model-driven engineering, uncertainty modeling, time modeling}


