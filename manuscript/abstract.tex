\chapter*{Abstract}

\boxed{Vision defended}
{
	%% Context & Problem
%    Due to the lack of efficient temporal structures which also abstract uncertainty, adaptation processes for dynamically adaptive systems make sub-optimal decisions.
	As state of the art techniques fail at efficiently modeling the dynamic and the uncertainty existing in dynamically adaptive systems, the adaptation process make suboptimal decision.
	%% Vision
	To tackle this challenge, modeling frameworks should encapsulate time and uncertainty as first-class concepts with efficient accesses (read, write), efficient process and effortless manipulation.
}

%% Context
Advances in software systems go to ...... dynamically adaptive systems (DAS).
As an example, smart grid systems are .....
Smart grid approach introduces information and communication technologies into traditional power grid in order to cope with new challenges of electricity distribution.
One example of these challenges is the increasing number of electrical vehicles that, combined with the 7pm consumption peak, will lead to grid overloading, according to the current estimations.
The Luxembourgish grid manager estimates that the grid will be overloaded with 20$\sim$30\% of vehicle are electric and charging during the peak hour.

One investigated approach to tackle these challenges is to define the smart grid as a dynamically adaptive systems (DAS).
By analyzing its context, an adaptation process detects any suboptimal state.
For the grid manager, a suboptimal state is a state with at least one overloading cable.
This process then triggers a reconfiguration following specifications (requirements and constraints) in order to optimize the state.
For example, the grid manage can modify the path taken by the power, also referred to as topology, to avoid overloading cables.

However, reconfiguring the low voltage (cables that are directly connected to the end users) grid is still a task made by humans: technicians take their car, go the re-configuration point and modify manually the grid through fuses.
Several minutes have then passed between the moment where the incident is detected and the moment where the grid is reconfigured whereas an incident should be detected in real-time (for the grid, real-time means in the minute).
In this thesis, I call these kind of actions \textit{delayed actions}.
Moreover, loads on the grid is not known with absolute confidence.
They are approximated from the measured consumption and the grid topology.
This topology is inferred  from fuse states, which are set by technicians after their intervention on the grid.
As human are not error-free, the topology is therefore not known with absolute confidence.
This uncertainty on data is propagated to the load through the computation made.

%% Problematic 
\bigskip
Data uncertainty and delayed actions are not a specificity of smart grids.

Data are, almost by definition, uncertain and developers work with estimates.
Hardware sensors have by construction a precision that can vary according to the current environment in which they are deployed.
A simple example is the temperature sensor which provides a temperature with a precision of one Celsius degree.
Software sensors approximates also values from these physical sensors and accentuate and have their uncertainty.
For example, CPU usage is computed counting the cycle used by a program.
As stated by Intel, this counter is not error-free~\footnote{https://software.intel.com/en-us/itc-user-and-reference-guide-cpu-cycle-counter}.

It always exist a delay between the moment where a suboptimal state is detected by the adaptation process and the moment where the effect of decision taken are measured.
This delayed is due to the time needed by a computer to process a send and, eventually, to send orders or data through networks.
For example, migrating a virtual machine from a server to another one can take several minutes.

Through this thesis, I argue that this uncertainty and this delay cannot be ignored for all dynamic adaptive systems.
In addition to the smart grid, these characteristics are also important for cloud or cyber-physical systems.

%% Challenge
\bigskip
These problematics come with different challenges concerning the representation of the knowledge for DAS.
Solutions should be bring to developers to ease their definition and manipulation of uncertain and temporal data.
In order to respect the real-time constraint of DAS, the read, write and process of this knowledge should be efficient in term of resources (memory, CPU) used and execution time.

%% Vision 
\bigskip
This thesis defends the need of a unified modeling framework which includes, despite all traditional elements, temporal and uncertainty as first-class concepts.
Therefore, a developer will be able to easily abstract information related to the adaptation process, the environment as well as the system itself.
This frameworks should enable efficient read and write operations and should provide data structures efficient to process in order to enable a real-time reasoning.

Concerning the adaption process, the framework should enable easy abstraction of the actions with their context and impact as well as the specification of this process (requirements and constraints).
Concerning the environment of the system, the framework should enable easy abstraction of its behavior and its structure.
Finally, the framework should represent the structure, behavior and specification of the system itself as well as the actuators and sensors.

%% Contributions which support the vision
\bigskip
Towards this vision, two contributions have been proposed: a temporal context model and a language for uncertain data.

The temporal context model allows to abstract past, on going and future actions with their impacts and context.
First, a developer can use this model to know what are the ongoing actions with their expect future impacts on the system.
Second, she/he can navigate through past decisions to understand why the system has taken some decisions which lead to a sub-optimal state.

The language, named \langName, integrates data uncertainty as a first-class concept.
It allows developers to attach data with a probability distribution which represents their uncertainty.
Plus, it mapped all arithmetic and boolean operators to uncertainty propagation operation. 
And so, developers will automatically propagate the uncertainty of data without additional effort, compared to an algorithm which manipulates certain data.

%% Validation
\bigskip
Each contribution have been evaluated separately. 
The language has been evaluated through two axis: its ability to detect errors at development time and its expressiveness.
Ain'tea can detect [...] and it is as expressive as any state of the art solution.
Moreover, we use this language to implement the load approximation of a smart grid furnished by an industrial partner, Creos SA.

The context model, has been evaluated through the performance access.
I show that it can be used to [....]


%% Perspective FW
%- Push the contrib. into GreyCat
\bigskip
\boxed{Keywords}{dynamically adaptive systems, knowledge representation, model-driven engineering, uncertainty modeling, time modeling}


