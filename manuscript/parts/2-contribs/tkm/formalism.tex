\section{Knowledge formalization}
\label{sec:tkm:k-formalism}
 
We consider \gls{knowledge} to be the association of \gls{context} information, \glspl{requirement}, and \gls{action} information, all in one global and unified model.
 While \gls{context} information captures the state of the system environment and its surroundings, the system \glspl{requirement} define the constraints that the system should satisfy along the way. 
 \Glspl{action}, on the other hand, are meant to reach the goals of the system.
  
 In this section, we provide a formalization of the \gls{knowledge} used by adaptation processes based on a temporal graph. 
Due to the complexity and interconnectivity of system entities, graph data representation is an appropriate way to represent the \gls{knowledge}. 
Augmented with a temporal dimension, temporal graphs are then able to symbolize the evolution of system entities and states over time. 
We benefit from the well-defined graph manipulation operations, namely temporal graph pattern matching and temporal graph relations to represent the traceability links between the \glspl{decision} made and their \glspl{circumstance}.

Before describing this formalism, we describe the semantics used for the temporal axis.
Then, we exemplify the knowledge formalism using the Luxembourg smart grid use case, detailed in Section~\ref{sec:tkm:intro:uc}.

\subsection{Formalization of the temporal axis}
\label{sec:tkm:k-formalism:timeAxis}

\begin{figure}
   \centering
	\includegraphics[width=\textwidth]{img/chapt-tkm/formalism/formalismeTime}
	\caption{Time definition used for the knowledge formalism}
	\label{fig:tkm:formalismeTime}
\end{figure}

The formalism described below has been defined with two goals in mind.
First, the definition of the time space should allow the distinction between past and future. 
Making this distinction enables the differentiation between measured data and predicted (or planned data).
Second, it should permit the definition of the life cycle of an element of the \gls{knowledge}, which can be seen as a succession of states with a validity period that should not overlap each other.

Time space $T$ is considered as an ordered discrete set of time points non-uniformly distributed. 
As depicted in Figure~\ref{fig:tkm:formalismeTime}, this set can be divided into 3 different subsets $T = T_{past} \cup \{t\} \cup T_{future}$, where:   \begin{itemize}  \item $T_{past}$ is the subdomain \{$t_{0}$;$t_{1}$;\ldots;$t_{current-1}$\}  representing graph data history starting from $t_0$, the oldest point, until the current time, $t$, excluded.

\item $\{t\}$ is a singleton representing the current time  point  \item $T_{future}$ is subdomain \{$t_{current+1};\ldots;t_{\infty}$\} representing future time points  \end{itemize}
The three domains depend completely on the current time \{t\} as these subsets slide as time passes. 
At any point in time, these domains never overlap: $T_{past} \cap \{t\} = \emptyset$, $T_{future} \cap \{t\} =  \emptyset$, and $T_{past} \cap T_{future} = \emptyset$.
The definition of these three subsets reaches the first goal.

In addition, there is a right-opened time interval $I \in T \times T$ as $[t_s, t_e)$ where $t_e - t_s > 0$.
In English words, it means that the interval should represent at least one time point and should follow the time order. 
For any $i \in I$, $start(i)$ denotes its lower bound and $end(i)$ its upper bound.
As detailed in Section~\ref{sec:tkm:k-formalism:formalism}, these intervals are used to define the validity period for each node of the graph (our second goal).

Figure~\ref{fig:tkm:formalismeTime} displays an example of a time space $T_1 = \{t_0, t_1, t_2, t_3, t_4, t_5, t_6, t_7\}$.
In this case, the current time is $t = t_4$.
According to the definition of the past subset ($T_{past}$) and the future one ($T_{future}$), there is: $T_{past1} =  \{t_0, t_1, t_2, t_3\}$ and $T_{future1} = \{t_5, t_6, t_7\}$.
Two intervals have been defined on $T_1$, namely $I_1$ and $I_2$.
The first one starts at $t_2$ and ends at $t_5$ and the last one is defined from $t_6$ to $t_7$.
As shown with $I_1$, an interval could be defined on different subsets, in this case it is on all of them ($T_{past}$, $t$, and $T_{future}$).

\subsection{Formalism of the knowledge}
\label{sec:tkm:k-formalism:formalism}
 
\paragraph{Graph definition}
First, let $K$ be an adaptive process over a system \gls{knowledge} represented by a graph such as $K = (N, E)$, comprising a set of nodes $N$ and a set of edges $E$.
Nodes represent any element of the knowledge (context, actions, \etc) and edges represent their relationships.
Nodes have a set of attribute values: $\forall n \in N, n = (id, P)$, where $P$ is the set of key-value attributes.
An attribute value has a type (numerical, boolean, \ldots). 
Every relationship $e \in E$ can be considered as a couple of nodes with a label $(n_s, n_t, label) \in N \times N$, where $n_s$ is the source node and $n_t$ is the target node.

\paragraph{Adding the temporal dimension}

\begin{figure}
   \centering
	\includegraphics{img/chapt-tkm/formalism/validityExample}
	\caption{Evolution of a temporal graph over time}
	\label{fig:tkm:validityEx}
\end{figure}

In order to augment the graph with a temporal dimension, the relation $V^T$ is added.
So now the knowledge $K$ is defined as a temporal graph such as $K = (N, E, V^T)$.

A node is considered valid either until it is removed or until one of its attributes value changes. 
In the latter case, a new node with the updated value is created.
Whilst, an edge is considered valid until either its source node and target node are valid, or until the edge itself is removed.
Otherwise, nodes and edges are considered invalid.
The temporal validity relation is defined as $V^T: N \cup E \rightarrow I$.
It takes as a parameter a node or an edge ($k \in N \cup E$) and returns a time interval ($i \in I$, \cf Section~\ref{sec:tkm:k-formalism:timeAxis}) during which the graph element is valid.

Figure~\ref{fig:tkm:validityEx} shows an example of a temporal graph $K_1$ with five nodes ($n_1$, $n_2$,$n_3$, $n_4$, and $n_5$) and three edges ($e_1$, $e_2$, and  $e_3$) over a lifecycle from $t_1$ to $t_3$.
In this way, $K_1$ equals $(\{n_1, n_2, n_3, n_4, n_5\}, \{e_1, e_2, e_3\}, V^{T}_1)$.
Let's assume that the graph is created at $t_1$.
As $n_1$ is modified at $t_2$, its validity period starts at $t_1$ and ends at $t_2$: $V^{T}_1(n_1) = [t_1, t_2)$.
$n_2$ and $n_3$ are not modified; their validity period thus starts at $t_1$ and ends at $t_\infty$: $V^{T}_1(n_2) = V^{T}_1(n_3) = [t_1, t_\infty)$.
Regarding the edges, the first one, $e_1$, is between $n_1$ and $n_2$ and the second one, $e_2$ from $n_2$ to $n_3$.
Both are created at $t_1$.
As $n_1$ is being modified at $t_2$, its validity period goes from $t_1$ to $t_2$:  $V^{T}_1(e_1) = [t_1, t_2)$.
$e_2$ is deleted at $t_3$.
Its validity period is thus equal to: $V^{T}_1(e_2) = [t_1, t_3)$.

\paragraph{Lifecycle of a knowledge element}
One node represents the state of exactly one knowledge element during a period named the validity period.
The lifecycle of a knowledge element is thus modeled by a unique set of nodes.
By definition, the validity periods of different nodes cannot overlap.
A same time period cannot be represented by two different nodes, which could create inconsistency in the temporal graph.

To keep track of this knowledge element history, the $Z^T$ relation is added to the graph formalism: $K = (N, E, V^T, Z^T)$.
It serves to trace the updates of a given knowledge element at any point in time. 
This relation can also be seen as a temporal identity function which takes as parameters a given node $n \in N$ and a specific time point $t \in T$, and returns the corresponding node at that point. 
Formally, $Z^T: N \times T \rightarrow N$. 

In order to consider this new relation in the example presented in Figure~\ref{fig:tkm:validityEx}, the definition of $K_1$ is modified to $K_1 = (\{n_1, n_2, n_3, n_4, n_5\}, \{e_1, e_2, e_3\}, V^{T}_1, Z^{T}_1)$
In Figure~\ref{fig:tkm:validityEx}, let's imagine that $n_1$, $n_4$, and $n_5$ represent the same knowledge element $k_e$.
The lifecycle of $k_e$ is thus:
\begin{itemize}
    \item $n_1$ for period $[t_1, t_2)$,
    \item $n_4$ for period $[t_2, t_3)$,
    \item $n_5$ for period $[t_3, t_\infty)$.
\end{itemize}

Let $t_1'$ be a timepoint between $t_1$ and $t_2$.
When one wants to resolve the node representing the knowledge element at $t_1'$, she or he gets $n_1$ node, no matter of the node input ($n_1$, $n_4$, or $n_5$): $Z^{T}_1(n_4, t_1) = n_1$.
On the other hand, applying the same relation with another node ($n_2$ or $n_3$) returns another node.
For example, if $n_2$ and $n_3$ do not belong to the same knowledge element, then it will return the node given as input, for example $Z^{T}_1(n_2, t_1) = n_2$.

\paragraph{Knowledge elements stored in nodes}
Nodes are used to store the different knowledge elements: context, requirements and actions.
The set of nodes $N$ is thus split in three subsets: $N = C \cup R \cup A$ where $C$ is the set of nodes which store context information, $R$ a set of nodes for requirement information and $A$ a set of nodes for action information.

Actions define processes that indirectly impact the context: they will change the behaviour of the system, which will be reflected in the context information.
Requirements are also processes that are continuously run over the system in order to check the specifications.
Here, the purpose of the $A$ and $R$ subset is not to store these processes but to list them.
It can be thought as a catalogue of actions and requirements, with their history.

Using a high-level overview, these processes can be depicted as: taking the knowledge as input, perform tasks, and modify this knowledge as output.
As detailed in the next two paragraphs, action executions and requirement analysis can be formalized by relations.

\paragraph{Temporal queries for requirements}
At the current state, the formalism of the knowledge $K$ does not contain any information regarding the requirement analysis.
To overcome this, system requirements analysis $R_A$ are added such as $K = (N, E, V^T, Z^T, R_A)$.
$R_A$ is a set of couples composed of patterns $P_{[t_j,t_k]}(K)$ and requirements $R$ over these patterns: $R_P = {P \cup R}$. 

$P_{[t_j, t_k]}$ denotes a temporal graph pattern, where $t_j$ and $t_k$ are the lower and upper bound of the time interval respectively.
$P_{[t_j, t_k]}$ is the result of a function which takes the knowledge and an interval as input: $P_{[t_j, t_k]} : K \times I$.
The time interval can be either fixed (absolute), \ie both bounds are precisely defined, or sliding (relative), \ie the upper bound is computed from the lower bound.
For example, $P_{[t_0, t_4]}$ is considered as fixed and $P_{[t_0, t_0+4]}$ is considered as relative.
Each element of the pattern should be valid for at least one timepoint: $\forall~p \in P_{[t_j,t_k)}, V^T(e) \cap [t_j,t_k) \neq \emptyset$.
Patterns can be seen as temporal subgraphs of $K$, with a time limiting constraint coming in the form of a time interval.

\paragraph{Temporal relations for actions}
Like for $R_A$, the knowledge $K$ needs to be augmented with action executions $A_E$: $K = (N, E, V^T, Z^T, R_A, A_E)$.
Actions executions $A_E$ can be regarded as a couple $(A, A_F)$, where $A$ is the action that is executed and $A_F$ a set of relations or isomorphisms mapping a source temporal graph pattern $P_{[t_j, t_k]}$ to a target one $P_{[t_l, t_m]}$,  $A_F : K \times I \rightarrow K \times I$.

The left-hand side of the $A_F$ relation depicts the temporal graph elements over which an action is applied.
Every relation may have a set of application conditions. 
They describe the circumstances under which an action should take place. 
These application conditions are either positive, should hold, or negative, should not hold. 
Application conditions come in the form of temporal graph invariants.  
The side effects of these actions are represented by the right-hand side. 

Finally, we associate to $A_E$ a temporal function $E_{A_E}$ to determine the time interval at which an action has been executed. 
Formally, $E_{A_E}: A_E \rightarrow I$.

\paragraph{Temporal relations for decisions}
Finally, the knowledge formalism needs to include the last, but not the least, element: decisions made by the adaptation, $K = (N, E, V^T, Z^T, R_A, A_E, D)$
While the source of relations in $D$ represents the state before the execution of an action, the target shows its impact on the \gls{context}. 
Its intent is \textbf{to trace back impacts of action executions to the decisions they originated from}.  

A decision in ${D}$ is defined as a set of executed actions, \ie a subset of ${A_E}$, combined with a set of requirement analysis, \ie a subset of ${R_A}$.
Formally, ${D} = \{\ {A_D \cup R_D}~|~{A_D}  \subseteq A_E, R_A \subseteq R_P\}$.
We assume that each action should result from only one decision: $\forall a \in {A}, \forall d1, d2 \in {D}~|~a \in d1 \wedge a \in d2 \rightarrow d1 = d2$.

The temporal function $E_{A_E}$ is extended to decisions in order to represent the execution time: $E_{A_E}: (A \cup D) \rightarrow I$.
For decision, the lower bound of the interval corresponds to the lowest bound of the action execution intervals.
Following the same principle, the upper bound of the interval corresponds to the uppermost bound of the action execution intervals.
Formally, $\forall d \in D \rightarrow E_{A_E}(d) = [l,u)$, where $l = \displaystyle \min_{a \in A_d} \{E_{A_E}(a)[start]\}$ and $u = \displaystyle \max_{a \in A_d} \{E_{A_E}(a)[end]\}$.

\paragraph{Sum up}
Knowledge of an adaptive system can be formalism with a temporal graph such as $K = (N, E, V^T, Z^T, R_A, A_E, D)$, wherein:
\begin{itemize}
    \item $N$ is a set of nodes to represent the different information (context, actions and requirements)
    \item $E$ is a set of edges which connects the different nodes,
    \item $V^T$ is a temporal relation which defines the temporal validity of each element,
    \item $Z^T$ is a relation to track the history of each knowledge elements,
    \item $R_A$ is a relation that defines the different requirements processes,
    \item $A_E$ is a relation that defines the different action processes,
    \item $D$ is a set of action executions, which result from the same decision, and requirement analysis.
\end{itemize}

Decisions $D$ can allow adaptation processes to reason over ongoing and future executions of decisions.
Moreover, it allows tracing the state of the knowledge before and after the decision has been or is executed, thanks to its $A_D$ component.
Plus, it represents which action has been used for this.
Thanks to the $R_A$ relation, one can access the requirements at the root of the decision and the state of the knowledge used by this requirement.

In the next section, we exemplify this formalism over our case study.


\subsection{Application on the use case}

\begin{figure}
	\centering
	\includegraphics[width=\linewidth]{img/chapt-tkm/formalism/application}
	\caption{Application of the formalism with a temporal graph that represents the knowledge of the smart grid described in Section~\ref{sec:tkm:intro:uc}}
	\label{fig:tkm:k-formalism:application}
\end{figure}

In this section we apply the formalism described on the use case presented in Section~\ref{sec:tkm:intro:uc}.

Let $K_{SG}$ be the temporal graph that represents the knowledge of this adaptive system: $K_{SG} = (N_{SG}, E_{SG}, V^T_{SG}, Z^T_{SG}, R_{P_{SG}}, A_{P_{SG}}, D_{SG})$.
Figure~\ref{fig:tkm:k-formalism:application} shows the nodes and edges of this knowledge.

\paragraph{Description of \pmb{$N_{SG}$}}
$N_{SG}$ is divided into three subsets: $C_{SG}$, $R_{SG}$ and $A_{SG}$.
$R_{SG}$ contains one node, $R_1$ in Figure~\ref{fig:tkm:k-formalism:application}, which represents the requirement of this example (minimizing the number of overloads): $R_{SG} = \{R_1\}$.
Two nodes, $A_1$ and $A_2$, belong to $A_{SG}$: $A_{SG} = \{ A_1, A_2\}$.
They represent the two actions of this example, respectively decreasing and increasing amps limits.
Regarding the context $C_{SG}$, there are three nodes to represent the three smart meters ($M_1$, $M_2$, and $M_3$), one for the substation ($S_1$), two for the fuses ($F_1$ and $F_2$), one for the dead-end cabinet ($E_1$), one for the cable ($C_1$) and one node per consumption value received ($V_i$): $C_{SG} = \{M_1, M_2, M_3, S_1, F_1, F_2, E_1, C_1 \} \cup \{ V_i | i \in [1..9]\}$.

According to the scenario, except for nodes to store consumption values, the other nodes are created at $t_0$ and are never modified.
Therefore, their validity period starts at $t_0$ and never ends: $\forall n \in A_{SG} \cup R_{SG} \cup \{M_1, M_2, M_3, S_1, F_1, F_2, E_1, C_1\}, V^T_{SG}(n) = [t_0, t_\infty)$.
Considering the consumption values, all the nodes represent the history of the values for the three smart meters.
In other words, there are three knowledge elements: the consumption measured for each meter.
Let $C_i$ notes the consumption measured by the smart meter $M_i$.
As shown in Figure~\ref{fig:tkm:k-formalism:application}, there is:
\begin{itemize}
	\item $C_1$ of $M_1$ is represented by $\{V_1, V_4, V_7\}$,
	\item $C_2$ of $M_2$ is represented by $\{V_2, V_5, V_8\}$,
	\item $C_3$ of $M_3$ is represented by $\{V_3, V_5, V_9\}$.
\end{itemize}
Taking $C_2$ as an example, $V_2$ is the initial consumption value, replaced by $V_5$ at $t_2$, itself replaced by $V_8$ at $t_3$. 
Applying the $V_{SG}^T$ on these different values, results are thus:
\begin{itemize}
	\item $V_{SG}^T(V_2) = [t_1, t_2)$,
	\item $V_{SG}^T(V_5) = [t_2, t_3)$,
	\item $V_{SG}^T(V_8) = [t_3, t_\infty)$.
\end{itemize}
These validity periods are shown in Figure~\ref{fig:tkm:validityC2}.
As meters send the new consumption values at the same time, this example can also be applied to $C_1$ and $C_3$.

\begin{figure}
	\centering
	\subfloat[Consumption values $C_2$] {
		\includegraphics[width=0.4\linewidth]{img/chapt-tkm/formalism/validitySchemaC2}
		\label{fig:tkm:validityC2}
	}
	\hfil
	\subfloat[Edges linking the meter node $M_2$ to its consumption values $C_2$]{
		\includegraphics[width=0.4\linewidth]{img/chapt-tkm/formalism/validitySchemaC2Edges}
		\label{fig:tkm:validityC2Edges}
	}
	\caption{Validity periods of consumption values and their edges to the smart meter $M_2$}
\end{figure}

From these validity periods, the $Z^T_{SG}$ can be used to navigate to the different values over time.
Let's continue with the same example, $C_2$.
In order to get the evolution of the consumption value $C_2$, given the initial one, one will use the $Z^T_{SG}$ relation:
\begin{itemize}
	\item $Z^T_{SG}(V_2, t_{s1}) = \emptyset$, where $t_0 \leqslant t_{s1} < t_1$
	\item $Z^T_{SG}(V_2, t_{s2}) = V_2$, where $t_1 \leqslant t_{s2} < t_2$
	\item $Z^T_{SG}(V_2, t_{s3}) = V_5$, where $t_2 \leqslant t_{s3} < t_\infty$.
	\item $Z^T_{SG}(V_2, t_{s4}) = V_8$, where $t_2 \leqslant t_{s4} < t_\infty$.
\end{itemize}


\paragraph{Description of \pmb{$E_{SG}$}}
In this example, edges are used to store the relationships between the different context elements.
For example, the edge between the substation $S_1$ and the fuse $F_1$ allow representing the fact that the fuse is physically inside the substation.
Another example, edges between the cable $C_1$ and the meters $M_1$, $M_2$ and $M_3$ represent the fact that these meters are connected to the smart grid through this cable.

One may consider that relations (validity, $Z^T$, decisions, action executions and requirements analysis) will be stored as edges.
But this decision is let to the implementation part of this formalism.

In our model, only consumption values ($V_i$ nodes) are modified over time.
Plus, since the scenario does not imply any edge modifications, only those between meters and values are modified.
The edge set contains thus sixteen edges: $E_{SG} = \{e_i \mid i \in [1..16] \}$.

By definition, the unmodified edges have a validity period starting from $t_0$ and never ends: $\forall i \in [1..7], V^T_{SG}(e_i) = [t_0, t\infty)$.
The history of the three knowledge elements that represent consumption values do not only impact the nodes which represent the values but also the edges between those nodes and the meters ones:
\begin{itemize}
    \item $C_1$ impacts edges between $M_1$ and $V_1$, $V_4$, and $V_7$, \ie $\{e_8, e_{11}, e_{14}\}$,
    \item $C_2$ impacts edges between $M_2$ and $V_2$, $V_5$, and $V_8$, \ie $\{e_9, e_{12}, e_{15}\}$,
    \item $C_3$ impacts edges between $M_3$ and $V_3$, $V_6$, and $V_9$, \ie $\{e_{10}, e_{13}, e_{16}\}$.
\end{itemize}

Continuing with $C_2$ as an example, the initial edge value is $e_9$ from $t_1$, which is replaced by $e_{12}$ from $t_2$, itself replaced by $e_{15}$ from $t_2$.
The validity relation, applied to these edges, thus returns:
\begin{itemize}
    \item $V^T_{SG}(e_9) = [t_1, t_2) = V^T_{SG}(V_2)$,
    \item $V^T_{SG}(e_{12}) = [t_2, t_3) = V^T_{SG}(V_5)$,
    \item $V^T_{SG}(e_{15}) = [t_3, t_\infty) = V^T_{SG}(V_8)$,
\end{itemize}

These validity periods are depicted in Figure~\ref{fig:tkm:validityC2Edges}.
As they are driven by those of consumption values ($V_2$, $V_5$, and $V_8$), they are equal.

As for nodes, the $Z^T_{SG}$ relation can navigate over time through these values.
For example, to get the history of the edges between the consumption value $C_2$ and the meter represented by $M_2$, one can apply the $Z^T_{SG}$ relation as follows:
\begin{itemize}
    \item $Z^T_{SG}(e_9, t_{s1}) = \emptyset$, where $t_0 \leqslant t_{s1} < t_1$
    \item $Z^T_{SG}(e_9, t_{s2}) = e_9$, where $t_1 \leqslant t_{s2} < t_2$,
    \item $Z^T_{SG}(e_9, t_{s3}) = e_12$, where $t_2 \leqslant t_{s3} < t_3$,
    \item $Z^T_{SG}(e_9, t_{s4}) = e_15$, where $t_3 \leqslant t_{s4} < t_\infty$.
\end{itemize}


\paragraph{Description of \pmb{$D_{SG}$}, \pmb{$A_{E_{SG}}$}, and \pmb{$R_{A_{SG}}$}}
As described in the scenario (cf. Section~\ref{sec:tkm:intro:uc}), the requirement analysis detects that $t_1$ the requirement is broken.
The adaptation process will thus apply the \textquote{decreasing amps limits} action on the three meters.
Following Example 2 detailed in Section~\ref{sec:tkm:intro:motiv:delayed_action}, we consider that the action will impact the consumption values on the next two measurements: $t_2$ and $t_3$. 

In the knowledge, we thus have one decision: $D_{SG} = {D_1}$.
This decision has been taken after one requirement analysis, $R_{A_{SG1}}$, that detects no respect of the requirement $R_1$.
To determine if there is an overload, this analysis needs to know the topology and the consumption values.
The pattern is thus defined by all nodes related to the grid network and consumption values at $t1$: $P_{1[t_1, t_1 + 1]} = \{S_1, F_1, F_2, C_1, E_1, M_1, M_2, M_3, V_1, V_2, V_3\}$.
So we have: $R_{A_{SG1}} = \{ R_1, P_{1[t_1, t_1 + 1]}\}$.

The knowledge also includes the three action executions: $A_{E_{SG1}}$, $A_{E_{SG2}}$, and $A_{E_{SG3}}$.
These actions have been executed on, respectively, $M_1$, $M_2$, and $M_3$.
Following the definition, they all contain the action $A_1$ and similar relation which linked the circumstances to the impacts.
The circumstances are the state of the knowledge at $t_0$, which contain all information of the grid network and the consumption values.
We denote them $P_{2[t_1, t_1 + 1]}$, $P_{3[t_1, t_1 + 1]}$, and $P_{4[t_1, t_1 + 1]}$, all equal $P_{1[t_1, t_1 + 1]}$.
The impact contains all consumption values received at $t_2$ and $t_3$.
Each action impacts the consumption value of the meter that it modifies.
For example, $A_{E_{SG2}}$ only impacts values of meter $M_2$.
For this action, the output pattern is thus : $P_{5[t_2, t_3]} = \{V_5, V_8\}$.
In summary, $A_{E_{SG1}}$, $A_{E_{SG2}}$, and $A_{E_{SG3}}$ are defined as follows: 
\begin{itemize}
    \item for the action executed on $M_1$: $A_{E_{SG1}} = (A_1, A_{F1})$, with $A_{F1}: P_{2[t_1, t_1 + 1]} \rightarrow \{V_4, V_7\}$,
    \item for the action executed on $M_2$: $A_{E_{SG2}} = (A_1, A_{F2})$, with $A_{F2}: P_{3[t_1, t_1 + 1]} \rightarrow \{V_5, V_8\}$,
    \item for the action executed on $M_3$: $A_{E_{SG3}} = (A_1, A_{F3})$, with $A_{F3}: P_{4[t_1, t_1 + 1]} \rightarrow \{V_6, V_9\}$,
\end{itemize}

The decision described in the scenario is thus equal to: $D_1 = \{ R_{A_{SG1}}, A_{E_{SG1}}, A_{E_{SG2}}, A_{E_{SG3}}\}$.
At $t_2$, this decision will still be valid.
The adaptation process can thus include it in the adaptation process to reason over the ongoing actions.
If at $t_3$ the cable remains overloaded, then one may use this element to check if the system tried to fix it, how and based on which information.