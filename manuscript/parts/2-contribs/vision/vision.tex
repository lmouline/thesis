%% Summary context / challenges
- adaptive systems
- need for abstraction
- MDE is great for that
- but uncertain data, time, \gls{longTermAct}

- challenges addressed (\Cref{chapt:intro})
	- Sub-challenge \#1: How to ease the manipulation of data uncertainty for software engineer?
	- Sub-challenge \#2: How to enable reasoning over unfinished actions and their expected effects?
	- Sub-challenge \#3: How to model the decisions of an adaptation process to diagnose it?
	
%In this document, we claim that \glspl{adptSyst} need 

%% Summary SOTA
- RQ
	- rq1: do state-of-the-art solutions that model adaptive system allow representing and reasoning over long-term actions
(design time and runtime)?
	- rq2: do state-of-the-art solutions allow modelling uncertainty of data and its manipulation (propagation, reasoning over)?
- No solution that models or reasons over \gls{longTermAct}
- Research efforts are still required to handle \gls{duc} at language level
%%% RQ1: long-term actions
- todo: summarize....
%%% RQ2: data uncertainty
- todo: summarize....

%% Description of the vision
- Modeling frameworks should encapsulate time and uncertainty time and uncertainty as first-class concern
%%% Pros
%%%% <!-- why a modeling framework? -->
- benefits of abstraction
    - remove unecessary details which may complexify the reasoning
    - bring specific view for stakeholders, view that use domain vocabulary / syntax / concepts
%%%% <!-- why time and uncertainty as first-class concepts? -->
- time and uncertainty appears in most of DAS
- improve performances
    - structure defined with time and uncertainty in mind
        - enable default index on the temporal aspect
- improve efforts in usage
    - temporal and/or uncertain data directly 
%%%% <!-- why combining time and uncertainty in the same structure? -->
- Concern linked: each data as a validity period (more or less long) [not a binary concept]
    - temporal data come with uncertainty
    - uncertainty may come with time
%%% Cons
%%%% <!-- why a modeling framework? -->
- limitation of abstraction mechanism
    - can introduce overhead that can impact the performances (CPU, memory)
    - lose of control of the full technology stack
    - abstraction may not consider 10% exceptionals cases
%%%%<!-- why time and uncertainty as first-class concepts? -->
- uncertainty and time can be defined as normal attributes in existing structures -> reuse of known tools by developers
    - indexes mechanism can also be added on them
%%%% <!-- why combining time and uncertainty in the same structure? -->
- Separation of concerns
    - reduce complexity for developers
    - improve their productivity



%% Introduction to the two new contributions
- \langName{}, a language to manage uncertain data (\cf \Cref{chapt:aintea})
%	This contribution addresses the challenge of the manipulation of uncertain data (cf. Sub-Challenge \#1). 
%	We propose \langName{}, a language able to represent uncertain data as built-in language types along with their supported operations.
%	An overview of the language is depicted in~\Cref{fig:intro:contrib:aintea}.
% 	It contains a sampling of distributions (Gaussian, Bernoulli, binomial, Dirac delta function, and Rayleigh) that covers the different data types (booleans, numbers, and references).
% 	We implement a prototype of the language, publicly available on GitHub\footnote{\url{https://github.com/lmouline/aintea/}}.
% 	We use a real-world case study based on \gls{sg}, built with our partner Creos S.A..
%	It shows first that our approach does not impact the conciseness of the language.
%	Second, it highlights the feasibility and the advantages of uncertainty-aware type checking systems on the language level.

- A temporal knowledge model (\cf \Cref{chapt:tkm})
%This contribution addresses the challenge of reasoning over unfinished actions, and understanding of \gls{adptSyst} \gls{behaviour} (cf. Sub-Challenge \#2 and \#3).
%First, we formalise the common core concepts implied in adaptation processes, also referred to as \gls{knowledge}.
%The formalisation is based on temporal graphs and a set of relations that trace decision impacts to circumstances.
%Second, we propose a framework to structure and store the state and behaviour of a running \gls{adptSyst}, together with a high-level \gls{api} to efficiently perform diagnosis routines.
%Our framework relies on a temporal model-based solution that efficiently abstracts decisions, their corresponding circumstances, and their effects.
%We give an overview of the \gls{metamodel} in~\Cref{fig:intro:contrib:tkm}.
%We demonstrate the applicability of our approach by applying it to a \gls{sg} based example.
%We also show that our approach can be used to diagnose the behaviour of at most the last five days of a district in the Luxembourg \gls{sg} in $\sim$2.4 seconds.