In this document, we claim that \glspl{adptSyst} need abstraction to help the reasoning process (\cf \Cref{chapt:intro}).
One way to specify this abstraction is to use the \gls{mde} methodology.
However, as we detailed in~\Cref{chapt:intro}, the research community should address different challenges to use state-of-the art solutions for such systems.
This thesis tackles three of the challenges identified (\cf \Cref{sec:intro:scope}):
\begin{itemize}
	\item \textit{[Sub-challenge \#1]}: How to ease the manipulation of data uncertainty for software engineer?
	\item\textit{ [Sub-challenge \#2]}: How to enable reasoning over unfinished actions and their expected effects? (\glspl{longTermAct})
	\item \textit{[Sub-challenge \#3]}: How to model the decisions of an adaptation process to diagnose it?
\end{itemize}

We show in~\Cref{chapt:sota} that it exists no solution to model \gls{longTermAct} or to reason over them.
Besides, we establish that research efforts are still required to handle \gls{duc} at language level.
To come with this conclusion, we perform a review driven by two research questions (\cf \Cref{sec:sota:methodo}):
\begin{itemize}
	\item \textit{[RQ1]}: Do state-of-the-art solutions that model adaptive system allow representing and reasoning over long-term actions?
	\item \textit{[RQ2]}: Do state-of-the-art solutions allow modelling uncertainty of data and its manipulation (propagation, reasoning over)?
\end{itemize}

In~\Cref{sec:sota:results:actions}, we answer the first research question.
We first show that the literature provides a few solutions to keep history of the \gls{structure}, \gls{context}, or \gls{behaviour} of systems~\cite{DBLP:conf/seke/0001FNMKT14, DBLP:conf/models/0001FNMKBT14, 	DBLP:conf/dbpl/MoffittS17, DBLP:conf/icse/TaharaOH17, DBLP:conf/pervasive/HenricksenIR02, DBLP:conf/smartgridsec/0001FKNT14}.
The other solutions, depicted in~\Cref{table:sota:results:actions:rq1.1},  such as goal models~ \cite{DBLP:conf/icse/CailliauL17, DBLP:conf/icse/IftikharW14a, DBLP:conf/icse/MendoncaAR14, DBLP:conf/icse/ChenPYNZ14, DBLP:conf/re/BaresiPS10} or object-based models~\cite{DBLP:conf/pervasive/HenricksenIR02, DBLP:conf/smartgridsec/0001FKNT14, DBLP:conf/icse/TaharaOH17} do not support natively the time dimension.
However, this feature remains important to enable the modelling of \glspl{longTermAct} with their effects on the context and their \glspl{circumstance}.
Then, we show that none of the state-of-the-art approaches allow stakeholders to model \glspl{longTermAct}.
Indeed, they do not incorporate a time dimension in their models.
We give an overview of the different solutions in~\Cref{table:sota:results:actions:rq1.2}.
Finally, we show that, using one of the current solution, one cannot  allow to reason over running actions with their expected effects.
Some solution exists to represent action effects at design time.
First, those that employ a state machine~\cite{DBLP:conf/sigsoft/MorenoCGS15, DBLP:conf/kbse/FilieriGLM11,DBLP:conf/wetice/DjoudiBZ14, DBLP:conf/aosd/ZhangGC09, DBLP:conf/icse/GhezziPST13, DBLP:conf/kbse/TajalliGEM10} represent actions as state transitions.
Effects are thus modelled as the target state.
In~\cite{DBLP:conf/smartgridsec/0001FKNT14}, the authors model and use the effects of actions to simulate different sequences of actions.
Finally, researchers defined the Stitch language~\cite{DBLP:journals/jss/ChengG12} that can abstract action effects.
They are used as conditions for the execution of the next action.
However, none of them model runtime information, useful, for instance, during a diagnosis task.

In~\Cref{sec:sota:results:duc}, we address the second research question.
We first explain that the literature addressed nine kinds of uncertainty, depicted in~\Cref{table:sota:results:duc:rq2.1}.
Among them, the most studied one is the uncertainty relative to data~\cite{DBLP:conf/models/BurguenoBMV18, baudin2017openturns, DBLP:journals/corr/BorgstromGGMG13, DBLP:conf/ecmdafa/BertoaMBBTV18, DBLP:conf/asplos/BornholtMM14, osti_1430202, DBLP:conf/sle/MayerhoferWV16, DBLP:journals/peerj-cs/SalvatierWF16, DBLP:conf/quatic/VallecilloMO16, DBLP:journals/sosym/Zhang00NO19, DBLP:journals/csi/Hall06, DBLP:journals/infsof/Jimenez-RamirezW0V15, DBLP:conf/ecmdafa/ZhangSAYON16, DBLP:journals/tkde/BarbaraGP92, DBLP:conf/vldb/BenjellounSHW06, DBLP:conf/popl/BhatAVG12, DBLP:conf/aistats/ChagantyNR13, DBLP:journals/siamsc/JaroszewiczK12, DBLP:journals/toplas/ParkPT08, DBLP:conf/ijcai/Pfeffer01, DBLP:conf/popl/RamseyP02, DBLP:conf/pldi/SankaranarayananCG13, DBLP:conf/uist/SchwarzMH11, DBLP:conf/icra/Thrun00, DBLP:journals/sac/LunnTBS00, plummer2003jags}.
In this thesis, we also concentrated on this kind of uncertainty.
Then, we detail how researchers model \gls{duc} (\cf \Cref{table:sota:results:duc:rq2.2}).
In our findings, the most used one is the probabilistic programming~\cite{baudin2017openturns, DBLP:conf/asplos/BornholtMM14, DBLP:journals/corr/BorgstromGGMG13, osti_1430202, DBLP:journals/peerj-cs/SalvatierWF16, DBLP:conf/popl/BhatAVG12, DBLP:conf/aistats/ChagantyNR13, DBLP:journals/siamsc/JaroszewiczK12, DBLP:journals/toplas/ParkPT08, DBLP:conf/ijcai/Pfeffer01, DBLP:conf/popl/RamseyP02, DBLP:conf/pldi/SankaranarayananCG13, DBLP:conf/icra/Thrun00, DBLP:journals/sac/LunnTBS00, plummer2003jags}.
Using this programming paradigms, developers can manipulate probability distribution as variables.
They can define complex probability distributions that result from a sequence of operations.
Another approach is to consider variable as a pair of a standard deviation and a value~\cite{DBLP:conf/models/BurguenoBMV18, DBLP:conf/ecmdafa/BertoaMBBTV18, DBLP:conf/sle/MayerhoferWV16, DBLP:conf/quatic/VallecilloMO16, DBLP:journals/tkde/BarbaraGP92, DBLP:conf/uist/SchwarzMH11}.
Finally, we look for studies that propose a solution to propagate or reason over uncertainty (\cf \Cref{table:sota:results:duc:rq2.3.1}).
The main approach found to propagate the uncertainty is to map this operation to language operators~\cite{DBLP:conf/models/BurguenoBMV18, baudin2017openturns, DBLP:journals/corr/BorgstromGGMG13, DBLP:conf/ecmdafa/BertoaMBBTV18, osti_1430202, DBLP:conf/sle/MayerhoferWV16, DBLP:journals/peerj-cs/SalvatierWF16, DBLP:conf/quatic/VallecilloMO16, DBLP:conf/popl/BhatAVG12, DBLP:conf/aistats/ChagantyNR13, DBLP:journals/siamsc/JaroszewiczK12, DBLP:journals/toplas/ParkPT08, DBLP:conf/ijcai/Pfeffer01, DBLP:conf/popl/RamseyP02, DBLP:conf/pldi/SankaranarayananCG13, DBLP:conf/icra/Thrun00, DBLP:journals/sac/LunnTBS00, plummer2003jags}.
Concerning reasoning approaches, we find only two ways, as shown in~\Cref{table:sota:results:duc:rq2.3.2}.
The first one consists in giving access to the confidence parameter~\cite{DBLP:conf/models/BurguenoBMV18, DBLP:conf/ecmdafa/BertoaMBBTV18, DBLP:conf/sle/MayerhoferWV16, DBLP:conf/quatic/VallecilloMO16, DBLP:journals/tkde/BarbaraGP92, DBLP:conf/uist/SchwarzMH11}.
The second one is to allow developers to read and manipulate features of the probability distribution~\cite{baudin2017openturns, DBLP:conf/asplos/BornholtMM14, DBLP:journals/corr/BorgstromGGMG13, osti_1430202, DBLP:journals/peerj-cs/SalvatierWF16, DBLP:conf/popl/BhatAVG12, DBLP:conf/aistats/ChagantyNR13, DBLP:journals/siamsc/JaroszewiczK12, DBLP:journals/toplas/ParkPT08, DBLP:conf/ijcai/Pfeffer01, DBLP:conf/popl/RamseyP02, DBLP:conf/pldi/SankaranarayananCG13, DBLP:conf/icra/Thrun00, DBLP:journals/sac/LunnTBS00, plummer2003jags}.

To summarize, our review of the state-of-the-art shows first that research efforts are still necessary to specify a model-based approach for \glspl{longTermAct}.
Besides, it demonstrates that, despite that it has been heavily studied, the research community should focus on defining solution to manipulate uncertain data at a higher level than the manipulation of probability distributions.
We think that both challenges can be addressed by the definition of a modeling framework that includes, despite all traditional elements, temporal and uncertainty as first-class concepts.

%% todo describes the vision


We present, in this thesis, two contributions towards this vision.
First, in~\Cref{chapt:aintea} we detail our language named \langName{}, which permit designers managing uncertainty at language level.
This solution addressed the challenge of the manipulation of uncertain data (cf. Sub-Challenge \#1). 
Second, we describe a temporal knowledge model in~\Cref{chapt:tkm} to structure and store the state and behaviour of a running \gls{adptSyst}, with running \glspl{longTermAct}.
This model addresses the challenge of reasoning over unfinished actions, and understanding of \gls{adptSyst} \gls{behaviour} (cf. Sub-Challenge \#2 and \#3).





%%% Description of the vision
%- Modeling frameworks should encapsulate time and uncertainty time and uncertainty as first-class concern
%%%% Pros
%%%%% <!-- why a modeling framework? -->
%- benefits of abstraction
%    - remove unecessary details which may complexify the reasoning
%    - bring specific view for stakeholders, view that use domain vocabulary / syntax / concepts
%%%%% <!-- why time and uncertainty as first-class concepts? -->
%- time and uncertainty appears in most of DAS
%- improve performances
%    - structure defined with time and uncertainty in mind
%        - enable default index on the temporal aspect
%- improve efforts in usage
%    - temporal and/or uncertain data directly 
%%%%% <!-- why combining time and uncertainty in the same structure? -->
%- Concern linked: each data as a validity period (more or less long) [not a binary concept]
%    - temporal data come with uncertainty
%    - uncertainty may come with time
%%%% Cons
%%%%% <!-- why a modeling framework? -->
%- limitation of abstraction mechanism
%    - can introduce overhead that can impact the performances (CPU, memory)
%    - lose of control of the full technology stack
%    - abstraction may not consider 10% exceptionals cases
%%%%%<!-- why time and uncertainty as first-class concepts? -->
%- uncertainty and time can be defined as normal attributes in existing structures -> reuse of known tools by developers
%    - indexes mechanism can also be added on them
%%%%% <!-- why combining time and uncertainty in the same structure? -->
%- Separation of concerns
%    - reduce complexity for developers
%    - improve their productivity
