\section{Scope of the thesis}
\label{sec:intro:scope}

Among all the challenges described in the previous section, this thesis focus on three of them: \gls{duc} (\Cref{sec:intro:challenges:duc}), \glspl{longTermAct} (\Cref{sec:intro:challenges:longTermAct}), and error-prone adaptation process (\Cref{sec:intro:challenges:diagnosis}).
More precisely, we address three sub-problems of these challenges.

Managing uncertainty requires significant expertise in probability and statistics theory.
The literature provides different solutions to manage uncertainty~\cite{zadeh1996fuzzy,metrology2008evaluation,shafer1992dempster}.
The application of these techniques requires a deep understanding of the underlying theories and is a time-consuming task~\cite{DBLP:conf/quatic/VallecilloMO16}.
Moreover, it is hard to test and perhaps most importantly, very error-prone.
In this thesis, we address thus the following problem:
\vspace{-2em}
\highlightbox[Sub-challenge \#1]{How to ease the manipulation of data uncertainty for software engineer?}

Adaptation processes may rely on \gls{longTermAct} like resource migration in cloud infrastructure.
The lack of information about unfinished actions and their expected effects on the system, the reasoning component may take repeated or sub-optimal decisions.
One step for enabling this reasoning mechanism is to have an abstraction layer which can represent these \glspl{longTermAct} efficiently.
In this thesis, we, therefore, cope with the following challenge:
\vspace{-2em}
\highlightbox[Sub-challenge \#2]{How to enable reasoning over unfinished actions and their expected effects?}

Due to the increasing complexity of systems, developers have difficulties in delivering error-free software~\cite{DBLP:conf/icse/BarbosaLMJ17, DBLP:conf/icse/MongielloPS15, DBLP:conf/icse/HassanBB15}.
Moreover, complex systems or large-scale systems may have emergent behaviours.
Systems very likely have an abnormal behaviour that was not foreseen at design time.
Existing formal modelling and verification approaches may not be sufficient to verify and validate such processes~\cite{DBLP:conf/icse/TaharaOH17}.
In such situations, developers usually apply diagnosis routines to identify the causes of the failures.
During our studies, we tackle the following challenge:
\vspace{-2em}
\highlightbox[Sub-challenge \#3]{How to model the decisions of an adaptation process to diagnose it?}

