\section{Challenge}

In this section, we present five challenges for the research community in \gls{mde} and \glspl{adptSyst}.
The last one has been published in our vision paper regarding time awareness in \gls{mde} in~\cite{DBLP:conf/models/Benelallam0MFBB17}.

\subsection{Data are uncertain}
\label{intro:challenges:u-data}
Data become a cornerstone piece to autonomously derive decisions from them, or at least to support decision-making processes.
We argue that their uncertainty will impact all the development stages of software, from the design to the execution.
Design techniques should provide mechanisms to help developers abstract and manipulating uncertain data.
Control flows use data, for example, in the branching conditions.
This branching should be redesigned to consider the uncertainty in the data.

The literature provides approaches to help engineers reason or manipulate data uncertainty, or at least probability distributions.
For example, believe functions~\cite{shafer1992dempster} help to reduce this uncertainty by combining several sources of data.
The probabilistic programming~\cite{DBLP:conf/icse/GordonHNR14} community provide frameworks and languages~\cite{url:InferNET18, baudin2017openturns} to propagate probabilities through computations.

However, from the best of our knowledge, no global study has been done to evaluate the impact of data uncertainty on the development of software.
The following challenge still remains an open question for the software engineering community:
\vspace{-2em}
\highlightbox{How to engineer uncertainty-aware software (design, implement, test, and validate)?}

\subsection{Actions have long-term effects}
\label{intro:challenges:long-term-act}
Decision-making processes follow the growing complexity of software.
They are more and more able to make not only decisions on the current state of the system but also its past and future ones.
And this decision may also have long-term effects.

Due to this complexity, developers and users may misunderstand the decisions taken by a system.
Plus, designers may neglect or underestimate the impact of a decision.
Moreover, as highlighted by Bencomo \etal \cite{DBLP:conf/iceccs/BencomoWSW12}, systems should be self-explained.
They should be able to explain the decisions made.

To achieve this vision and to help designers and users understanding the impact of a decision, we argue that the software engineering community should address the following question:
\vspace{-2em}
\highlightbox{How to represent, query, store, and understand the impacts of long-term actions?}

\subsection{Systems may have emergent behaviours}
\label{intro:challenges:ermger-bhv}
The growing complexity of systems also has another impact: they have emergent behaviour.
This behaviour may be suboptimal and hard to understand by designers, who generally have a local vision of the system.

However, when this behaviour leads to failure, engineers still need to understand why and how to avoid a novel occurrence of the problem.
Plus, as the behaviour might be suboptimal, they need to optimise it.

To reach this goal, engineers need tooling support to help them in their investigation process.
In other words, the research community should answer the following global challenge:
\vspace{-2em}
\highlightbox{How to understand, predict, and optimise emergent behaviours?}

\subsection{Different part of a system evolve at different rates}
\label{intro:challenges:diff-paces}
Systems may be heterogeneous and diverse, in terms of hardware but also software.
Due to this diversity, the different part of a system may evolve at different rates.

However, engineers may need to reason on a global view.
They will thus need to deal with data that have different freshness but also with components that can have their behaviour changed at different spaces.
While decisions are executed to optimise the global behaviour, the system may be in an inconsistent state or a less good state than the initial one.

When designing the adaptation process, engineers need thus to consider this difference in freshness.
For example, they need to implement estimation functions to estimate the value of an outdated data.
Plus, they need to consider the fact that the system can be in an inconsistent state while being updated.
To sum up, the software engineering community should answer the following global challenge:
\vspace{-2em}
\highlightbox{How to represent, query, and store inconsistent system states and behaviours?}

\subsection{Evolution of systems are linked with time}
\label{intro:challenges:evol-syst}
System behaviours are temporal: the execution of the different actions are made over time.
Plus, the structure of a system evolves over time.
Not only may the last state be important but also how it evolves.
Engineers need thus to analyse this temporal evolution.

Time in software engineering is not a new challenge.
For example, Riviera \etal \cite{DBLP:conf/models/RiveraRV08} have already identified time as a challenge for the \gls{mde} community.
Different approaches have been defined~\cite{DBLP:conf/sle/BousseCCGB15, DBLP:conf/sle/KansoT12, DBLP:conf/icse/KoegelH10, DBLP:conf/seke/0001FNMKT14}.

However, we notice that modelling, persistence, and processing of data evolution remain understudied.
Thomas Hartmann started addressing these challenging in his PhD thesis~\cite{DBLP:phd/basesearch/Hartmann16}.
The final global challenge, not fully addressed, is thus:
\vspace{-2em}
\highlightbox{How to structure, represent query, and store efficiently temporal data on a large scale?}