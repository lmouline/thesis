\section{Conclusion}
%\label{sec:sota:conclusion}

In this chapter, we review the state-of-the-art approaches to answer two research questions.
First, we look for studies that model \glspl{adptSyst}, their \glspl{context} or \glspl{behaviour} to see if they also consider \gls{longTermAct} (RQ1).

In~\Cref{sec:sota:results:actions}, we answer the first research question.
We first show that the literature provides a few solutions to keep the history of the \gls{structure}, \gls{context}, or \gls{behaviour} of systems~\cite{DBLP:conf/seke/0001FNMKT14, DBLP:conf/models/0001FNMKBT14, 	DBLP:conf/dbpl/MoffittS17, DBLP:conf/icse/TaharaOH17, DBLP:conf/pervasive/HenricksenIR02, DBLP:conf/smartgridsec/0001FKNT14}.
The other solutions, depicted in~\Cref{table:sota:results:actions:rq1.1},  such as goal models~ \cite{DBLP:conf/icse/CailliauL17, DBLP:conf/icse/IftikharW14a, DBLP:conf/icse/MendoncaAR14, DBLP:conf/icse/ChenPYNZ14, DBLP:conf/re/BaresiPS10} or object-based models~\cite{DBLP:conf/pervasive/HenricksenIR02, DBLP:conf/smartgridsec/0001FKNT14, DBLP:conf/icse/TaharaOH17} do not support the time dimension natively.
However, this feature remains important to enable the modelling of \glspl{longTermAct}.
Then, we show that none of the state-of-the-art approaches allows stakeholders to model \glspl{longTermAct}.
Indeed, they do not incorporate a time dimension in their models.
We give an overview of the different solutions in~\Cref{table:sota:results:actions:rq1.2}.
Finally, we show that, using one of the current solutions, one cannot reason over running actions with their expected effects.
Some solution exists to represent action effects at design time.
First, those that employ a state machine~\cite{DBLP:conf/sigsoft/MorenoCGS15, DBLP:conf/kbse/FilieriGLM11,DBLP:conf/wetice/DjoudiBZ14, DBLP:conf/aosd/ZhangGC09, DBLP:conf/icse/GhezziPST13, DBLP:conf/kbse/TajalliGEM10} represent actions as state transitions.
Effects are thus modelled as the target state.
In~\cite{DBLP:conf/smartgridsec/0001FKNT14}, the authors model and use the effects of actions to simulate different sequences of actions.
Lastly, researchers defined the Stitch language~\cite{DBLP:journals/jss/ChengG12} that can abstract action effects.
They are used as conditions for the execution of the next action.
However, none of them models runtime information, useful, for instance, during a diagnosis task.

Then, we search for solutions that model \gls{duc} and present our findings in~\Cref{sec:sota:results:duc}.
We first explain that the literature addressed nine kinds of uncertainties, depicted in~\Cref{table:sota:results:duc:rq2.1}.
Among them, the most studied one is the uncertainty relative to data~\cite{DBLP:conf/models/BurguenoBMV18, baudin2017openturns, DBLP:journals/corr/BorgstromGGMG13, DBLP:conf/ecmdafa/BertoaMBBTV18, DBLP:conf/asplos/BornholtMM14, osti_1430202, DBLP:conf/sle/MayerhoferWV16, DBLP:journals/peerj-cs/SalvatierWF16, DBLP:conf/quatic/VallecilloMO16, DBLP:journals/sosym/Zhang00NO19, DBLP:journals/csi/Hall06, DBLP:journals/infsof/Jimenez-RamirezW0V15, DBLP:conf/ecmdafa/ZhangSAYON16, DBLP:journals/tkde/BarbaraGP92, DBLP:conf/vldb/BenjellounSHW06, DBLP:conf/popl/BhatAVG12, DBLP:conf/aistats/ChagantyNR13, DBLP:journals/siamsc/JaroszewiczK12, DBLP:journals/toplas/ParkPT08, DBLP:conf/ijcai/Pfeffer01, DBLP:conf/popl/RamseyP02, DBLP:conf/pldi/SankaranarayananCG13, DBLP:conf/uist/SchwarzMH11, DBLP:conf/icra/Thrun00, DBLP:journals/sac/LunnTBS00, plummer2003jags}.
In this thesis, we also concentrated on this kind of uncertainty.
Then, we detail how researchers model \gls{duc} (\cf \Cref{table:sota:results:duc:rq2.2}).
In our findings, the most used one is the probabilistic programming~\cite{baudin2017openturns, DBLP:conf/asplos/BornholtMM14, DBLP:journals/corr/BorgstromGGMG13, osti_1430202, DBLP:journals/peerj-cs/SalvatierWF16, DBLP:conf/popl/BhatAVG12, DBLP:conf/aistats/ChagantyNR13, DBLP:journals/siamsc/JaroszewiczK12, DBLP:journals/toplas/ParkPT08, DBLP:conf/ijcai/Pfeffer01, DBLP:conf/popl/RamseyP02, DBLP:conf/pldi/SankaranarayananCG13, DBLP:conf/icra/Thrun00, DBLP:journals/sac/LunnTBS00, plummer2003jags}.
Using this programming paradigms, developers can manipulate probability distribution as variables.
They can define complex probability distributions that result from a sequence of operations.
Another approach is to consider a variable as a pair of a standard deviation and a value~\cite{DBLP:conf/models/BurguenoBMV18, DBLP:conf/ecmdafa/BertoaMBBTV18, DBLP:conf/sle/MayerhoferWV16, DBLP:conf/quatic/VallecilloMO16, DBLP:journals/tkde/BarbaraGP92, DBLP:conf/uist/SchwarzMH11}.
Finally, we look for studies that propose a solution to propagate or reason over uncertainty (\cf \Cref{table:sota:results:duc:rq2.3.1}).
The principal approach found to propagate the uncertainty is to map this operation to language operators~\cite{DBLP:conf/models/BurguenoBMV18, baudin2017openturns, DBLP:journals/corr/BorgstromGGMG13, DBLP:conf/ecmdafa/BertoaMBBTV18, osti_1430202, DBLP:conf/sle/MayerhoferWV16, DBLP:journals/peerj-cs/SalvatierWF16, DBLP:conf/quatic/VallecilloMO16, DBLP:conf/popl/BhatAVG12, DBLP:conf/aistats/ChagantyNR13, DBLP:journals/siamsc/JaroszewiczK12, DBLP:journals/toplas/ParkPT08, DBLP:conf/ijcai/Pfeffer01, DBLP:conf/popl/RamseyP02, DBLP:conf/pldi/SankaranarayananCG13, DBLP:conf/icra/Thrun00, DBLP:journals/sac/LunnTBS00, plummer2003jags}.
Concerning reasoning approaches, we find only two ways, as shown in~\Cref{table:sota:results:duc:rq2.3.2}.
The first one consists in giving access to the confidence parameter~\cite{DBLP:conf/models/BurguenoBMV18, DBLP:conf/ecmdafa/BertoaMBBTV18, DBLP:conf/sle/MayerhoferWV16, DBLP:conf/quatic/VallecilloMO16, DBLP:journals/tkde/BarbaraGP92, DBLP:conf/uist/SchwarzMH11}.
The second one is to allow developers to read and manipulate features of the probability distribution~\cite{baudin2017openturns, DBLP:conf/asplos/BornholtMM14, DBLP:journals/corr/BorgstromGGMG13, osti_1430202, DBLP:journals/peerj-cs/SalvatierWF16, DBLP:conf/popl/BhatAVG12, DBLP:conf/aistats/ChagantyNR13, DBLP:journals/siamsc/JaroszewiczK12, DBLP:journals/toplas/ParkPT08, DBLP:conf/ijcai/Pfeffer01, DBLP:conf/popl/RamseyP02, DBLP:conf/pldi/SankaranarayananCG13, DBLP:conf/icra/Thrun00, DBLP:journals/sac/LunnTBS00, plummer2003jags}.\looseness-1


To summarise, our review shows that none of the current approaches model or enable reasoning over \glspl{action} with delayed effects.
Thus, some research efforts are still required to specify solutions that allow designers to add \gls{longTermAct} in their model and to implement techniques to reason over them.
In this thesis, we start studying this problem, and we present a model-based solution, detailed in \Cref{chapt:tkm}.
Moreover, it different solutions have been found in our review to manage data uncertainty, the main one being referred to as probabilistic programming.
This solution allows developers to manipulate probability distributions as common variables of a programming language.
However, as seen in our review, not all kinds of uncertainty can be represented by this approach.
For example, some researchers represent the uncertainty of a value with a set of different possibilities.
Therefore, open challenges still need research efforts to handle \gls{duc}.
Towards solving these challenges, we start by defining a language that integrates \gls{duc} as a first-class citizen (\cf \Cref{chapt:aintea}).

