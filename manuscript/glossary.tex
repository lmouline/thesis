\newglossaryentry{action}
{
	name={action},
	description={\textquote{Process that, given the \gls{context} and \glspl{requirement} as input, adjusts the system behavior}, IEEE Standards~\cite{iso2017systems}}
}

\newglossaryentry{context}
{
	name={context},
	description={In this document, I use the definition provided by Anind K. Dey~\cite{DBLP:journals/puc/Dey01}: \textquote{Context is any information that can be used to characterize the situation of an entity. An entity is a person, place, or object that is considered relevant to the interaction between a user and [the system], including the user and [the system] themselves}}
}

\newglossaryentry{knowledge}
{
	name={knowledge},
	description={The knowledge of an adaptive system gathers information about the \gls{context}, \glspl{action} and \glspl{requirement}}
}

\newglossaryentry{requirement}
{
	name={requirement},
	description={\textquote{(1) Statement that translates or expresses a need and its associated constraints and conditions, (2) Condition or capability that must be met or possessed by a system [...] to satisfy an agreement, standard, specification, or other formally imposed documents}, IEEE Standards~\cite{iso2017systems}}
}

\newglossaryentry{circumstance}
{
	name={circumstance},
	description={State of the \gls{knowledge} when a \gls{decision} has been taken}
}

\newglossaryentry{decision}
{
	name={decision},
	description={A set of \glspl{action} taken after comparing the state of the \gls{knowledge} with the \gls{requirement}}
}

\newglossaryentry{metamodel}
{
	name={metamodel},
	description={
% todo: check :
% https://www.sciencedirect.com/science/article/pii/S1477842415000408
% http://citeseerx.ist.psu.edu/viewdoc/download?doi=10.1.1.106.9720&rep=rep1&type=pdf
% https://www.cs.kent.ac.uk/projects/kmf/Documents/ifm02paper.pdf
% https://www.researchgate.net/profile/Ed_Seidewitz/publication/3248044_What_models_mean/links/06187482fdd792dc482d00f7/What-models-mean.pdf
Through this thesis, I use the definition of Seidewitz: \textquote{A metamodel is a specification model for a class of [system  under  study] where each [system  under  study] in the class is it-self a valid model expressed in a certain modeling  language.}~\cite{DBLP:journals/software/Seidewitz03}
	}
}

\newglossaryentry{mapekg}
{
	name={MAPE-k},
    description={A theoretical model of the adaptation process proposed by Kephart and Chess~\cite{DBLP:journals/computer/KephartC03}. It divides the process in four stages: monitoring, analysing, planning and executing. These four stages share a \gls{knowledge}},
    see=[Abbreviation:]{mapek}
}

\newglossaryentry{mapek}
{
	type=\acronymtype, 
	name={MAPE-k}, 
	description={Monitor, Analyze, Plan, and Execute over knowledge}, 
	first={Monitor, Analyze, Plan, and Execute over knowledge (MAPE-k)\glsadd{mapekg}}, 
	see=[Glossary:]{mapekg}
}




