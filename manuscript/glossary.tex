%% A
\newglossaryentry{action}
{
	name={ac\-tion},
	description={In this document, we use the definition provided by IEEE Standards~\cite{iso2017systems}: \textquote{Process that, given the \gls{context} and \glspl{requirement} as input, adjusts the \gls{systBhv}}}
}

\newglossaryentry{adptSyst}
{
	name={a\-dap\-tive sys\-tem},
	description={In this document, we modified the definition of self-adaptive systems provided by Cheng~\etal in \cite{DBLP:conf/dagstuhl/ChengLGIMABBBCSDFGGGKKKLMMMPSTTWW09}. Adaptive systems are able to have their \gls{behaviour} adjusted in response to the perception of the \gls{env} and the system themselves. If a system perform this adjustment on itself, the literature refers to it as self-adaptive system}
}

\newglossaryentry{api}
{
	type=\acronymtype,
	name={API},
	description={Application Programming Interface},
	first={Application Programming Interface (API)}
}

%% B
\newglossaryentry{behaviour}
{
	name={be\-ha\-viour},
	description={We refer to \gls{systBhv}}
}

%% C
\newglossaryentry{circumstance}
{
	name={cir\-cums\-tance},
	description={State of the \gls{knowledge} when a \gls{decision} has been taken}
}

\newglossaryentry{context}
{
	name={con\-text},
	description={In this document, we use the definition provided by Anind K. Dey~\cite{DBLP:journals/puc/Dey01}: \textquote{Context is any information that can be used to characterise the situation of an entity. An entity is a person, place, or object that is considered relevant to the interaction between a user and [the system], including the user and [the system] themselves}}
}

\newglossaryentry{cps}
{
	type=\acronymtype,
	name={CPS},
	description={Cyber-Physical System},
	first={Cyber-Physical System (CPS)}
}

\newglossaryentry{cpu}
{
	type=\acronymtype,
	name={CPU},
	description={Central Processing Unit},
	first={Central Processing Unit (CPU)}
}

%% D
\newglossaryentry{duc}
{
	name={data uncertainty},
	description={\todo{TBD}}
}

\newglossaryentry{decision}
{
	name={de\-ci\-sion},
	description={A set of \glspl{action} taken after comparing the state of the \gls{knowledge} with the \gls{requirement}}
}

\newglossaryentry{dom}
{
	type=\acronymtype,
	name={DOM},
	description={Document Object Model},
	first={Document Object Model (DOM)}
}

\newglossaryentry{dslg}
{
	name={DSL},
	description={In this document, we use the definition provided by Deursen \etal \cite{DBLP:journals/sigplan/DeursenKV00}: \textquote{is a programming language or executable specification language that offers, through appropriate notations and abstractions, expressive power focused on, and usually restricted to, a particular problem domain.}},
	see=[Abbreviation:]{dsl}
}

\newglossaryentry{dsl}
{
	type=\acronymtype,
	name={DSL},
	description={Domain-Specific Language},
	first={Domain Specific Language (DSL)\glsadd{dslg}},
	see=[Glossary:]{dslg}
}

\newglossaryentry{dsml}
{
	type=\acronymtype,
	name={DSML},
	description={Domain Specific Modelling Language},
	first={Domain Specific Modelling Language (DSML)}
}

%% E
\newglossaryentry{emf}
{
	type=\acronymtype,
	name={EMF},
	description={Eclipse Modelling Framework},
	first={Eclipse Modelling Framework (EMF)}
}

\newglossaryentry{emof}
{
	type=\acronymtype,
	name={E-MOF},
	description={Essential MOF (EMOF)},
	first={Essential MOF (EMOF)}
}

\newglossaryentry{env}
{
	name={envi\-ron\-ment},
	description={See \gls{systEnv}}
}

\newglossaryentry{exec}
{
	name={exe\-cu\-tion},
	description={In this document, we use the definition provided in the \gls{uml} specification~\cite{omg2017umlspec}: \textquote{An execution is a performance of a set of \glspl{action} (potentially over some period of time) that may generate and respond to occurrences of events, including accessing and changing the state of [the system]}}
}

%% G
\newglossaryentry{gcm}
{
	type=\acronymtype,
	name={GCM},
	description={GreyCat Modelling Environment},
	first={GreyCat Modelling Environment (GCM)}
}

\newglossaryentry{gpl}
{
	type=\acronymtype,
	name={GPL},
	description={General Purpose Language},
	first={General Purpose Language (GPL)}
}

\newglossaryentry{gps}
{
	type=\acronymtype,
	name={GPS},
	description={Global Positioning System},
	first={Global Positioning System (GPS)}
}

%% I
\newglossaryentry{ict}
{
	type=\acronymtype,
	name={ICT},
	description={Information and communication technology},
	first={Information and Communication Technology (ICT)}
}

\newglossaryentry{iot}
{
	type=\acronymtype,
	name={IoT},
	description={Internet of Things},
	first={Internet of Things (IoT)}
}

%% K
\newglossaryentry{kmf}
{
	type=\acronymtype,
	name={KMF},
	description={Kevoree Modelling Framework},
	first={Kevoree Modelling Framework (KMF)}
}

\newglossaryentry{knowledge}
{
	name={knowl\-ed\-ge},
	description={The knowledge of an adaptive system gathers information about the \gls{context}, \glspl{action} and \glspl{requirement}}
}

%% L
\newglossaryentry{longTermAct}
{
	name={long-term action},
	description={An \gls{action} that is not immediate, or that takes time to be executed, or that has long-term effects.}
}

%% M
\newglossaryentry{mapekg}
{
	name={MAPE-k},
    description={A theoretical model of the adaptation process proposed by Kephart and Chess~\cite{DBLP:journals/computer/KephartC03}. It divides the process in four stages: monitoring, analysing, planning and executing. These four stages share a \gls{knowledge}},
    see=[Abbreviation:]{mapek}
}

\newglossaryentry{mapek}
{
	type=\acronymtype, 
	name={MAPE-k}, 
	description={Monitor, Analyse, Plan, and Execute over knowledge}, 
	first={Monitor, Analyse, Plan, and Execute over knowledge (MAPE-k)\glsadd{mapekg}}, 
	see=[Glossary:]{mapekg}
}

\newglossaryentry{mde}
{
	type=\acronymtype, 
	name={MDE},
	description={Model-Driven Engineering},
	first={Model-Driven Engineering (MDE)}
}

\newglossaryentry{metamodel}
{
	name={meta\-model},
	description={In this document, we use the definition provided by Douglas C. \linebreak Schmidt~\cite{DBLP:journals/computer/Schmidt06}: \textquote{[Metamodels] define the relationships among concepts in a domain and precisely specify the key semantics and constraints associated with these domain concepts}}
}

\newglossaryentry{model}
{
	name={model},
	description={In this document, we use the definition provided by Brambilla \etal~\cite{DBLP:series/synthesis/2017Brambilla}: \textquote{[A model is] a simplified or partial representation of reality, defined in order to accomplish a task or to reach an agreement on a topic.} The model should conform to a \gls{metamodel}: each element of the model instantiates one from the \gls{metamodel} and satisfies all semantics rules~\cite{DBLP:conf/iceccs/BezivinJT05}}
}

\newglossaryentry{mof}
{
	type=\acronymtype, 
	name={MOF},
	description={Meta Object Facility},
	first={Meta Object Facility (MOF)}
}

\newglossaryentry{m@rt}
{
	name={models@run.time},
	description={In this document, we use the definition provided by Blair \etal~\cite{DBLP:journals/computer/BlairBF09}: \textquote{A model@run.time is a causally connected self-representation of the associated system that emphasises the structure, behaviour, or goals of the system from a problem space perspective}}
}

%% N
\newglossaryentry{nist}
{
	type=\acronymtype,
	name={NIST},
	description={National Institute of Standards and Technology},
	first={National Institute of Standards and Technology (NIST)}
}

%% O
\newglossaryentry{omg}
{
	type=\acronymtype,
	name={OMG},
	description={Object Management Group},
	first={Object Management Group (OMG)}
}

\newglossaryentry{ocl}
{
	type=\acronymtype,
	name={OCL},
	description={Object Constraint Language},
	first={Object Constraint Language (OCL)}
}


%% R
\newglossaryentry{requirement}
{
	name={requirement},
	description={In this document, we use the definition provided by IEEE Standards~\cite{iso2017systems}: \textquote{(1) Statement that translates or expresses a need and its associated constraints and conditions, (2) Condition or capability that must be met or possessed by a system [...] to satisfy an agreement, standard, specification, or other formally imposed documents}}
}

%% S
\newglossaryentry{sadapt}
{
	name={self-adaptive system},
	description={See \gls{adptSyst}}
}

\newglossaryentry{shealingSyst}
{
	name={self-healing system},
	description={In this document, we use the definition provided by Kephart and Chess~\cite{DBLP:journals/computer/KephartC03}: \textquote{[A self-healing] system automatically detects, diagnoses, and repairs localised software and hardware problems.}}
}

\newglossaryentry{shealing}
{
	name={self-healing},
	description={Refers to the capacity of detecting, diagnosing, and repairing any error in the system. See \gls{shealingSyst}}
}

\newglossaryentry{sg}
{
	name={smart grid},
	description={In this document, we use the definition provided by the \gls{nist}~\cite{NIST:SmartGrid:Def:What}: \textquote{a modernized grid that enables bidirectional flows of energy and uses two-way communication and control capabilities that will lead to an array of new functionalities and applications.}}
}
\newglossaryentry{sleg}
{
	name={SLE},
 	description={In this document, we use the definition provided by Anneke Kleppe \cite{kleppe2008software}: \textquote{the application of systematic, disciplined, and measurable approaches to the development, use, deployment, and maintenance of software languages}},
 	see=[Abbreviation:]{sle}
}

\newglossaryentry{sle}
{
	type=\acronymtype,
	name={SLE},
	description={Software Language Engineering},
	first={Software Language Engineering (SLE)\glsadd{sleg}},
	see=[Glossary:]{sleg}
}

\newglossaryentry{softLge}
{
	name={software language},
	description={In this document, we use the definition provided by Anneke Kleppe \cite{kleppe2008software}: \textquote{any language that is created to describe and create software systems}}
}

\newglossaryentry{structure}
{
	name={structure},
	description={See \gls{systStruct}}
}

\newglossaryentry{systEnv}
{
	name={system environment},
	description={Where the system is executed or is running. For example, a cloud application have the cloud infrastructure as environment}
}

\newglossaryentry{systBhv}
{
	name={system behaviour},
	description={In this document, we use the definition provided in the \gls{uml} specification~\cite{omg2017umlspec}: \textquote{A behaviour describes a set of possible \glspl{exec}.}(p. 12) \textquote{Behaviour may be \textit{executed}, either by direct invocation or through the creation of an [element] that hosts the behaviour. Behaviour may also be \textit{emergent}, resulting from the interaction of one or more [elements] that are themselves carrying out their own individual behaviours.}(p. 285)}
}

\newglossaryentry{systStruct}
{
	name={system structure},
	description={The structure describes all elements that compose the system}
}

%% U
\newglossaryentry{uml}
{
	type=\acronymtype,
	name={UML},
	description={Unified Modelling Language},
	first={Unified Modelling Language (UML)}
}

\newglossaryentry{uncertainty}
{
	name={uncertainty},
	description={Refers to \gls{duc}.}
}

%% X
\newglossaryentry{xmi}
{
	type=\acronymtype,
	name={XMI},
	description={XML Metadata Interchange},
	first={XML Metadata Interchange (XMI)}
}

