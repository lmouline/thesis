%\thchapter{TCM: a temporal context model to represent actions, their contexts and their impacts}
%{
%The evolving complexity of adaptive systems impairs our ability to deliver anomaly-free solutions. 
%Fixing these systems require a deep understanding on the reasons behind decisions which led to faulty or suboptimal system states. 
%Developers thus need diagnosis support that trace system states to the previous circumstances –targeted requirements, input context– that had resulted in these decisions. 
%However, the lack of efficient temporal representation limits the tracing ability of current approaches. 
%To tackle this problem, we first propose a knowledge formalism to define the concept of a decision. 
%Second, we describe a novel temporal data model to represent, store and query decisions as well as their relationship with the knowledge (context, require- ments, and actions). 
%We validate our approach through a use case based on the smart grid at Luxembourg. We also demonstrate its scalability both in terms of execution time and consumed memory
%}
%
%\thchapter[TCM2]{TCM: a temporal context model to represent actions, their contexts and their impacts}
%{
%The evolving complexity of adaptive systems impairs our ability to deliver anomaly-free solutions. 
%Fixing these systems require a deep understanding on the reasons behind decisions which led to faulty or suboptimal system states. 
%Developers thus need diagnosis support that trace system states to the previous circumstances –targeted requirements, input context– that had resulted in these decisions. 
%However, the lack of efficient temporal representation limits the tracing ability of current approaches. 
%To tackle this problem, we first propose a knowledge formalism to define the concept of a decision. 
%Second, we describe a novel temporal data model to represent, store and query decisions as well as their relationship with the knowledge (context, require- ments, and actions). 
%We validate our approach through a use case based on the smart grid at Luxembourg. We also demonstrate its scalability both in terms of execution time and consumed memory
%}

\thchapter{TCM1}{abtract 1}

an empry paragraph.....

\section{SSection 1}

\thchapter{TCM2}{abstract 2}

an empry paragraph.....

