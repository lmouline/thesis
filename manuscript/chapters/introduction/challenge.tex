\section{Challenge}

\subsection{Data are uncertain}
\label{intro:challenges:u-data}
Data become a cornerstone piece to autonomously derive decisions from them, or at least to support decision-making processes.
We argue that their uncertainty will impact all the development stages of software, from the design to the execution.
Design techniques should provide mechanism to help developers abstracting and manipulating uncertain data.
Control flows use data, for example, in the branching conditions.
This branching should be redesign to consider the uncertainty in the data.

The literature provides approaches to help reasoning or manipulating data uncertainty, or at least probability distributions.
For example, beliefs functions~\cite{shafer1992dempster} help to reduce this uncertainty by combining several sources of data.
Probabilistic programming~\cite{DBLP:conf/icse/GordonHNR14} community provide frameworks and languages~\cite{url:InferNET18, baudin2017openturns} to propagate probabilities through computations.

However, from the best of our knowledge, no global study have been done to evaluate the impact of data uncertainty on the development of software.
The following challenge still remain an open-question for the software engineering community:
\vspace{-2em}
\highlightbox{How to engineer uncertainty-aware software (design, implement, test, and validate)?}

\subsection{Actions have long-term effects}
\label{intro:challenges:long-term-act}
Decision-making processes follow the growing complexity of software.
They are more and more able to not only make decisions on the current state of the system, but also its past and future ones.
And this decision may have also long-term effects.

Due to this complexity, developers and users may misunderstand the decisions taken by a system.
Plus, designers may neglect or underestimate the impact of a decision.
Moreover, as highlighted by Bencomo\etal, systems should be self-explained.
They should be able to explain the decisions made.

To achieve this vision and to help designers and users understanding the impact of a decision, we argue that the software engineering community should address the following question:
\vspace{-2em}
\highlightbox{How to represent, query, store, and understand the impacts of long-term actions?}

\subsection{Systems may have emergent behaviours}
\label{intro:challenges:ermger-bhv}
The growing complexity of systems have also another impact: they have emergent behaviour.
This behaviour may be suboptimal and hard to understand by designers, who generally have a local vision of the system.

However, when this behaviour lead to failure, engineer still need to understand why and how to avoid a novel occurrence of the problem.
Plus, as the behaviour might be suboptimal, they need to optimise it.

To reach this goal, engineers need tooling support to help them in their investigation process.
In other words, the research community should answer the following global challenge:
\vspace{-2em}
\highlightbox{How to understand, predict, and optimise emergent behaviours?}

\subsection{Different part of a system evolve at different paces}
\label{intro:challenges:diff-paces}
\highlightbox{How to represent, query, and store inconsistent system states and behaviours?}

\subsection{Evolution of systems are linked with time}
\label{intro:challenges:evol-syst}
\highlightbox{How to structure, represent query, and store efficiently temporal data at a large scale?}