\section{Introduction}

%% Context
Utilities introduce more and more \gls{ict} in the grid in order to face the new challenges of electricity delivery~\cite{farhangi2010path, ipakchi2009grid, DBLP:journals/comsur/FangMXY12}.
These novel power grids are referred to as \gls{sg}.

Among the different features that they possess, the one that will interest us in this document is the \textbf{\gls{shealing}} one.
As defined by Jeffrey O. Kephart and David M. Chess~\cite{DBLP:journals/computer/KephartC03}, a system can be described as a \gls{shealing} system if it can automatically fix any incident at runtime.
For example, a smart grid can automatically optimize the power flow in order to deal with failures of transformers\footnote{Transformers change the voltage in cables.}~\cite{DBLP:journals/comsur/FangMXY12}.

This healing mechanism can be performed only if the smart grid has a deep understanding of itself and its environment.
To tame their complexity, a common approach in software engineers is to use an \textbf{abstraction} mechanism.
Abstractions provide an illuminating description of systems, their behaviors, and/or their environment.
For example, Hartmann~\etal \cite{DBLP:conf/smartgridcomm/0001FKTPTR14} provide a class diagram that describes the smart grid topology, when it uses power-line communications\footnote{Data are sent through cables that also distribute electricity.}.

\bigskip

More generally, a \gls{shealing} is a \gls{sadapt}. 
Cheng \etal define \glspl{sadapt} as \textquote{systems that are able to adjust their behaviour in response to their perception of the environment and the system itself}~\cite{DBLP:conf/dagstuhl/ChengLGIMABBBCSDFGGGKKKLMMMPSTTWW09}.
Jeffrey O. Kephart and David M. Chess~\cite{DBLP:journals/computer/KephartC03} laid the groundwork of this approach, based on a IBM white paper~\cite{computing2006architectural}.
Since then, it has been used in different domain~\cite{DBLP:journals/corr/abs-1904-01518} such as cloud infrastructure~\cite{DBLP:conf/icac/JavadiG17, OpenStack:Watcher:Wiki, DBLP:conf/icse/BarnaKFL17} or \gls{cps}~\cite{DBLP:conf/icac/LalandaGC17, DBLP:conf/cbse/FouquetMFBPJ12, DBLP:conf/smartgridsec/0001FKNT14}.

\Gls{mde} uses the abstraction mechanism to facilitate the development of nowadays software~\cite{DBLP:journals/computer/Schmidt06, DBLP:conf/ifm/Kent02, DBLP:series/synthesis/2017Brambilla}.
This methodology can be applied on the different stages of software development.
In this thesis, we focus on one of its paradigm: \gls{m@rt}~\cite{DBLP:journals/computer/BlairBF09, DBLP:journals/computer/MorinBJFS09}.







%% Problematic

%% Challenge

%% Vision

%% Contribution

%% Validation

%% Remainder