\section{Introduction}

%% Context
Utilities introduce more and more \gls{ict} in the grid in order to face the new challenges of electricity delivery~\cite{farhangi2010path, ipakchi2009grid, DBLP:journals/comsur/FangMXY12}.
These novel power grids are referred to as \gls{sg}.
Among the different features that they possess, the one that will interest us in this document is the \gls{shealing} one.
As defined by Jeffrey O. Kephart and David M. Chess~\cite{DBLP:journals/computer/KephartC03}, a system can be described as a \gls{shealing} system if it can automatically fix any incident at runtime.
For example, a smart grid can automatically optimize the power flow in order to deal with failures of transformers\footnote{Transformers change the voltage in cables.}~\cite{DBLP:journals/comsur/FangMXY12}.

This healing mechanism can be performed only if the smart grid has a deep understanding of itself and its environment.
It can be provided by an abstraction 
For example, Hartmann~\etal \cite{DBLP:conf/smartgridcomm/0001FKTPTR14} provide a class diagram that describes the smart grid topology, when it uses power-line communications\footnote{Data are sent through cables that also distribute electricity.}.

\bigskip

More generally, a \gls{shealing} is a \gls{sadapt}. 
Based on a IBM white paper~\cite{computing2006architectural}, Jeffrey O. Kephart and David M. Chess laid the groundwork of the (self-)\gls{adptSyst}.
Since then, it has been used in different domain~\cite{DBLP:journals/corr/abs-1904-01518} such as cloud infrastructure~\cite{DBLP:conf/icac/JavadiG17, OpenStack:Watcher:Wiki, DBLP:conf/icse/BarnaKFL17} or \gls{cps}~\cite{DBLP:conf/icac/LalandaGC17, DBLP:conf/cbse/FouquetMFBPJ12, DBLP:conf/smartgridsec/0001FKNT14}.

 







%% Problematic

%% Challenge

%% Vision

%% Contribution

%% Validation

%% Remainder