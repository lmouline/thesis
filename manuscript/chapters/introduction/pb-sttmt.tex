\section{Problem statement}

During our study, we have identified several characteristics of \glspl{adptSyst} that bring challenges to the software engineering research community.
First, information gathered is not always known with absolute confidence.
Second, reconfigurations may not be immediate, and their effects are not instantaneously measured.
Third, system behaviour may be emergent~\cite{zio2011uncertainties}, \ie it cannot be entirely known at design time.
Four, the different sub-parts of the system do not evolve at the same rate.
Five, structure and behaviour of systems have a time dimension.


\subsection{Data are uncertain}
Most fuses are manually open and close by technicians rather than automatically modified.
Then, technicians manually report the modifications done on the grid.
Due to human mistakes, this results in errors.
The grid topology is thus uncertain.
This uncertainty is propagated to the load approximation, used to detect overloads in the grid.
Wrong reconfigurations might be triggered, which could be even worse than if no change would have been applied.

More generally, \textbf{data are, almost by definition, uncertain and developers always work with estimates}~\cite{DBLP:conf/asplos/BornholtMM14, metrology2008evaluation, DBLP:journals/tkde/AggarwalY09}.
The uncertainty may be explained by how data are collected.
We can distinguish three categories: sensors, humans, and results of computations.
Sensors (software or hardware) always estimate the value and have a precision value due to the method of measurement~\cite{metrology2008evaluation, DBLP:conf/asplos/BornholtMM14}.
Humans are error-prone.
And computations can either give an approximation or be based on uncertain data.
This uncertainty is then propagated through all steps until the final result.

For a specific domain, this uncertainty may impact the understanding of the real situation.
For example, the uncertainty of the \gls{cpu} clock is too low to damage the percentage load of the processor.
However, the uncertainty of the \gls{gps} may impact the understanding by, for instance, showing the user on the wrong road (compared to the real one).
\textbf{If the data uncertainty can mislead the understanding of a system behaviour or state, then developers should implement an uncertainty-aware system.}
For \glspl{adptSyst}, this lack of confidence may trigger suboptimal adaptations.


\subsection{Actions have long-term effects}
Reconfiguring a smart grid implied to change the power flow.
It is done by connecting or disconnecting specific cables.
That is, opening or closing fuses.
As said before, technicians need to drive physically to fuse locations to modify their states.
Besides, in the case of the Luxembourg smart grid, meters send energy measurement every 15 min, non-synchronously.
Between the time a reconfiguration of the smart grid is decided, and the time the effects are measured, a delay of at least 15 min occurs.
On the other hand, an incident should be detected in the next minutes.
If the adaptation process does not consider this difference of rates, it can cause repeated decisions.

\textbf{Actions are never immediate, take time to be executed, and have long-term effects.}
Through this thesis, we refer to such actions as \glspl{longTermAct}.
In computer science, the definition of "immediate" is specific to a domain.
For example, for graphical user interfaces, a response time of less than 200 ms is considered as immediate.
However, while working at the processor level, the execution time of one instruction is measured at the nanoseconds scale.

Not considering this delay may lead to sub-optimal decisions.
For example, not considering the delay for the system to handle the migration of a virtual machine may lead to repeat decisions.
We argue that \textbf{developers should take into account this delay if the frequency of the monitoring stage is lower than the time of action effects to be measurable}.

\subsection{Systems may have emergent behaviours}
Smart grid behaviour is affected by several factors that cannot be controlled by the grid manager.
One example is the weather conditions.
Smart grids rely on an energy production that is distributed over several actors.
For instance, users, who were mainly consumers before, now produce energy by adding solar panels on the roof of their houses.
The production of such energy depends on the weather, and even on the season\footnote{The angle of the sun has an impact on the amount of energy produced by solar panels. This angle varies according to the season.}.
Another example is the increasing adoption of electric vehicles, which de facto drastically increase the consumption of electricity during the night.
Ignoring this characteristic of \glspl{adptSyst} may result in suboptimal situations that can be understood with difficulties.

\textbf{System behaviour may be emergent.}~\cite{zio2011uncertainties}
Different factors can explain this phenomenon.
As for \glspl{sg}, systems may evolve in a stochastic and uncertain environment.
Or, some system, like those name ultra-large systems, are so complex that a few engineers can tame it.
But, groups of engineers will have an understanding of some part of the system.

Despite the complexity, engineers still need to understand how the system is behaving or behaved.
It will help them to optimise the global behaviour or to understand and repair errors.
\textbf{Engineers need tooling support to trace back previous behaviours and or replay them.}


\subsection{Different part of a system evolve at different rates}
Every meter sends consumption and production data every 15 min.
However, this collection is not synchronous.
That is, all meters do not send their data at the same timestamp.
The global system, which receives all data, has not thus a global vision with the same freshness for all the part of the grid.
Electricity data are volatile: a peak or a drop may happen in less than a minute due to, for instance, the starting or the finishing of a washing machine.
When analysing the data, the process should consider this difference of rates, and estimate the evolution of data.
Otherwise, they will reason over outdated data, which do not reflect the real situation.
It may lead to suboptimal decisions.

\textbf{Different parts of the same system may evolve at different rates.}
Some systems are heterogeneous in terms of hardware and software.
This diversity results in different evolution or reaction rates.
For example, if some components are working on batteries, they will have a sleep cycle to save energy.
Contrary, if some others are running connected directly to a power source, they can react faster.

Despite this difference of rates, a global vision of a system at a precise time point may be still required.
The vision should deal with data that have different freshness.
In the worst case, some data would be outdated and cannot be used.
\textbf{Solutions to seamlessly predict or estimate what should be the current state of these outdated elements are thus required.}


\subsection{Evolution of systems is linked with time}
Power flow is impacted by consumption and production of users, and by the modifications of the topology.
Knowing the last status of the grid is as important as knowing how it evolves.
Based on the evolution, the grid operator can predict any future incidents, like overloads.
It could also compare this evolution of behaviour with a normal one to detect, for example, any malicious behaviour.

\textbf{Evolution of systems is inherently linked with a time dimension.}
Evolution and time are two related concepts.
For some systems, not only the last states are important but also how they evolve.
Then, analysis processes will investigate this evolution to know if it is a normal one or not.
They can also use this evolution to predict how systems will evolve.
Based on these predictions, they can proact on future incidents in the system.

Decisions are not made upon the last state of the system but how it evolves.
The analysis process should thus navigate both in the structure of the system and its behaviour over time.
\textbf{Engineers need efficient tooling to structure, represent, query, and store temporal data on a large scale.}





