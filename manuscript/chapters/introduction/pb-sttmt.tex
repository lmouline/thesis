\section{Problem statement}

Whereas \glspl{sg} introduce more and more automation capacity, human interventions are still required.
First, information gathered by smart grids is therefore not always known with absolute confidence.
Second, smart grids reconfigurations are not immediate, and their effects are not instantaneously measured.
Third, smart grids behaviour is emergent~\cite{zio2011uncertainties}, \ie it cannot be entirely known at design time.

Most fuses are manually open and close by technicians rather than automatically modified.
Then, technicians manually report the modifications done on the grid.
Due to human mistakes, this results in errors.
The grid topology is thus uncertain.
This uncertainty is propagated to the load approximation, used to detect overloads in the grid.
Wrong reconfigurations might be triggered, which could be even worse than if no change would have been applied.

Reconfiguring a smart grid implied to change the power flow.
It is done by connecting or disconnecting specific cables.
That is, opening or closing fuses.
As said before, a technician needs to drive physically to the fuse location to modify its state.
Besides, in the case of the Luxembourg smart grid, meters send energy measurement every 15 mins, non-synchronously.
Between the time a reconfiguration of the smart grid is decided, and the time the effects are measured, a delay of at least 15 mins occurs.
On the other hand, an incident should be detected in the next minutes.
If the adaptation process does not consider this difference of paces, it can cause repeated decisions.

Smart grid behaviour is affected by several factors that cannot be controlled by the grid manager.
One example is the weather condition.
Smart grids rely on an energy production that is distributed over several actors.
For instance, users, who were mainly consumers before, now produce energy by adding solar panels on the roof of their houses.
The production of such energy depends on the weather, and even on the season\footnote{The angle of the sun has an impact on the amount of energy produced by solar panels. This angle varies according to the season.}.
Another example is the increasing adoption of electric vehicles, which de facto drastically increase the consumption of electricity during the night.
Ignoring this characteristic of \gls{adptSyst} may result in suboptimal situations that can be understood with difficulties.






\textbf{Data are uncertain, actions have delayed effects, and the behavior of \glspl{adptSyst} is emergent.}

Data are, almost by definition, uncertain and developers always work with estimates.
This is due to how data are collected.
We can identify three different sources.
They can .....
\textbf{Uncertainty impacts the understanding of the global situation}

Actions....
\textbf{if the frequency of the monitoring stage is lower than the time of action effects to be measurable}


Behavior ...
\textbf{The behavior cannot be perfectly known at design time or inferred from the set of executed actions.}
