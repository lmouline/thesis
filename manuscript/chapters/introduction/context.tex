\section{Context}
Utilities are introducing more and more \gls{ict} in the grid to face the new challenges of electricity supply~\cite{farhangi2010path, ipakchi2009grid, DBLP:journals/comsur/FangMXY12}.
The literature and the industry refer to these nowadays power grids as \gls{sg}.

In this document, we focus on the \textbf{\gls{shealing}} capacity of such grids.
A \gls{shealingSyst} can automatically repair any incident, software or hardware, at runtime~\cite{DBLP:journals/computer/KephartC03}.
For example, a smart grid can optimise the power flow to deal with failures of transformers\footnote{Transformers change the voltage in the cables.}~\cite{DBLP:journals/comsur/FangMXY12}.
In this way, the incident will impact as few users as possibles, ideally none. 

This healing mechanism can be performed only if the \gls{sg} has a deep understanding of itself (its \gls{structure} and its \gls{behaviour}) and its \gls{env} (where it is executed).
This understanding can be extracted from an, or a set of, \textbf{abstraction}(s) of these elements.
Abstractions provide an illuminating description of systems, their \glspl{behaviour}, or their \glspl{env}.
For example, Hartmann~\etal \cite{DBLP:conf/smartgridcomm/0001FKTPTR14} provide a class diagram that describes the smart grid topology, when it uses power lines communications\footnote{Data are sent through cables that also distribute electricity.}.

\bigskip

More generally, a \gls{shealingSyst} is a \textbf{\gls{sadapt}}. 
\Glspl{sadapt} optimize their \glspl{behaviour} or configurations at runtime in response to a modification of their \glspl{env} or their \glspl{behaviour}~\cite{DBLP:conf/dagstuhl/ChengLGIMABBBCSDFGGGKKKLMMMPSTTWW09}.
Kephart and Chess~\cite{DBLP:journals/computer/KephartC03} laid the groundwork for this approach, based on an IBM white paper~\cite{computing2006architectural}.
Since then, practitioners have applied it to different domains~\cite{DBLP:journals/corr/abs-1904-01518} such as cloud infrastructure~\cite{DBLP:conf/icac/JavadiG17, OpenStack:Watcher:Wiki, DBLP:conf/icse/BarnaKFL17} or \gls{cps}~\cite{DBLP:conf/icac/LalandaGC17, DBLP:conf/cbse/FouquetMFBPJ12, DBLP:conf/smartgridsec/0001FKNT14}.

\textbf{\Gls{mde}} uses the abstraction mechanism to facilitate the development of nowadays software~\cite{DBLP:journals/computer/Schmidt06, DBLP:conf/ifm/Kent02, DBLP:series/synthesis/2017Brambilla}.
This methodology can be applied to different stages of software development.
In this thesis, we focus on one of its paradigms: \textbf{\gls{m@rt}}~\cite{DBLP:journals/computer/BlairBF09, DBLP:journals/computer/MorinBJFS09}.
The state of the system, its \gls{env}, or its \gls{behaviour} is reflected in a \gls{model}, used for analysis.
Developers can use this paradigm to implement \glspl{adptSyst}~\cite{DBLP:journals/computer/MorinBJFS09, DBLP:conf/smartgridsec/0001FKNT14}.

\bigskip

In this thesis, we focus on the use of \gls{mde} techniques, and more specifically, the \gls{m@rt} paradigm, for the implementation of \glspl{sadapt}.
Adaptation processes use \glspl{model} to have a deep understanding of the system, its structure and its behaviour, and its environment.
Using the vocabulary of the research community, the \gls{model} represents the \gls{knowledge}. 
That is, \textbf{we studied the representation of the \gls{knowledge} of \glspl{adptSyst}, using the \gls{m@rt} paradigm.}








