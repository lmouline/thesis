\section{Perspectives}
B oth our contributions suffer from limitations that call for further research.
First, our language \langName{} do not include all the different kinds of uncertainty representations, uncertainty is only applied on simple data type, and the propagation is limited to some of the language operators (arithmetic, boolean, and comparison).
Second; when we define our temporal knowledge model, we assume that designers are able to link actions with their expected impacts at design time and the final \gls{metamodel} remains large.
In this section, we describe research directions that could remove these limitations and other perspectives let by our work.


\subsection{Other uncertainty representation}
Our current work addresses only uncertainty on values and with a limited number of probability distributions.
However, other kinds of uncertainties exist such as uncertainty of existence or temporal uncertainty.
The first one corresponds to the confidence that a value exists or not.
It can result from faulty data sources that send wrong data.
The second one can be used to represent the loss of confidence in value over time.
Moreover, researchers defined other strategies to represent uncertain data such as keeping multiple possibilities.

A first future work would be to investigate how to introduce such techniques inside a programming language.
A first research effort can be to define a language that uses different strategies to handle uncertainty.
It will open new challenges regarding the type system, the semantics, and the syntax of the language.

Second, introducing new probability distributions lead to complex combination of probability distributions.
In the current approach, we use an analytical approach (we compute the exact solution).
However, this cannot be performed between some probability distributions.
In such situation, a numerical method should be applied.
This lead to challenges regarding the threshold between the performance of the language and the accuracy of the method.

\subsection{More complex data structure}
In our language, we focus our studies on the primitive data types (numeric and boolean) and references (1:1 relation).
However, it exists several other data structures, from the simplest ones like arrays to complex ones like graphs or trees.
In the \gls{uml} specification, different kind of relations have been defined (1:1, 1:n, n:n, \etc).

While these data structure and relations are useful to build algorithms to reason over data, they come with new challenges.
First, it opens new question concerning the meaning of an uncertain data structure: what an uncertain array? what an uncertain tree, graph?
Let us take an uncertain array.
The uncertainty can be related to the full collection, or on each element, or on both.
Plus, research efforts has to be done to specify the semantics of operators on these uncertain structure.
For example, one may focus on how the uncertainty of the array will be impacted by an \textit{add} or \textit{remove} operation.
Lastly, one may investigate the impact of introducing uncertain data structure in common algorithms.
For example: how to sort an uncertain collection? how to balance an uncertain tree? how to compute the shortest path?

\subsection{Impact of uncertainty to control flow}
%In our language, we map the uncertainty propagation to operators.
%We did not study the impact of control flow statements on the uncertainty propagation.
%Introducing uncertainty as a first-class citizen will inevitably modify current behavior of control flow statements such as \textit{IF}-conditions.
%We strongly think that this new data type in the type system will lead to further research directions.
%However, we identify two other situations that should be considered, with their challenges.
%First, the propagation through control flow statements: how the uncertainty is propagated after an \textit{IF}-condition, a \textit{FOR}-loop and a \textit{WHILE}-loop.
%Second, the propagation from one kind of uncertainty to another one.
%For example, how the uncertainty of presence should be propagated to the uncertainty of a sum, average or variance computation?


%Conditional expressions, which have boolean type, modify the control flow by forking it.
%As uncertain booleans have similar semantic to certain ones, the question that we wonder was: should we allow conditional expressions with uncertain boolean type?
%
%With certain one, the expression is evaluated at runtime and one branch is selected according to the result.
%With uncertain ones, the executor cannot decide which branch to execute.
%We thus identified two possible controls flows: the classical one and the uncertain one.
%
%For the classical one, an implicit operation should be performed to get a certain one.
%However, it can be done using at least two strategies.
%First, a random selection can be made using the draw operator.
%It is specially used in the probability programming domain.
%By executing several times the same piece of code, it allows inferring the distribution of the result without propagating the uncertainty by applying the probability formula.
%Second, a cast can be done using the cast or confidence operator.
%This allows going to a certain world.
%
%For the uncertain execution, as the executor cannot decide which branch should be executed, it should execute all of them and propagate the uncertainty throw the execution.
%For example, let us imagine the following code:
%
%\begin{lstlisting}[style=javaStyle, caption=Example for uncertain control flow, label=lst:ucf-bool-example]
%uncertain_bool b = (TRUE, 0.4)
%uncertain_int n;
%if(uncertain_bool)
% n = (5, 0.8)
%else
% n = (-5, 0.8)
%\end{lstlisting}
%
%As \textit{b} is true with a confidence of 40\%, n should be equal to 5 with a confidence of (40\% * 80\% = 32\%) and to -5 with a confidence of (60\% * 80\% = 48\%) (considering b and n independent).
%
%These three execution semantics come with several questions.
%For the classical one: how to combine different execution semantic? what are the impact on performances?
%For the other one: should the execution be parallel or not? how to ensure that the execution do not have side effect(s)? what are the impact on the performance?
%
%As we focus on integrating uncertainty propagation to data type, we decide to let all these questions for future research.
%Plus, we do not want to choose an arbitrary default semantic.
%We thus forbid the usage of uncertain boolean as condition of \textit{IF} and \textit{WHILE} expression.
%We thus cannot allow implicit conversion in our language.
%Developers must then dismiss the uncertainty using the uncertainty operators.


\subsection{Unknown effects of actions}
%Throughout this work, we assumed that designers are able to link actions with their expected impacts at design time. 
%However, this is not always true. 
%Some impacts  cannot be known in advance. 
%In this perspective, in addition to the future plans already mentioned throughout the paper, we will investigate techniques to identify unknown impacts on the context model, for instance, by studying the use of machine learning  techniques. 

%Throughout this work, we assumed that designers are able to link actions with their expected impacts at design time. However, all impacts are not known in advance by the designer: some could be unknown or not precisely known (the impact cannot be characterized at design time).
%In this perspective, in addition to the future plans already mentioned throughout the paper, we will investigate techniques to specify these unknown parts of the context model. 
%Additionally, we plan to study the use of analytics techniques to identify unknown impacts or to specify the unknown of the impacts.

\subsection{Uncertain effects of actions}
% As explained in Section~\ref{sec:adaptation-req}, data uncertainty is a characteristic of knowledge information.
% Moreover, links between action and their impacts are also uncertain.
% All traceability links that we model in this paper should not be considered as strongly accurate.
% Indeed, action impacts can be mixed up with the constant evolution of the system.
% Without considering uncertainty, diagnosis routines could end up with inaccurate fault or suboptimal state analyses.
% The approach needs to be extended with uncertainty management, that remains an open question in the modeling community.
% Uncertainty raises two kinds of questions: how to model it to ease its manipulation by an engineer? How to propagate it automatically from data to complex processes like diagnosis routines?

\subsection{Manipulation and population of the large model}
% In order to address larger systems, one of our future work consists in improving our approach to handling distributed adaptive systems.
% Plus, our approach will benefit from a language to manipulate the model.

\subsection{Uncertainty evaluation and reduction}
