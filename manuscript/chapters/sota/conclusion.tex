\section{Conclusion}
%\label{sec:sota:conclusion}

In this chapter, we review the state-of-the-art approaches to answer two research questions.
First, we look for studies that model \glspl{adptSyst}, their \glspl{context} or \glspl{behaviour} to see if they also consider \gls{longTermAct} (RQ1).
Our review shows that none of the current approaches model or enable reasoning over \glspl{action} with delayed effects.
Thus, some research efforts are still required to specify solutions that allow designers to add \gls{longTermAct} in their model and to implement techniques to reason over them.
In this thesis, we start studying this problem, and we present a model-based solution, detailed in \Cref{chapt:tkm}.

Then, we search for solutions that model \gls{duc}.
Different solutions have been found in our review, the main one being referred to as probabilistic programming.
This solution allows developers to manipulate probability distributions as common variables of a programming language.
However, as seen in our review, not all kinds of uncertainty can be represented by this approach.
For example, some researchers represent the uncertainty of a value with a set of different possibilities.
Therefore, open challenges still need research efforts to handle \gls{duc}.
Towards solving these challenges, we start by defining a language that integrates \gls{duc} as a first-class citizen (\cf \Cref{chapt:aintea}).





%Through this review, we found that none of the state-of-the-art solutions model \glspl{action}, their effects, and their \glspl{circumstance} over time (RQ1), both at design time and at runtime.
%Some approaches will partially model these elements, for example, \glspl{action} and their effects at design time.
%However, these models cannot be automatically navigated by a process, like a reasoning engine.
%Moreover, some approaches do not consider the temporal dimension of the effects.
%
%Our findings show that different community have studied uncertainty in software.
%For instance, modelling community has focused on design uncertainty and \gls{adptSyst} community has studied \gls{env} uncertainty.
%Researchers have put a lot of efforts to encapsulate probability distributions behind programming language concepts such as variables.
%But we strongly think that higher-level abstraction should be defined to help engineers.
%Additionally, tools to help developers to manipulate uncertain data are needed.
