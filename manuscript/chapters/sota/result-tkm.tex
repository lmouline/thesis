\section[Results RQ1: long-term actions]{Results RQ1: \glspl{longTermAct}}
\label{sec:sota:results:actions}

% 1.1 %
\subsection[Modelling the evolution of system's context or behaviour]{Modelling the evolution of system's context or \gls{behaviour}}

%%% Context %%%

\paragraph{Modeling paradigm}
- \cite{DBLP:journals/computer/BlairBF09, DBLP:journals/computer/MorinBJFS09}
	- causality link between a model and the system
	- model should reflect the system, modification of the model triggers modification of the system
- \cite{DBLP:conf/seke/0001FNMKT14, DBLP:conf/models/0001FNMKBT14}
	- extend \gls{m@rt} to introduce time in the model

\paragraph{Formal model} %% Context %%
- \cite{DBLP:journals/taas/WeynsMA12}
	- reference model for AS
	- no time consideration
- \cite{DBLP:journals/taas/WeynsHH10}
	- use the Z Language~\cite{DBLP:books/daglib/0011651}
		- language based on the set theory and first order predicate calculus
		- used for formal specification
	- behaviour model with "laws"
		- functions from one set to another
- \cite{DBLP:conf/icse/BartelsK11}
	- use of Communicating Sequential Process

\paragraph{Low level model} %% Context %%
- \cite{DBLP:conf/dbpl/MoffittS17}
	- low level model: temporal graph
	- can be used by higher level model tool
	- define a temporal graph
	- add 2 functions:
		- one which indicate the existence of a node/edge for a time period
		- one to retrieve the value of a node or an edge at a time period
		
\paragraph{Object-based model} %% Context %%
- \cite{DBLP:conf/pervasive/HenricksenIR02}
	- define a model for pervasive computing systems\footnote{A pervasive system is composed of cheap and interconnected devices that are ubiquitous and can support users' tasks.\cite{DBLP:conf/pervasive/HenricksenIR02}}
	- model use object-based approach
	- define entity, and attribute, both link with uni-directional association
	- these associations can be dynamic, one category of dynamically is temporal one: the association ends with a set of values attached to a timestamp
- \cite{DBLP:conf/smartgridsec/0001FKNT14}
	- model the context and its history using following the \gls{m@rt} paradigm
- \cite{DBLP:conf/icse/TaharaOH17}
	- use of the Maude~\cite{DBLP:journals/tcs/ClavelDELMMQ02} language  to represent the context
	- no time aspect
	
	
\paragraph{Goal model} %% Context %%
- \cite{DBLP:conf/icse/CailliauL17}
	- use a probabilistic goal model, can be thought as a high-level model of the context
		- probability: describes the satisfactory rate of each goals
- \cite{DBLP:conf/icse/IftikharW14a, DBLP:conf/icse/MendoncaAR14, DBLP:conf/icse/ChenPYNZ14}

\paragraph{State machine} %% Behaviour + context %%
- \cite{DBLP:conf/smartgridsec/0001FKNT14}
	- behaviour of the system model with a \gls{fsm}
- \cite{DBLP:conf/icse/IftikharW14a}
	- use of a \gls{fsm}
- \cite{DBLP:conf/icse/ArcainiRS15, DBLP:conf/rv/ArcainiGR11}
	- use of an \gls{asm}
- \cite{DBLP:conf/icse/BarbosaLMJ17}
	- probabilistic labelled transition system (probabilistic state machine) 
		- expected values set at design time, then compared with those get at runtime
- \cite{DBLP:journals/computing/BencomoBGBI13}
- \cite{DBLP:conf/sigsoft/MorenoCGS15}
	- use a Markov model, which a probabilistic model
	- context contains action that are executing with their progress status
	- decision look ahead, effects are "predicted" -> probability to reach next state
	- no history
- \cite{DBLP:conf/kbse/FilieriGLM11, DBLP:conf/dagstuhl/GhezziS10}
	- Discrete Time Markov Chain
- \cite{DBLP:conf/icse/DuarteMS18}
	- Labelled Transition Model (Markov Chain)
- \cite{DBLP:conf/wetice/DjoudiBZ14}
- \cite{DBLP:conf/aosd/ZhangGC09}
- \cite{DBLP:conf/icse/GhezziPST13}
	- FSM
	- sate=functionalities with their impact on the system, edges=possible executable flow
- \cite{DBLP:conf/kbse/TajalliGEM10}
	- 2 models: 1 domain and 1 adaptation

\paragraph{Sequential diagram}
- \cite{DBLP:conf/icse/TaharaOH17}

	
\paragraph{Component model} %% Context %%
- \cite{DBLP:conf/soco/DavidL06}
	- use Fractal~\cite{DBLP:conf/cbse/BrunetonCLQS04}, a component model
- \cite{DBLP:conf/wetice/DjoudiBZ14}
	- extension of Maude
	- the description of the component contain also behaviour description
		- sate machine description
	
\paragraph{Trace models}
- \cite{DBLP:journals/computer/Maoz09}
	- create model by analysing the traces of the system and reflect the runtime state of the system
	
\paragraph{Graph model}
- \cite{DBLP:journals/tse/KramerM90}
	- node=process unit
	- edge=communication
	
	
	
	
	
	
	
	
	
% 1.2 %		
\subsection[Modelling actions, their circumstances, and their effects]{Modelling \glspl{action}, their \glspl{circumstance}, and their effects}

\paragraph{Rule-based approach}
- \cite{DBLP:conf/icse/TaharaOH17}: use the Maude language to define the adaptation mechanism
	- rewriting rules
	- rl [ (label)] : (source)=> (target)
		- label: name of the rule
		- source: init state of the system
		- target: final state of the system
	- source/target => circumstances/effects of action
	 - but no time dimension, side effects, no delayed effects, no action execution tracing
- \cite{DBLP:conf/icse/ArcainiRS15}
	- define a set of rules to change state of the system
	- circumstances = init state + conditions
	- effects = target states
		- but no time consideration, side effects, delayed effects
- \cite{DBLP:conf/wrla/BruniCGLV12}
	- apply rewriting rules
	- no effects 
- \cite{DBLP:conf/eurosys/GraceHPBCT08}
	- no effects
	- action = scripts
- \cite{DBLP:conf/gpce/PintoFT03}
	- rules to weave adaptation
		
\paragraph{State machine}
- \cite{DBLP:conf/icse/ArcainiRS15} use of an \gls{asm}
	- use rules to define state transition
- \cite{DBLP:conf/icse/IftikharW14a}
- \cite{DBLP:conf/smartgridsec/0001FKNT14}
- \cite{DBLP:conf/sigsoft/MorenoCGS15}
	- Markov decision process
- \cite{DBLP:conf/kbse/FilieriGLM11}
	- Discrete Time Markov Chain
- \cite{DBLP:conf/wetice/DjoudiBZ14}
- \cite{DBLP:conf/aosd/ZhangGC09}
- \cite{DBLP:conf/icse/GhezziPST13}
- \cite{DBLP:conf/kbse/TajalliGEM10}
		 
\paragraph{Publish/Subscribe approach}
- \cite{DBLP:conf/icse/BarbosaLMJ17}
	- adaptation mechanism is triggered following the publish/subscribe mechanism
	- no additional information
	- circumstances = conditions
	- no effects modeled, no time dimension, model cannot be navigable
	
\paragraph{Goal-modelling}
- \cite{DBLP:conf/icse/MendoncaAR14, DBLP:conf/iceccs/BencomoWSW12}
	- actions are part of the goal model, and are represented as mean to achieve goals
	- effects are thus represented as fulfilment of a goal
	- no circumstances modelled
	- high-level abstraction of actions
	- no time consideration
	
\paragraph{Programming language}
- \cite{DBLP:journals/jss/ChengG12} defines a language to define the adaptation mechanism
	- here, talk about strategies
	- the language allow also to describe the effects of the strategies 
	- condition can be thought as the circumstances of the streategy
	- time aspect: time out for the execution of a strategy
	- nothing for "temporal effects" / cannot be navigated / nothing about the runtime, only design time
	
\paragraph{Temporal planning}
- \cite{DBLP:conf/aaai/CimattiMR15}
	- formalise the planning following temporal planning principle
	- define planning algorithm for a set of actions (not precisely define)
	- planning consider the time of the execution of the action, that might not be controllable (uncertain)
	- nothing about the effects of their circumstances
	 
\paragraph{Event-Condition Action}
- \cite{DBLP:conf/soco/DavidL06} defines FScript
	- action: weave/remove aspects, or control component (start, stop, replace)
 	- no time, effect
 	- circumstances = condition
 	- not navigable
 	- only design time
- \cite{DBLP:conf/icws/CharfiDM09}
	- aspects have pre/post condition to trigger the adaptation mechanism (weaving or not an aspect) 
- \cite{DBLP:journals/scp/ParraBCD11}
	- aspects have a condition, triggered by events
	
\paragraph{Model transformation}
- \cite{DBLP:conf/icse/ChenPYNZ14}
	- use Query/View/Transofrmation~\cite{QVT:Spec} approach
	- then transfo trigger adaptations
	- no effects
	
\paragraph{Graph modification}
- \cite{DBLP:journals/tse/KramerM90}
	- graph changes: adding/removing node(s)/edge(s)
	
\paragraph{Formal model}
- \cite{DBLP:journals/taas/WeynsHH10}
	- function from set to set
	- represent adding,removing merging, splitting organisations (group of agents/components)
- \cite{DBLP:conf/icse/BartelsK11}
	- speical stuff......
	
\paragraph{Dynamic Software Product-Lines}
- \cite{DBLP:conf/dagstuhl/GhezziS10}
	- model variation point in a SPL model
	 
	 
	 
	 
	 
% 1.3 %
\subsection{Reasoning over evolving context or behaviour}

\paragraph{Model-based approach}
- \gls{m@rt} \cite{DBLP:journals/computer/BlairBF09, DBLP:journals/computer/MorinBJFS09}
	- extension to support temporal axis \cite{DBLP:conf/seke/0001FNMKT14, DBLP:conf/models/0001FNMKBT14}
- Lotus@Runtime~\cite{DBLP:conf/icse/BarbosaLMJ17}
- \cite{DBLP:conf/icse/ChenPYNZ14}
		
\paragraph{Rule-based adaptation}
- \cite{DBLP:conf/icse/ArcainiRS15, DBLP:conf/icse/TaharaOH17, DBLP:conf/eurosys/GraceHPBCT08}

\paragraph{Architecture-based adaptation}
- \cite{DBLP:journals/jss/ChengG12} defines a language for architecture-based adaptation

\paragraph{Simulation-based adaptation}
- \cite{DBLP:conf/smartgridsec/0001FKNT14}
	- simulate different sequences of actions, and evaluate them to evaluate optimal one

\paragraph{Formal model}
- \cite{DBLP:journals/taas/WeynsMA12}: Forms
- \cite{DBLP:conf/icse/IftikharW14a}, extension of FORMS
	- define a global formal model that represent abstract \gls{sadapt}, here a state machine
- \cite{DBLP:journals/taas/WeynsHH10}
	- formal model for self-organisation
- \cite{DBLP:conf/icse/BartelsK11}
	- speical stuff......

\paragraph{Complex Event Processing}
- \cite{DBLP:conf/rr/AnicicFRSSS10}

\paragraph{Graph model}
- \cite{DBLP:conf/dbpl/MoffittS17}: temporal graph algebra
	- can be used as low-level "basis" to define a temporal context and behaviour
- \cite{DBLP:journals/tse/KramerM90}
	- representation of the context as a graph, and reason over it
	
\paragraph{Aspect oriented programming}
- \cite{DBLP:journals/taosd/GreenwoodB06, DBLP:conf/soco/DavidL06, DBLP:conf/icws/CharfiDM09, DBLP:journals/scp/ParraBCD11, DBLP:conf/ewsa/FalcarinA04, DBLP:conf/gpce/PintoFT03}: dynamic Aspect Oriented Programing
	- adaptation by dynamically weaving/removing aspects in the program
	- no time aspect, no effects
	- circumstances == conditions to remove/weave an aspect

\paragraph{Component-based adaptation}
- \cite{DBLP:conf/soco/DavidL06}: adaptation by modifying component of the system

\paragraph{State machine}
- \cite{DBLP:conf/sigsoft/MorenoCGS15}
	- Markov Decision Process
	- actions (strategies) are triggered by a condition
- \cite{DBLP:conf/kbse/FilieriGLM11}
	- Discrete Time Markov Chain
- \cite{DBLP:conf/wetice/DjoudiBZ14}
- \cite{DBLP:conf/aosd/ZhangGC09}
- \cite{DBLP:conf/icse/GhezziPST13}
	- adaptation: finding a path in the FSM to achieve the goals
- \cite{DBLP:conf/kbse/TajalliGEM10}

\paragraph{Dynamic Software Product-Lines}
- \cite{DBLP:conf/dagstuhl/GhezziS10}


	


\subsection[Modelling and reasoning over long-term actions]{Modelling and reasoning over \glspl{longTermAct}}