\section[Results RQ1: long-term actions]{Results RQ1: \glspl{longTermAct}}
\label{sec:sota:results:actions}

% 1.1 %
\subsection[Modelling the evolution of system's context or behaviour]{Modelling the evolution of system's context or \gls{behaviour}}

%%% Context %%%

\paragraph{Modeling paradigm}
- \cite{DBLP:journals/computer/BlairBF09, DBLP:journals/computer/MorinBJFS09}
	- causality link between a model and the system
	- model should reflect the system, modification of the model triggers modification of the system
- \cite{DBLP:conf/seke/0001FNMKT14, DBLP:conf/models/0001FNMKBT14}
	- extend \gls{m@rt} to introduce time in the model

\paragraph{Formal model} %% Context %%
- \cite{DBLP:journals/taas/WeynsMA12}
	- reference model for AS
	- no time consideration

\paragraph{Low level model} %% Context %%
- \cite{DBLP:conf/dbpl/MoffittS17}
	- low level model: temporal graph
	- can be used by higher level model tool
	- define a temporal graph
	- add 2 functions:
		- one which indicate the existence of a node/edge for a time period
		- one to retrieve the value of a node or an edge at a time period
		
\paragraph{Object-based model} %% Context %%
- \cite{DBLP:conf/pervasive/HenricksenIR02}
	- define a model for pervasive computing systems\footnote{A pervasive system is composed of cheap and interconnected devices that are ubiquitous and can support users' tasks.\cite{DBLP:conf/pervasive/HenricksenIR02}}
	- model use object-based approach
	- define entity, and attribute, both link with uni-directional association
	- these associations can be dynamic, one category of dynamically is temporal one: the association ends with a set of values attached to a timestamp
- \cite{DBLP:conf/smartgridsec/0001FKNT14}
	- model the context and its history using following the \gls{m@rt} paradigm
	
\paragraph{Goal model} %% Context %%
- \cite{DBLP:conf/icse/CailliauL17}
	- use a probabilistic goal model, can be thought as a high-level model of the context
		- probability: describes the satisfactory rate of each goals
- \cite{DBLP:conf/icse/IftikharW14a}
	- use a goal model
- \cite{DBLP:conf/icse/MendoncaAR14}

\paragraph{State machine} %% Behaviour + context %%
- \cite{DBLP:conf/smartgridsec/0001FKNT14}
	- behaviour of the system model with a \gls{fsm}
- \cite{DBLP:conf/icse/IftikharW14a}
	- use of a \gls{fsm}
- \cite{DBLP:conf/icse/ArcainiRS15}
	- use of an \gls{asm}
- \cite{DBLP:conf/icse/BarbosaLMJ17}
	- probabilistic labelled transition system (probabilistic state machine) 
		- expected values set at design time, then compared with those get at runtime

\paragraph{Sequential diagram}
- \cite{DBLP:conf/icse/TaharaOH17}

\paragraph{Maude~\cite{DBLP:journals/tcs/ClavelDELMMQ02}x}
- \cite{DBLP:conf/icse/TaharaOH17}: use of the Maude language  to represent the context
	- no time aspect
	
	
	
	
	
	
	
	
	
% 1.2 %		
\subsection[Modelling actions, their circumstances, and their effects]{Modelling \glspl{action}, their \glspl{circumstance}, and their effects}

\paragraph{Rule-based approach}
- \cite{DBLP:conf/icse/TaharaOH17}: use the Maude language to define the adaptation mechanism
	- rewriting rules
	- rl [ (label)] : (source)=> (target)
		- label: name of the rule
		- source: init state of the system
		- target: final state of the system
	- source/target => circumstances/effects of action
	 - but no time dimension, side effects, no delayed effects, no action execution tracing
- \cite{DBLP:conf/icse/ArcainiRS15}
	- define a set of rules to change state of the system
	- circumstances = init state + conditions
	- effects = target states
		- but no time consideration, side effects, delayed effects
		
\paragraph{State machine}
- \cite{DBLP:conf/icse/ArcainiRS15} use of an \gls{asm}
	- use rules to define state transition
- \cite{DBLP:conf/icse/IftikharW14a}
		 
\paragraph{Publish/Subscribe approach}
- \cite{DBLP:conf/icse/BarbosaLMJ17}
	- adaptation mechanism is triggered following the publish/subscribe mechanism
	- no additional information
	- circumstances = conditions
	- no effects modeled, no time dimension, model cannot be navigable
	
\paragraph{Goal-modelling}
- \cite{DBLP:conf/icse/MendoncaAR14}
	- actions are part of the goal model, and are represented as mean to achieve goals
	- effects are thus represented as fulfilment of a goal
	- no circumstances modelled
	- high-level abstraction of actions
	- no time consideration
	
\paragraph{}
	 
	 
	 
	 
	 
	 
	 
	 
% 1.3 %
\subsection{Reasoning over evolving context or behaviour}

\paragraph{Modeling paradigm}
- \gls{m@rt} \cite{DBLP:journals/computer/BlairBF09, DBLP:journals/computer/MorinBJFS09}
	- extension to support temporal axis \cite{DBLP:conf/seke/0001FNMKT14, DBLP:conf/models/0001FNMKBT14}
- Lotus@Runtime~\cite{DBLP:conf/icse/BarbosaLMJ17}
	
\paragraph{FORMS}
- \cite{DBLP:conf/icse/IftikharW14a}
	- define a global formal model that represent abstract \gls{sadapt}, here a state machine
	
\paragraph{Rule-based adaptation}
- \cite{DBLP:conf/icse/ArcainiRS15}
	


\subsection[Modelling and reasoning over long-term actions]{Modelling and reasoning over \glspl{longTermAct}}