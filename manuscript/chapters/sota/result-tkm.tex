\section[Results RQ1: long-term actions]{Results RQ1: \glspl{longTermAct}}
\label{sec:sota:results:actions}

In this section, we detail our findings regarding the first research questions.
First, we detail all approaches that propose a solution to model the evolution of system's context and behaviour.
Then, we list approaches that model actions.
Before summarising and answering the research question, we list solutions that model and reason over evolving context and \gls{behaviour}, \ie that implement an adaptation process.

\subsection[Modelling the evolution of system's context or behaviour]{Modelling the evolution of system's context or \gls{behaviour}}

Different categories of approaches exist to represent the context or the \gls{behaviour} of a system.
In this section, we details our findings, with an overview given in \Cref{table:sota:results:actions:rq1.1}.

\paragraph{Modeling paradigm}
In the \gls{mde} community, researchers have defined the \gls{m@rt} paradigm to implement an adaptation process~\cite{DBLP:journals/computer/BlairBF09, DBLP:journals/computer/MorinBJFS09}.
This approach is based on a runtime model that reflects the current state of the system.
It can contain information about either the context of the system, its \gls{behaviour}, or both.
Moreover, there is a causal link between the model and the system: modifications of the model, made by a stakeholder or a process, trigger modification of the system.
For example, changing the status of a fuse in a model that reflects a smart grid triggers the action to open or close it.
Hartmann \etal extended this paradigm to introduce a temporal dimension~\cite{DBLP:conf/seke/0001FNMKT14, DBLP:conf/models/0001FNMKBT14}.
This allow to not only store the current state of the system, but also its past (previous states) and future (predicted context).
In this thesis, we will use this extension of the paradigm to build our knowledge model, and more precisely to represent \glspl{longTermAct}.


\paragraph{Formal model}
In~\cite{DBLP:journals/taas/WeynsMA12}, Weyns \etal defined a formal model for \gls{adptSyst}, called FORMS.
Their goal was to define a reference model, which can be used for discussion or implementation.
For self-organisation systems, the literature provide another formal model: MACODO~\cite{DBLP:journals/taas/WeynsHH10}.
It uses the Z language~\cite{DBLP:books/daglib/0011651} to formalise the context of the system following the set theory and the first order predicate calculus.
The \gls{behaviour} is formalised with what they call \textit{laws}.
A law is a function from one set to another.
The third formal model found in our review has been specified by Bartels and Kleine~\cite{DBLP:conf/icse/BartelsK11}.
This one uses Communicating Sequential Process principles~\cite{DBLP:journals/cacm/Hoare78}, a formalism designed for reactive and concurrent systems.
However, none of these models include a time dimension, important to abstract \gls{longTermAct}.

\paragraph{Object-based model}
One category of approach found in the literature is object-based models.
These models follow object-oriented principles.
First, Henricksen \etal defined a model for pervasive computing systems\footnote{A pervasive system is composed of cheap and interconnected devices that are ubiquitous and can support users' tasks.\cite{DBLP:conf/pervasive/HenricksenIR02}}.
In their model, there are entities that are linked to their attributes through uni-directional association.
These associations can be dynamic, can evolve, or static, do not evolve.
A dynamic association can also be temporal.
In this case, the entity can have several attributes with a timestamp attached.
Second, Hartmann \etal use a temporal model to store the context and the history of a smart grid system~\cite{DBLP:conf/smartgridsec/0001FKNT14}.
This model is based on their extension of the \gls{m@rt} described above~\cite{DBLP:conf/seke/0001FNMKT14, DBLP:conf/models/0001FNMKBT14}.
Third, Tahara \etal \cite{DBLP:conf/icse/TaharaOH17} use the Maude language~\cite{DBLP:journals/tcs/ClavelDELMMQ02}, to represent the context. 
Among these three solutions, only the last one do not include a temporal dimension.

\paragraph{Goal model}
Goal modelling is a technique used by several contributions in our findings~\cite{DBLP:conf/icse/CailliauL17, DBLP:conf/icse/IftikharW14a, DBLP:conf/icse/MendoncaAR14, DBLP:conf/icse/ChenPYNZ14, DBLP:conf/re/BaresiPS10}.
This technique, mainly used in requirement engineering, represent the different model the different goals of an application.
By modelling the requirements of a system's context, we can argue that they de facto represent it in a high level perspective.
The context is mainly represented by the goals that are achieved or not.
Moreover, in \cite{DBLP:conf/icse/CailliauL17}, authors add satisfactory rate information on each goal.
However, these methods do not include a time dimension.

\paragraph{State machine}
State machines have the capacity to model, in a same model, the context and the behaviour of a system.
The context is represented by the different states while the behaviour can be abstracted by the transitions between the states.
\cite{DBLP:conf/smartgridsec/0001FKNT14, DBLP:conf/icse/IftikharW14a, DBLP:conf/icse/ArcainiRS15, DBLP:conf/rv/ArcainiGR11, DBLP:conf/icse/GhezziPST13} use the \gls{fsm} formalism.
For example, Hartmann \etal abstracts the behaviour of a smart meter in \cite{DBLP:conf/smartgridsec/0001FKNT14}.
In~\cite{DBLP:conf/icse/GhezziPST13}, the authors represent the functionalities of the system and their impact with states.
Transitions abstract the different execution flow between the different functionalities.
In~\cite{DBLP:conf/icse/BarbosaLMJ17, DBLP:journals/computing/BencomoBGBI13}, authors apply the labelled transition system~\cite{DBLP:journals/cacm/Keller76}.
When authors want to consider the the stochastic behaviour of the system, then they use probabilistic state machines
In \cite{DBLP:conf/icse/BarbosaLMJ17}, the authors extended  the model with probabilities, which represent the probability for a transition of a transition to be executed.
Another strategy is to use Markov Chain~\cite{DBLP:conf/sigsoft/MorenoCGS15, DBLP:conf/kbse/FilieriGLM11, DBLP:conf/dagstuhl/GhezziS10, DBLP:conf/icse/DuarteMS18}
A Markov chain can be thought as a \gls{fsm} with probabilities attached to the transition.
In~\cite{DBLP:conf/sigsoft/MorenoCGS15}, authors also add information regarding current actions being executed with their progress status.
But, no history is kept, when the action is finished, the information is lost.
Other approaches use state machine without specifying the formalism used~\cite{DBLP:conf/wetice/DjoudiBZ14, DBLP:conf/aosd/ZhangGC09, DBLP:conf/kbse/TajalliGEM10}.
Tajalli \etal use two state machines: one to represent the system context and behaviour, and another one to represent the adaptation mechanism.

\paragraph{Sequential diagram}
Through our review, we find one approach that uses sequential diagram to represent the behaviour of the the system~\cite{DBLP:conf/icse/TaharaOH17}.
However, nothing is mentioned regarding the context of the system and its history.
	
\paragraph{Component model}
In order to represent the context of a system, one can use a component models.
This model is at the architecture level and described the different entity (component) that composes a system with their interaction.
Four contributions apply this technique in our review~\cite{DBLP:conf/soco/DavidL06, DBLP:conf/wetice/DjoudiBZ14, DBLP:journals/computer/GarlanCHSS04, DBLP:conf/cbse/FouquetMFBPJ12}.
In order to also represent the behaviour, some have extended this model with a sate machine description in~\cite{DBLP:conf/wetice/DjoudiBZ14} or with annotation in~\cite{DBLP:journals/computer/GarlanCHSS04}.
However, no time dimension is considered in these approaches.

\paragraph{Trace models}
Context and behaviour of a system can be inferred by analysing its logs.
In~\cite{DBLP:journals/computer/Maoz09}, researchers defined an approach to create a model that reflects the runtime state of the system.
However, this approach do not keep the history of the system.
	
\paragraph{Graph model}
Finally, the last technique used to represent the context of a system is to use a graph model~\cite{DBLP:journals/tse/KramerM90, DBLP:journals/computer/GeorgasHT09, DBLP:conf/dbpl/MoffittS17}.
In the former, node represents the process unit and the edges the communication between them.
In the second one, node represent the possible configuration of the system and the edges represent the actions to reach a configuration.
Only the latter include a time dimension.
The graph represents the different configuration over time of the system.
Plus, some meta-data about previous adaptations (\eg rate of configuration, average time in this configuration) are added.
The latter defined a temporal graph~\cite{DBLP:conf/dbpl/MoffittS17}.
Their temporal graph is a graph that is augmented with two functions.
One function serves to indicate if a graph element, a node or an edge, exists for a given time period.
The other can retrieve the value of a graph element for a given time period.
Using this temporal graph, one can define a model that abstract \glspl{longTermAct}.
However, in our work, we use another temporal graph definition provided by Hartmann \etal and implemented in our research group~\cite{DBLP:journals/is/HartmannFMRT19}.

\paragraph{Sum-up}
Different approaches are used in the literature to represent the context, the \gls{behaviour}, or both of systems (\cf \Cref{table:sota:results:actions:rq1.1}).
However, only a few can be used to keep the history of these informations~\cite{DBLP:conf/seke/0001FNMKT14, DBLP:conf/models/0001FNMKBT14, 	DBLP:conf/dbpl/MoffittS17, DBLP:conf/icse/TaharaOH17, DBLP:conf/pervasive/HenricksenIR02, DBLP:conf/smartgridsec/0001FKNT14}.
This feature remains a key concern to represent \glspl{longTermAct} as information about delayed effects and previous \glspl{circumstance} (next and previous context of an action).
In the next section, we detail approaches that model actions.
	
\begin{table}
	\begin{center}
    	\begin{tabular}{p{0.25\textwidth}p{0.69\textwidth}}
    		\hline
    		\textbf{Approach} & \textbf{Reference} \\
    		\hline
    		Modeling paradigm & \cite{DBLP:journals/computer/BlairBF09, DBLP:journals/computer/MorinBJFS09, DBLP:conf/seke/0001FNMKT14, DBLP:conf/models/0001FNMKBT14} \\
    		Formal model & \cite{DBLP:journals/taas/WeynsMA12, DBLP:journals/taas/WeynsHH10, DBLP:conf/icse/BartelsK11} \\
    		Low level model & \cite{DBLP:conf/dbpl/MoffittS17} \\
    		Object-based model & \cite{DBLP:conf/pervasive/HenricksenIR02, DBLP:conf/smartgridsec/0001FKNT14, DBLP:conf/icse/TaharaOH17}\\
    		Goal model & \cite{DBLP:conf/icse/CailliauL17, DBLP:conf/icse/IftikharW14a, DBLP:conf/icse/MendoncaAR14, DBLP:conf/icse/ChenPYNZ14, DBLP:conf/re/BaresiPS10} \\
    		State machine & \cite{DBLP:conf/smartgridsec/0001FKNT14, DBLP:conf/icse/IftikharW14a, DBLP:conf/icse/ArcainiRS15, DBLP:conf/rv/ArcainiGR11, DBLP:conf/icse/BarbosaLMJ17, DBLP:journals/computing/BencomoBGBI13, DBLP:conf/sigsoft/MorenoCGS15, DBLP:conf/kbse/FilieriGLM11, DBLP:conf/dagstuhl/GhezziS10, DBLP:conf/icse/DuarteMS18, DBLP:conf/wetice/DjoudiBZ14, DBLP:conf/aosd/ZhangGC09, DBLP:conf/icse/GhezziPST13, DBLP:conf/kbse/TajalliGEM10} \\
    		Sequential diagram & \cite{DBLP:conf/icse/TaharaOH17} \\
    		Component model & \cite{DBLP:conf/soco/DavidL06, DBLP:conf/wetice/DjoudiBZ14, DBLP:journals/computer/GarlanCHSS04, DBLP:conf/cbse/FouquetMFBPJ12} \\
    		Trace models & \cite{DBLP:journals/computer/Maoz09} \\
    		Graph model & \cite{DBLP:journals/tse/KramerM90, DBLP:journals/computer/GeorgasHT09} \\
    		\hline
    	\end{tabular}
    	\caption{Approaches to model systems' context and \gls{behaviour} (RQ1.1)}
    	\label{table:sota:results:actions:rq1.1}
    \end{center}
\end{table}

	
\subsection[Modelling actions, their circumstances, and their effects]{Modelling \glspl{action}, their \glspl{circumstance}, and their effects}

In~\Cref{table:sota:results:actions:rq1.2}, we regroup the different approaches of our review that model \glspl{action}.
In this section, we describe the different categories that we identified.

\paragraph{Rule-based approach}
One solution to model \glspl{action} is to use a rule en-\linebreak gine~\cite{DBLP:conf/icse/TaharaOH17, DBLP:conf/icse/ArcainiRS15, DBLP:conf/wrla/BruniCGLV12, DBLP:conf/eurosys/GraceHPBCT08, DBLP:conf/gpce/PintoFT03, DBLP:journals/computer/GarlanCHSS04}.
Rules are characterised by a condition and an executable code.
The executable code is executed if the current state of the system meets the condition.
Conditions can thus serve to abstract the \Glspl{circumstance} of an action and the executable code as its effect.
However, these information are available at design time and lost during the execution.
Plus, these approach do not allow the representation of the side effects of an action.
For example, changing the fuse state has a direct effect on the fuse state.
But it also impacts the power grid load.
We can notice two exceptions in our review: \cite{DBLP:conf/icse/TaharaOH17} and \cite{DBLP:conf/icse/ArcainiRS15}.
In both cases, rules are used to trigger a state modification in a sate machine.
We explain the advantages and disadvantages of the state machine approach in the next paragraph. 



\paragraph{State machine}
Several approaches use a state machine to represent the adaptation mechanism~\cite{DBLP:conf/icse/ArcainiRS15, DBLP:conf/icse/IftikharW14a, DBLP:conf/smartgridsec/0001FKNT14, DBLP:conf/sigsoft/MorenoCGS15, DBLP:conf/kbse/FilieriGLM11, DBLP:conf/wetice/DjoudiBZ14, DBLP:conf/aosd/ZhangGC09, DBLP:conf/icse/GhezziPST13, DBLP:conf/kbse/TajalliGEM10}.
States represent the state of the system and the transition the execution of \glspl{action}. 
One advantage of this approach is that they represent both the \glspl{circumstance} and the effects of actions.
Additionally, it can be used to represent actions at design time and runtime.
But, this link remains at high level.
The all state is considered at the circumstance or the effect of the action while, in most cases, it just a subset of the element of the state that triggers the action or are affected by it. 

\paragraph{Publish/Subscribe approach}
In our review, we found one approach that uses the publish/subscribe mechanism to trigger \glspl{action}~\cite{DBLP:conf/icse/BarbosaLMJ17}.
The \glspl{circumstance} are thus modelled with the condition of the consumer.
In this case, the action is a script.
The effects are thus spread, and cannot be navigated.
Moreover, this solution only represent the actions at design time, and not their executions.
	
\paragraph{Goal-modelling}
In addition to the goals of the system, goal models offer the capacity to represent the \glspl{action} that can achieve them.
Three approaches have used this ability to represent action in their model~\cite{DBLP:conf/icse/MendoncaAR14, DBLP:conf/iceccs/BencomoWSW12, DBLP:conf/re/BaresiPS10}.
Baresi \etal extended goal model with \textit{adaptive goals}.
This goals contain a condition, objectives (to weaken or enforce some goals), and \textit{actions} (here, an action modify goals or operation-executable code- in the goal model).
Here, effects can be seen as the fulfilment of a goal.
And a condition is when a goal is not satisfied anymore.
However, these information are not kept over time.
Furthermore, no information about the runtime execution of the actions is kept.
	
\paragraph{Programming language}
Among our findings, one approach defined its own language to define actions: \cite{DBLP:journals/jss/ChengG12}.
This language allow developers modelling their actions, with their conditions, and their effects.
However, the language does not include a temporal dimension.
Plus, the language is suitable to describe actions at design time but provide no mechanism to track them during their execution.
	
\paragraph{Event-Condition Action}
In order to trigger adaptation, one can use the Event-Condition Action approach: the action is triggered if an event respect the condition.
In our review, all the works use this methodology in order to weave or remove aspects of the program, following Aspect-Oriented Programming\footnote{Aspect-Oriented Programming is a programming paradigm that prones the separation of concern. For that, a program is seen as a set of aspects, each aspect implementing one concern.}~\cite{DBLP:conf/icws/CharfiDM09, DBLP:journals/scp/ParraBCD11, DBLP:conf/soco/DavidL06}.
However, there is no temporal dimension.
Plus, this solution is suitable to describe actions at design time, but does not allow navigation through runtime information.

	
\paragraph{Model transformation}
Following the \gls{m@rt}, one way to adapt a system is to modify the model to trigger the actions.
One way to do this, is to use the Query/View/Transformation~\cite{QVT:Spec} approach as Chen \etal did in \cite{DBLP:conf/icse/ChenPYNZ14}.
Here, the \gls{circumstance} and the effects are represented at the model level in the query and the transformation part.
When the context is represented as a graph, \glspl{action} will thus be modelled as graph modifications.
In~\cite{DBLP:journals/tse/KramerM90}, authors represent an action by adding or removing graph elements (node and edge).
However, these solution do not take into account any time dimension, side-effects or runtime information.
	
\paragraph{Formal model}
- \cite{DBLP:journals/taas/WeynsHH10}
	- function from set to set
	- represent adding,removing merging, splitting organisations (group of agents/components)
- \cite{DBLP:conf/icse/BartelsK11}
	- speical stuff......
- \cite{DBLP:conf/aaai/CimattiMR15} %% previous temporal planning %%
	- formalise the planning following temporal planning principle
	- define planning algorithm for a set of actions (not precisely define)
	- planning consider the time of the execution of the action, that might not be controllable (uncertain)
	- nothing about the effects of their circumstances
	
	
\begin{table}
	\begin{center}
    	\begin{tabular}{p{0.25\textwidth}p{0.69\textwidth}}
    		\hline
    		\textbf{Approach} & \textbf{Reference} \\
    		\hline
    		Rule-based & \cite{DBLP:conf/icse/TaharaOH17, DBLP:conf/icse/ArcainiRS15, DBLP:conf/wrla/BruniCGLV12, DBLP:conf/eurosys/GraceHPBCT08, DBLP:conf/gpce/PintoFT03, DBLP:journals/computer/GarlanCHSS04} \\
    		Sate machine & \cite{DBLP:conf/icse/ArcainiRS15, DBLP:conf/icse/IftikharW14a, DBLP:conf/smartgridsec/0001FKNT14, DBLP:conf/sigsoft/MorenoCGS15, DBLP:conf/kbse/FilieriGLM11, DBLP:conf/wetice/DjoudiBZ14, DBLP:conf/aosd/ZhangGC09, DBLP:conf/icse/GhezziPST13, DBLP:conf/kbse/TajalliGEM10}\\
    		Goal-modelling & \cite{DBLP:conf/icse/MendoncaAR14, DBLP:conf/iceccs/BencomoWSW12, DBLP:conf/re/BaresiPS10}\\
    		Programming language & \cite{DBLP:journals/jss/ChengG12} \\
    		Event-Condition Action & \cite{DBLP:conf/soco/DavidL06, DBLP:conf/icws/CharfiDM09, DBLP:journals/scp/ParraBCD11} \\
    		Model transformation & \cite{DBLP:conf/icse/ChenPYNZ14, DBLP:journals/tse/KramerM90} \\
    		Formal model & \cite{DBLP:journals/taas/WeynsHH10, DBLP:conf/icse/BartelsK11, DBLP:conf/aaai/CimattiMR15} \\ 
    		\hline
    	\end{tabular}
    	\caption{Approaches to model \glspl{action}, their \glspl{circumstance}, and their effects (RQ1.2)}
    	\label{table:sota:results:actions:rq1.2}
    \end{center}
\end{table}

	
\paragraph{Dynamic Software Product-Lines}
- \cite{DBLP:conf/dagstuhl/GhezziS10}
	- model variation point in a SPL model
- \cite{DBLP:series/lncs/CordyCHLS13}
	- use of Adaptive Featured Transition Systems
	
\paragraph{Graph model}
- \cite{DBLP:journals/computer/GeorgasHT09}
	- action = represented by edges in a graph where nodes represent possible configuration
	 
	 
\paragraph{Sum up}	 

	 
% 1.3 %
\subsection{Reasoning over evolving context or behaviour}

\paragraph{Model-based approach}
- \gls{m@rt} \cite{DBLP:journals/computer/BlairBF09, DBLP:journals/computer/MorinBJFS09}
	- extension to support temporal axis \cite{DBLP:conf/seke/0001FNMKT14, DBLP:conf/models/0001FNMKBT14}
- Lotus@Runtime~\cite{DBLP:conf/icse/BarbosaLMJ17}
- \cite{DBLP:conf/icse/ChenPYNZ14}
		
\paragraph{Rule-based adaptation}
- \cite{DBLP:conf/icse/ArcainiRS15, DBLP:conf/icse/TaharaOH17, DBLP:conf/eurosys/GraceHPBCT08}

\paragraph{Architecture-based adaptation}
- \cite{DBLP:journals/jss/ChengG12} defines a language for architecture-based adaptation
- \cite{DBLP:journals/computer/GarlanCHSS04}
- \cite{DBLP:journals/computer/GeorgasHT09}
- \cite{DBLP:conf/cbse/FouquetMFBPJ12}
	- architecture modification / adaptation of component configuration
	- adaptation executed by scripts, computed by applying a diff between current and expected component model

\paragraph{Simulation-based adaptation}
- \cite{DBLP:conf/smartgridsec/0001FKNT14}
	- simulate different sequences of actions, and evaluate them to evaluate optimal one

\paragraph{Formal model}
- \cite{DBLP:journals/taas/WeynsMA12}: Forms
- \cite{DBLP:conf/icse/IftikharW14a}, extension of FORMS
	- define a global formal model that represent abstract \gls{sadapt}, here a state machine
- \cite{DBLP:journals/taas/WeynsHH10}
	- formal model for self-organisation
- \cite{DBLP:conf/icse/BartelsK11}
	- speical stuff......

\paragraph{Complex Event Processing}
- \cite{DBLP:conf/rr/AnicicFRSSS10}

\paragraph{Graph model}
- \cite{DBLP:conf/dbpl/MoffittS17}: temporal graph algebra
	- can be used as low-level "basis" to define a temporal context and behaviour
- \cite{DBLP:journals/tse/KramerM90}
	- representation of the context as a graph, and reason over it
	
\paragraph{Aspect oriented programming}
- \cite{DBLP:journals/taosd/GreenwoodB06, DBLP:conf/soco/DavidL06, DBLP:conf/icws/CharfiDM09, DBLP:journals/scp/ParraBCD11, DBLP:conf/ewsa/FalcarinA04, DBLP:conf/gpce/PintoFT03}: dynamic Aspect Oriented Programing
	- adaptation by dynamically weaving/removing aspects in the program
	- no time aspect, no effects
	- circumstances == conditions to remove/weave an aspect
- \cite{DBLP:conf/icse/MorinBNJ09}
	- Aspect-oriented modelling
	 - family of MDE
	- model aspects and their point cut

\paragraph{Component-based adaptation}
- \cite{DBLP:conf/soco/DavidL06}: adaptation by modifying component of the system

\paragraph{State machine}
- \cite{DBLP:conf/sigsoft/MorenoCGS15}
	- Markov Decision Process
	- actions (strategies) are triggered by a condition
- \cite{DBLP:conf/kbse/FilieriGLM11}
	- Discrete Time Markov Chain
- \cite{DBLP:conf/wetice/DjoudiBZ14}
- \cite{DBLP:conf/aosd/ZhangGC09}
- \cite{DBLP:conf/icse/GhezziPST13}
	- adaptation: finding a path in the FSM to achieve the goals
- \cite{DBLP:conf/kbse/TajalliGEM10}

\paragraph{Dynamic Software Product-Lines}
- \cite{DBLP:conf/dagstuhl/GhezziS10, DBLP:series/lncs/CordyCHLS13}

\paragraph{Requirement-driven adaptation}
- goal model: \cite{DBLP:conf/re/BaresiPS10}

\paragraph{Extension of \gls{mapek} loop}
- \cite{DBLP:conf/iscc/MaurerBEB11}

	


\subsection[Modelling and reasoning over long-term actions]{Modelling and reasoning over \glspl{longTermAct}}