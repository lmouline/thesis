\section[Results RQ2: data uncertainty]{Results RQ2: \gls{duc}}
\label{sec:sota:results:duc}


%%% 2.1 %%%
\subsection{Categories of data uncertainty}

\paragraph{Data uncertainty}
- \cite{DBLP:conf/models/BurguenoBMV18, baudin2017openturns, DBLP:journals/corr/BorgstromGGMG13, DBLP:conf/ecmdafa/BertoaMBBTV18, DBLP:conf/asplos/BornholtMM14, osti_1430202, DBLP:conf/sle/MayerhoferWV16, DBLP:journals/peerj-cs/SalvatierWF16, DBLP:conf/quatic/VallecilloMO16, DBLP:journals/sosym/Zhang00NO19, DBLP:journals/csi/Hall06, DBLP:journals/infsof/Jimenez-RamirezW0V15, DBLP:conf/ecmdafa/ZhangSAYON16, DBLP:journals/tkde/BarbaraGP92, DBLP:conf/vldb/BenjellounSHW06, DBLP:conf/popl/BhatAVG12}

\paragraph{Model transformation}
- \cite{DBLP:conf/models/BurguenoBMV18, DBLP:conf/sle/EramoPR15, DBLP:conf/icse/EramoPR14}
	- express uncertainty in transformation rules
	
\paragraph{Design uncertainty}
- \cite{DBLP:conf/icse/FamelisSC12, DBLP:journals/sosym/FamelisC19, DBLP:conf/sle/EramoPR15, DBLP:conf/icse/EramoPR14, DBLP:journals/re/SalayCHS13, DBLP:conf/ecmdafa/ZhangSAYON16}

\paragraph{Time uncertainty}
- \cite{DBLP:conf/icst/Garousi08}
	- uncertainty about when an event is expected to occur
	
\paragraph{Requirement uncertainty}
- \cite{DBLP:journals/re/WhittleSBCB10, DBLP:conf/re/WhittleSBCB09, DBLP:journals/re/SalayCHS13}
	- uncertainty in the requirements of a software

\paragraph{Uncertainty in \gls{behaviour}}
- \cite{DBLP:journals/sosym/Zhang00NO19}
	- behaviour of a system in a production environment remain uncertain as the environment itself is uncertain
	
\paragraph{Uncertainty in \gls{env}}
- \cite{DBLP:conf/dagstuhl/EsfahaniM10}
	- SAS
- \cite{DBLP:conf/ecmdafa/ZhangSAYON16}
	
\paragraph{Uncertain process}
- \cite{DBLP:journals/infsof/Jimenez-RamirezW0V15}
	- uncertainty in Business Process
	
\paragraph{Uncertain hardware}
- \cite{DBLP:conf/oopsla/CarbinMR13}
	- reliability of a computation made on an uncertain hardware
	
%%% 2.2 %%%
\subsection[Modelling data uncertainty]{Modelling \gls{duc}}

\paragraph{Data type with uncertain field}
- \cite{DBLP:conf/models/BurguenoBMV18, DBLP:conf/ecmdafa/BertoaMBBTV18, DBLP:conf/sle/MayerhoferWV16, DBLP:conf/quatic/VallecilloMO16}
	- define a new type: UReal
	- Type is composed of 2 elements: the value + a standard uncertainty
	- allow to represent uncertainty that follow a normal distribution
- \cite{DBLP:journals/tkde/BarbaraGP92}
	- attached a probability to a value in databases

	
\paragraph{Probabilistic program}
- \cite{baudin2017openturns, DBLP:conf/asplos/BornholtMM14, DBLP:journals/corr/BorgstromGGMG13, osti_1430202, DBLP:journals/peerj-cs/SalvatierWF16, DBLP:conf/popl/BhatAVG12}
	- define new types to encapsulate probability distributions

\paragraph{Multiple possibilities}
- \cite{DBLP:conf/icse/FamelisSC12}
	- define the concept of partial model: graph extended with annotation on graph elements that are true, maybe, false
		- true/false: think should be present or not in the model
		- maybe: it may be present + formula to describe this "maybe"
- \cite{DBLP:journals/sosym/FamelisC19}
	- alternatives in models
- \cite{DBLP:conf/sle/EramoPR15, DBLP:conf/icse/EramoPR14}
	- model is a partial model
	- transformation generates partial model
- \cite{DBLP:journals/re/SalayCHS13}
	- use partial model to reflect requirement uncertainty
- \cite{DBLP:conf/vldb/BenjellounSHW06}	
	- in databases
	- different possibilities for a same data
	
\paragraph{Randomness}
- \cite{DBLP:conf/icst/Garousi08}
	- introduce randomness in the time of occurrence during testing phase
	
\paragraph{Language}
- \cite{DBLP:journals/re/WhittleSBCB10, DBLP:conf/re/WhittleSBCB09}
	- add fuzzy words in a language to define requirements 
- \cite{DBLP:journals/infsof/Jimenez-RamirezW0V15}
	- declarative description of business process
	- properties can be probability distribution
- \cite{DBLP:conf/oopsla/CarbinMR13}
	- defines a language that can specify constraints on the reliability before executing a piece of code (function)
	
\paragraph{Model-level uncertainty}
- \cite{DBLP:journals/sosym/Zhang00NO19}
	- uses a UML profile which allow to define different kind of uncertainties
	- U-Model~\cite{DBLP:conf/ecmdafa/ZhangSAYON16}
	- use to test CPS to generate test cases based on uncertainty
	
\paragraph{Formal model}
- \cite{DBLP:journals/csi/Hall06}
	- implements the \gls{gum}~\cite{metrology2008evaluation}
- \cite{DBLP:conf/ecmdafa/ZhangSAYON16}
	- defines a conceptual model of uncertainty which regroup different kind of uncertainty in a \gls{cps}
	- UML model
%%% 2.3 %%%
\subsection{Propagation and reasoning over uncertainty}

% Propagation
\paragraph{Attach to language operators}
- \cite{DBLP:conf/models/BurguenoBMV18, baudin2017openturns, DBLP:journals/corr/BorgstromGGMG13, DBLP:conf/ecmdafa/BertoaMBBTV18, osti_1430202, DBLP:conf/sle/MayerhoferWV16, DBLP:journals/peerj-cs/SalvatierWF16, DBLP:conf/quatic/VallecilloMO16, DBLP:conf/popl/BhatAVG12}

\paragraph{Require manual propagation}
- \cite{DBLP:conf/models/BurguenoBMV18}
	- manual propagation from data uncertainty to model transformation uncertainty 
	
% Reasoning
\paragraph{Access to the confidence parameter}
- \cite{DBLP:conf/models/BurguenoBMV18, DBLP:conf/sle/MayerhoferWV16, DBLP:conf/quatic/VallecilloMO16}
	- confidence is considered as an attribute, accessible if pointed syntax
	
\paragraph{Access to probability features}
- \cite{baudin2017openturns, DBLP:journals/corr/BorgstromGGMG13, DBLP:conf/ecmdafa/BertoaMBBTV18, DBLP:conf/asplos/BornholtMM14, osti_1430202, DBLP:journals/peerj-cs/SalvatierWF16}
	- allow to get the different element of a probability distribution such as the mean, variance, ...




%%% 2 %%%
\subsection[Modelling of data uncertainty and its manipulation]{Modelling of \gls{duc} and its manipulation}