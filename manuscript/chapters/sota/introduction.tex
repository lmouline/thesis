\section{Introduction}

%% Research questions
\paragraph{Delayed actions}
General research question:
\begin{center}
	\textbf{RQ1}: Do current state of the art solutions that model \gls{adptSyst} allow representing and reasoning over \glspl{longTermAct}?  
\end{center}

Sub-research questions:
\begin{itemize}
	\item \textbf{RQ1.1}: How current approaches model the evolution of the context and/or the evolution of the behaviours of systems over time?
	\item \textbf{RQ1.2}: Do these solutions model actions, their circumstances and their effects over time? Can this model be processed or navigated automatically?
	\item \textbf{RQ1.3}: What are the solutions that enable the reasoning over the evolving context and/or behaviours of systems?
\end{itemize}


\paragraph{Data uncertainty}
General research question:
\begin{center}
	\textbf{RQ2}: Do current state of the art solutions allow modelling uncertainty of data and its manipulation (propagation, reasoning over)? 
\end{center}

Sub-research questions:
\begin{itemize}
	\item \textbf{RQ2.1}: What are the categories of uncertainties that have been addressed by the literature, and how?
	\item \textbf{RQ2.1}: How the uncertainty of data is modelled?
	\item \textbf{RQ2.3}: What are the solutions that enable an imperceptible propagation and a reasoning over the uncertainty?
\end{itemize}

%% Methodology
\paragraph{Methodology}
Snowballing approach~\cite{DBLP:conf/ease/Wohlin14}

\paragraph{Inclusion criteria}
\begin{itemize}
	\item \textbf{IC1}: The paper has been published before the May 31 2019
	\item \textbf{IC2}: The paper is available online and written in English
	\item \textbf{IC3}: The paper describes a modelling approach that abstract the context or behaviour of a system, an approach that enables to reason or navigate through a temporal model, or an approach that handles data uncertainty.
\end{itemize}

\paragraph{Exclusion criteria}
\begin{itemize}
	\item \textbf{EC1}: The paper has at most 4 pages (short paper).
	\item \textbf{EC2}: The paper presents a work in progress (workshop papers), a poster, a vision, a position, an exemplar, a data set, a tutorial or the paper is a Bachelor, Master or PhD dissertation.
	\item \textbf{EC3}: The paper describes a secondary study (\eg literature reviews, lessons learned).
	\item \textbf{EC4}: The document has not been published in a venue with a peer-review process. For example, technical and research report or white papers.
	\item \textbf{EC5}: The document is an introduction to the proceedings of a venue or a special issue or it is a guest paper.
\end{itemize}

However, the references of papers rejected are considered for the snowballing iteration.