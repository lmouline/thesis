\section{Introduction}

In this thesis, we focus on the capacity of \glspl{adptSyst} to adjust their \gls{behaviour} or \gls{structure} in response to changes in their \gls{env} or \gls{structure} (\cf \Cref{chapt:intro}).
Following the \gls{m@rt} paradigm, these systems can use a modelling layer that abstracts all information needed to enable reasoning processes (\cf \Cref{chapt:background}).
This process leads to the execution of a set of adaptation \glspl{action}, among which we identified \gls{longTermAct} (\cf \Cref{chapt:intro}).

However, we have identified three open challenges due to \gls{longTermAct}, the uncertainty of data received, and the emergence of the \gls{behaviour} of these systems (\cf \Cref{sec:intro:scope}).
In the past years, researchers heavily studied \glspl{adptSyst} and uncertainty.
We thus need to identify any potential work that has already tackled, partially or totally, these challenges.
Besides, we have to review approaches that model \glspl{adptSyst}, \gls{structure}, \gls{behaviour}, \gls{env}, or manipulate uncertain data.

We set two research questions to drive this review process:
\begin{itemize}
	\item \textbf{RQ1:} Do state-of-the-art solutions that model \gls{adptSyst} allow representing and reasoning over \glspl{longTermAct} (design time and runtime\footnote{Here we refer to the execution of an \gls{action}.})? 
	\item \textbf{RQ2:} Do state-of-the-art solutions allow modelling uncertainty of data and its manipulation (propagation, reasoning over)? 
\end{itemize} 

Through this review, we found that none of the state-of-the-art solutions model \glspl{action}, their effects, and their \glspl{circumstance} over time (RQ1), both at design time and at runtime.
Some approaches will partially model these elements, for example, \glspl{action} and their effects at design time.
However, these models cannot be automatically navigated by a process, like a reasoning engine.
Moreover, some approaches do not consider the temporal dimension of the effects.

Our findings show that different community have studied uncertainty in software.
For instance, modelling community has focused on design uncertainty and \gls{adptSyst} community has studied \gls{env} uncertainty.
Researchers have put a lot of efforts to encapsulate probability distributions behind programming language concepts such as variables.
But we strongly think that higher-level abstraction should be defined to help engineers.
Additionally, tools to help developers to manipulate uncertain data are needed.

The remainder of this chapter is structured as follows.
First, we describe the methodology used to perform this review in \Cref{sec:sota:methodo}.
In \Cref{sec:sota:results:actions} and \Cref{sec:sota:results:duc}, we present and discuss our findings.
Before concluding in \Cref{sec:sota:conclusion}, we detail some threats to the validity of this literature review.

