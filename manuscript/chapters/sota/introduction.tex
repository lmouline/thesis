\section{Introduction}

%% Link with previous chapters
- Thesis focus on the capacity of \gls{adptSyst}
- Need a modelling layer for reasoning (\gls{m@rt})

%% Motivation of the need of this chapters
- identification of several problematics and challenges in \gls{adptSyst}:
	- \gls{longTermAct}
	- \gls{duc}
- \gls{adptSyst} and uncertainty (general) have been heavily studied by researchers
- need to identify any potential studies that have already addressed these challenges

%% RQs
this state of the art aims at answering two research questions:
- RQ1: Do current state of the art solutions that model \gls{adptSyst} allow representing and reasoning over \glspl{longTermAct}? 
- RQ2: Do current state of the art solutions allow modelling uncertainty of data and its manipulation (propagation, reasoning over)? 


%% Sum-up findings
- RQ1: state of the art approaches do not model actions, their effects, and their circumstances over time, or they do not allow querying them
- RQ2:
	- modelling community has mainly focused on design uncertainty
	- adaptive systems community has focused on the uncertainty of system environment / context in which they evolve
	- probabilistic languages community focused on the manipulation of probability distributions and learning models
	

%% Structure of the chapter
- \Cref{sec:sota:methodo}: describes the methodology
- \Cref{sec:sota:results:actions}: presents the results to answer RQ1
- \Cref{sec:sota:results:duc}: presents the results to answer RQ2
- \Cref{sec:sota:validity}: presents threat to validity to this literature review
- \Cref{sec:sota:conclusion}: synthesises the findings and concludes the chapter