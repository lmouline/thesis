\section{Review methodology}
\label{sec:sota:methodo}

This review aim at answering two global research questions.
As they remain large, we split both in three sub-research questions to help us answering them.

\paragraph{Research questions}
The first research question has been set to study the presence of \glspl{longTermAct} in the modelling layer: (\textbf{RQ1}) do current state of the art solutions that model \gls{adptSyst} allow representing and reasoning over \glspl{longTermAct}?
We split it the following sub-questions:
\begin{itemize}
	\item \textbf{RQ1.1}: How current approaches model the evolution of the context and/or the evolution of the behaviours of systems over time?
	\item \textbf{RQ1.2}: What are the solutions that model actions, their circumstances and their effects over time? Can this model be processed or navigated automatically?
	\item \textbf{RQ1.3}: What are the solutions that enable the reasoning over the evolving context and/or behaviours of systems?
\end{itemize}
With RQ1.1, we investigated the approaches that model the context and the structure of \glspl{adptSyst}.
The second one filter those that also consider \glspl{longTermAct}.
Finally, we use RQ1.3 to list solutions that provide a technique, such as an algorithm, to reason over evolving context or behaviour.

With the second research question, we seek modelling solutions that consider data uncertainty and its propagation: \textbf{RQ2}: Do current state of the art solutions allow modelling uncertainty of data and its manipulation (propagation, reasoning over)?
The three sub-questions are:
\begin{itemize}
	\item \textbf{RQ2.1}: What are the categories of uncertainties that have been addressed by the literature, and how?
	\item \textbf{RQ2.2}: How the uncertainty of data is modelled?
	\item \textbf{RQ2.3}: What are the solutions that enable an imperceptible propagation and a reasoning over the uncertainty?
\end{itemize}
We set RQ2.1 to list approaches that tackle challenges brought by uncertainty in software engineering.
And we scope them to \gls{duc} with RQ2.2.
Lastly, we review approaches that allow developers to propagate uncertainty without writing special code for that and to reason over uncertainty.

%% Methodology
\paragraph{Methodology}
In order to review the literature, we applied a technique inspired from the snowballing approach~\cite{DBLP:conf/ease/Wohlin14}.
Other, due to limited resources, we did not fully applied it.
This methodology advocates the use of bibliography (backward navigation) and papers that cite (forward navigation) the selected ones to navigate in the literature.
Each paper should be evaluated according to a set of inclusion and exclusion criteria.
And a starting set should be defined.

In our case, we use the bibliography of the papers that ground in this thesis as starting set (\cf \Cref{sec:intro:contrib}).
Then, we apply the backward navigation for a subset of them.
Then, we select the paper to add in this review according to a set of inclusion and exclusion criteria.
To be picked, a paper should satisfy all inclusion criteria and should not fulfil any of the exclusion criteria.
These criteria are the following:
\begin{itemize}
	\item Inclusion criteria (IC):
	\begin{itemize}
		\item \textbf{IC1}: The paper has been published before the August 9 2019
		\item \textbf{IC2}: The paper is available online and written in English
		\item \textbf{IC3}: The paper describes a modelling approach that abstract the context or \gls{behaviour} of a system, an approach that enables to reason or navigate through a temporal model, an approach that handles uncertainty, or an approach that helps the manipulation of probability distributions. 
	\end{itemize}
	\item Exclusion criteria (EC):
	\begin{itemize}
		\item \textbf{EC1}: The paper has at most 4 pages (short paper).
		\item \textbf{EC2}: The paper presents a work in progress (workshop papers), a poster, a vision, a position, an exemplar, a data set, a tutorial, a project, or a Bachelor, Master or PhD dissertation.
		\item \textbf{EC3}: The paper describes a secondary study (\eg literature reviews, lessons learned).
		\item \textbf{EC4}: The document has not been published in a venue with a peer-review process. For example, technical and research report or white papers.
		\item \textbf{EC5}: The document is an introduction to the proceedings of a venue or a special issue or it is a guest paper.
	\end{itemize}
\end{itemize}

Two first inclusion criteria are accessibility criteria: they guarantee that the paper is accessible for any reader of this document.
With the third one, IC3, we can include all papers that can be used to answer our research questions.
We define the exclusion criteria in order to keep only paper that have been published in a peer-reviewed venue and that present an approach.

In this review, 412 papers have been processed and 84 kept for the review.
We report our selection results in an Excel file publicly available on GitHub~\footnote{\url{https://github.com/lmouline/thesis/tree/master/sota/src}}.
Results have also been exported in \gls{csv} files for those who cannot open Excel files.

