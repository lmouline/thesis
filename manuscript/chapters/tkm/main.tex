\chapter[TKM: a temporal knowledge model]{TKM: a temporal knowledge model to represent actions, their contexts and their impacts}
\chapterPage{
%% SAC
%Distributed adaptive systems are composed of federated entities offering remote inspection and reconfiguration abilities. 
%This is often realized using a MAPE-K loop, which constantly evaluates system and environmental parameters and derives corrective actions if necessary. 
%The OpenStack Watcher project uses such a loop to implement resource optimization services for multi-tenant clouds.
%To ensure a timely reaction in the event of failures, the MAPE-K loop is executed with a high frequency.
%A major drawback of such reactivity is that many actions, e.g., the migration of containers in the cloud, take more time to be effective and their effects to be measurable than the MAPE-k loop execution frequency. 
%Unfinished actions as well as their expected effects over time are not taken into consideration in MAPE-K loop processes, leading upcoming analysis phases potentially take sub-optimal actions. 
%In this paper, we propose an extended context representation for MAPE-K loop that integrates the history of planned actions as well as their expected effects over time into the context representations.
%This information can then be used during the upcoming analysis and planning phases to compare measured and expected context metrics. We demonstrate on a cloud elasticity manager case study that such temporal action-aware context leads to improved reasoners while still be highly scalable.
%% ICAC
%The evolving complexity of adaptive systems impairs our ability to deliver anomaly-free solutions. 
%Fixing these systems require a deep understanding on the reasons behind decisions which led to faulty or suboptimal system states. 
%Developers thus need diagnosis support that trace system states to the previous circumstances –targeted requirements, input context– that had resulted in these decisions. 
%However, the lack of efficient temporal representation limits the tracing ability of current approaches. 
%To tackle this problem, we first propose a knowledge formalism to define the concept of a decision. 
%Second, we describe a novel temporal data model to represent, store and query decisions as well as their relationship with the knowledge (context, require- ments, and actions). 
%We validate our approach through a use case based on the smart grid at Luxembourg. We also demonstrate its scalability both in terms of execution time and consumed memory.
%
%
% Context
% Challenge
% COntribution
 }
 
 \section{Introduction}

In this thesis, we focus on the capacity of \glspl{adptSyst} to adjust their \gls{behaviour} or \gls{structure} in response to changes in their \gls{env} or \gls{structure} (\cf \Cref{chapt:intro}).
Following the \gls{m@rt} paradigm, these systems can use a modelling layer that abstracts all information needed to enable reasoning processes (\cf \Cref{chapt:background}).
This process leads to the execution of a set of adaptation \glspl{action}, among which we identified \gls{longTermAct} (\cf \Cref{chapt:intro}).

However, we have identified three open challenges due to \gls{longTermAct}, the uncertainty of data received, and the emergence of the \gls{behaviour} of these systems (\cf \Cref{sec:intro:scope}).
\Glspl{adptSyst} and uncertainty have been heavily studied by researchers in the past years.
We thus need to identify any potential work that have already tackled, partially or totally, these challenges.
In addition, we have to review approaches that model \glspl{adptSyst}, \gls{structure}, \gls{behaviour}, \gls{env}, or manipulate uncertain data.

We set two research questions to drive this review process:
\begin{itemize}
	\item \textbf{RQ1:} Do current state-of-the art solutions that model \gls{adptSyst} allow representing and reasoning over \glspl{longTermAct} (design time and runtime\footnote{Here we refer to the execution of an \gls{action}.})? 
	\item \textbf{RQ2:} Do current state-of-the art solutions allow modelling uncertainty of data and its manipulation (propagation, reasoning over)? 
\end{itemize} 

Through this review, we found that none of the state-of-the art solutions model \glspl{action}, their effects, and their \glspl{circumstance} over time (RQ1), both at design time and at runtime.
Some approaches will partially model these elements, for example \glspl{action} and their effects at design time.
However, these models cannot be automatically navigated by a process, like a reasoning engine.
Or, some approaches do not consider the temporal dimension of the effects.

Our findings show that different community have studies uncertainty in software.
Modelling community has focused on design uncertainty, \gls{adptSyst} community has studied \gls{env} uncertainty.
Researchers have put a lot of efforts to encapsulate probability distributions behind programming language concepts such as variables.
But, we strongly think that higher-level abstraction should be defined, to help engineers.
Additionally, tools to help developers to manipulate uncertain data are needed.

The remainder of this chapter is structured as followed.
First, we describe the methodology used to perform this review in \Cref{sec:sota:methodo}.
In \Cref{sec:sota:results:actions} and \Cref{sec:sota:results:duc}, we present and discuss our findings.
Before concluding in \Cref{sec:sota:conclusion}, we detail some threats to validity of this literature review.


 \section{Knowledge formalization}
\label{sec:tkm:k-formalism}
 
  As discussed previously, I consider \gls{knowledge} to be the association of \gls{context} information, \glspl{requirement}, and \gls{action} information, all in one global and unified model.
 While \gls{context} information captures the state of the system environment and its surroundings, the system \glspl{requirement} define the constraints that the system should satisfy along the way. 
 The \glspl{action}, on the other hand, are means to reach the goals of the system.
  
 In this section, I provide a formalization of the \gls{knowledge} used by adaptation processes based on a temporal graph. 
Indeed, due to the complexity and interconnectivity of system entities, graph data representation seems to be an appropriate way to represent the \gls{knowledge}. 
Augmented with a temporal dimension, temporal graphs are then able to symbolize the evolution of system entities and states over time. 
We benefit from the well-defined graph manipulation operations, namely temporal graph pattern matching and temporal graph relations to represent the traceability links between the \glspl{decision} made and their \glspl{circumstance}.

Before describing this formalism, I describe the semantic used for the temporal axis.
Then, I exemplify the knowledge formalism using the Luxembourg smart grid use case.

\subsection{Formalization of the temporal axis}
\label{sec:tkm:timeDef}

\begin{figure}
   \centering
	\includegraphics[width=\textwidth]{img/chapt-tkm/formalism/formalismeTime}
	\caption{Time definition used for the knowledge formalism}
	\label{fig:tkm:formalismeTime}
\end{figure}

The formalism describe below has been made with two goals in mind.
First, the definition of the time space should allow the distinction between past and future. 
Doing this distinction enable the differentiation between measured data and estimated (or predicted data).
Second, it should permit the definition of the life cycle of an element of the \gls{knowledge}, which can be seen as a succession of states with a validity period that should not overlap each other.

Time space $T$ is considered as an ordered discrete set of time points non-uniformly distributed. 
As depicted in Figure~\ref{fig:tkm:formalismeTime}, this set can be divided into 3 different subsets $T = T_{past} \cup \{t\} \cup T_{future}$, where:  
\begin{itemize}
	\item $T_{past}$ is the sub-domain \{$t_{0}$;$t_{1}$;\ldots;$t_{current-1}$\}  representing graph data history starting from $t_0$, the oldest point, until current time, t, excluded.
	\item \{t\} is a singleton representing the current time 
point
	\item $T_{future}$ is sub-domain \{$t_{current+1};\ldots;t_{\infty}$\} representing future time points 
\end{itemize}
The three domains depend completely on the current time \{t\} as these subsets slide as time passes. 
At any point in time, these domains never overlap: $T_{past} \cap \{t\} = \emptyset$, $T_{future} \cap \{t\} =  \emptyset$, and $T_{past} \cap T_{future} = \emptyset$.
The definition of these three subsets reachs the first goal.

In addition, there is a right-opened time interval $I \in T \times T$ as $[t_s, t_e)$ where $t_e - t_s > 0$.
In English words, it means that the interval cannot represent a single time point and should follow the time order. 
For any $i \in I$, $start(i)$ denotes its lower bound and $end(i)$ its upper bound.
As detailed in Section~\ref{sec:tkm:formalism}, these intervals are used to define the validity period for each node of the graph. 

Figure~\ref{fig:tkm:formalismeTime} displays an example of a time space $T_1 = \{t_0, t_1, t_2, t_3, t_4, t_5, t_6, t_7\}$.
Here, the current time is $t = t_4$.
According to the definition of the past subset ($T_{past}$) and the future one ($T_{future}$), there is: $T_{past1} =  \{t_0, t_1, t_2, t_3\}$ and $T_{future1} = \{t_5, t_6, t_7\}$.
Two intervals have been defined on $T_1$, namely $I_1$ and $I_2$.
The first one starts at $t_2$ and ends at $t_5$ and the last one is defined from $t_6$ to $t_7$.
As shown with $I_1$, an interval could be defined on different subsets, here it is on all of them ($T_{past}$, $t$, and $T_{future}$).

\subsection{Formalism}
\label{sec:tkm:k-formalism:formalism}
 
\paragraph{Graph definition}
First, let $K$ be an adaptive process over a system \gls{knowledge} represented by a graph such as $K = (N, E)$, comprising a set of nodes $N$ and a set of edges $E$.
Nodes represent any element of the knowledge (context, actions, \etc) and edges represent their relationships.
Nodes have a set of attribute values.
An attribute value has a type (numerical, boolean, \ldots). 
Every relationship $e \in E$ can be considered as a couple of nodes $(n_s, n_t) \in N \times N$, where $n_s$ is the source node and $n_t$ is the target node.

\paragraph{Adding the temporal dimension}

\begin{figure}
   \centering
	\includegraphics{img/chapt-tkm/formalism/validityExample}
	\caption{Evolution of a temporal graph over time}
	\label{fig:tkm:validityEx}
\end{figure}

In order to augment the graph with a temporal dimension, the relation $V^T$ is added.
So now the knowledge $K$ is defined as a temporal graph such as $K = (N, E, V^T)$.

A node is considered valid either until it is removed or until one of its attributes value changes. 
In the latter case, a new node with the updated value is created.
Whilst, an edge is considered valid until either its source node and target node is valid, or until the edge itself is removed.
Otherwise, nodes and edges are considered invalid.
The temporal validity relation is defined as $V^T: N \cup E \rightarrow I$.
It takes as a parameter a node or an edge ($k \in N \cup E$) and returns a time interval ($i \in I$, \cf Section~\ref{sec:tkm:timeDef}) during which the graph element is valid.

Figure~\ref{fig:tkm:validityEx} shows an example of a temporal graph $K_1$ with five nodes ($n_1$, $n_2$,$n_3$, $n_4$, and $n_5$) and three edges ($e_1$, $e_2$, and  $e_3$) over a lifecycle from $t_1$ to $t_3$.
In this way, $K_1$ equals to $(\{n_1, n_2, n_3, n_4, n_5\}, \{e_1, e_2, e_3\}, V^{T}_1)$.
Let's assume that the graph is created at $t_1$.
As $n_1$ is modified at $t_2$, its validity period starts at $t_1$ and ends at $t_2$: $V^{T}_1(n_1) = [t_1, t_2)$.
$n_2$ and $n_3$ are not modified; their validity period thus starts at $t_1$ and ends at $t_\infty$: $V^{T}_1(n_2) = V^{T}_1(n_3) = [t_1, t_\infty)$.
Regarding the edges, the first one, $e_1$, is between $n_1$ and $n_2$ and the second one, $e_2$ from $n_2$ to $n_3$.
Both are created at $t_1$.
As $n_1$ is being modified at $t_2$, its validity period goes from $t_1$ to $t_2$:  $V^{T}_1(e_1) = [t_1, t_2)$.
$e_2$ is deleted at $t_3$.
Its validity period is thus equal to: $V^{T}_1(e_2) = [t_1, t_3)$.

\paragraph{Lifecycle of a knowledge element}
One node represents the state of exactly one knowledge element during a period named the validity period.
The lifecycle of a knowledge element is thus modeled by a unique set of nodes.
By definition, the validity periods of the different nodes cannot overlap.
A same time period cannot be represented by two different nodes, which could create inconsistency in the temporal graph.

To keep track of this knowledge element history, the $Z^T$ relation is added to the graph formalism: $K = (N, E, V^T, Z^T)$.
It serves to trace the updates of a given knowledge element at any point in time. 
This relation can also be seen as a temporal identity function which takes as parameters a given node $n \in N$ and a specific time point $t \in T$, and returns the corresponding node at that point. 
Formally, $Z^T: N \times T \rightarrow N$. 

In order to consider this new relation in the example presented in Figure~\ref{fig:tkm:validityEx}, the definition of $K_1$ is modified to $K_1 = (\{n_1, n_2, n_3, n_4, n_5\}, \{e_1, e_2, e_3\}, V^{T}_1, Z^{T}_1)$
In Figure~\ref{fig:tkm:validityEx}, let's imagine that $n_1$, $n_4$, and $n_5$ represent the same knowledge element $k_e$.
The lifecycle of $k_e$ is thus:
\begin{itemize}
	\item $n_1$ for period $[t_1, t_2)$,
	\item $n_4$ for period $[t_2, t_3)$,
	\item $n_5$ for period $[t_3, t_\infty)$.
\end{itemize}

Let $t_1'$ be a timepoint between $t_1$ and $t_2$.
When one wants to resolve the node representing the knowledge element at $t_1'$, she or he gets $n_1$ node, no matter of the node input ($n_1$, $n_4$, or $n_5$): $Z^{T}_1(n_4, t_1) = n_1$.
On the other hand, applying the same relation with another node ($n_2$ or $n_3$) returns another node.
For example, if $n_2$ and $n_3$ do not belongs to the same knowledge element, then it will return the node given as input, for example $Z^{T}_1(n_2, t_1) = n_2$.

\paragraph{Knowledge elements stored in nodes}
Nodes are used to store the different knowledge elements: context, requirements and actions.
The set of nodes $N$ is thus split in three subset: $N = C \cup R \cup A$ where $C$ is the set of nodes which store context information, $R$ a set of nodes for requirement information and $A$ the set of nodes for actions information.

Actions define a process that indirectly impact the context: they will change the behavior of the system, which will be reflected on the context information.
Requirements are also processes that are continuously run over the system in order to check the specifications.
Here, the purpose of the $A$ and $R$ subset is not to store these processes but to list them.
It can be thought as a catalogue of actions and requirements, with their history.

Using a high level overview, these processes can the depicted as: taking the knowledge as input, perform task, and modify this knowledge as output.
As detailed in the next two paragraphs, they can be formalized by relations.


\paragraph{Temporal queries for requirements}
At the current state, the formalism of the knowledge $K$ do not contain any information regarding the requirement processes.
To overcome this, system requirements processes $R_P$ are added such as $K = (N, E, V^T, Z^T, R_P)$.
$R_P$ is a set of patterns $P_{[t_j,t_k]}(K)$ and queries $Q$ over these patterns: $R_P = {P \cup Q}$. 

$P_{[t_j, t_k]}$ denotes a temporal graph pattern, where $t_j$ and $t_k$ are the lower and upper bound of the time interval respectively.
The time interval can be either fixed (absolute) or sliding (relative).
Each element of the pattern should be valid for at least one time point: $\forall~p \in P_{[t_j,t_k)}, V^T(e) \cap [t_j,t_k) \neq \emptyset$.
Patterns can be seen as temporal subgraph of $K$, with a time limiting constraint coming in the form of a time interval.
Temporal graph queries $Q$ consist commonly of two parts: (i) path description to traverse the graph nodes, at both structural and temporal dimensions; (ii) arithmetic expressions on nodes, edges, and attribute values.    

\paragraph{Temporal relations for actions}
Like for $R_P$, the knowledge $K$ needs to be augmented with the action processes $A_P$: $K = (N, E, V^T, Z^T, R_P, A_P)$.
Actions processes $A_P$ can be regarded as  a set of relations or isomorphisms mapping a source temporal graph pattern $P_{[t_j, t_k]}$ to a target one $P_{[t_l, t_m]}$,  $A_P : K \times I \rightarrow K \times I$.

The left-hand side of the relation depicts the temporal graph elements over which an action is applied.
Every relation may have a set of application conditions. 
They describe the circumstances under which an action should take place. 
These application conditions are either positive, should hold, or negative, should not hold. 
Application conditions come in the form of temporal graph  invariants.  
The side effects of these actions are represented by the right-hand side. 

Finally, we associate to $A_P$ a temporal function $E_{A_P}$ to determine the time interval at which an action has been executed. 
Formally, $X: A \rightarrow I$.

\paragraph{Temporal relations for decisions}
Finally, the knowledge formalism needs to include the last, but not the least, element: decisions made by the adaptation, $K = (N, E, V^T, Z^T, R_P, A_P, D)$
While the source of relations in $D$ represents the state before the execution of an action, the target shows its impact on the \gls{context}. 
Its intent is \textbf{to trace back impacts of actions execution to the decisions they originated from}.  

A decision present in ${D}$ is defined as a set of executed actions, \ie a subset of ${A_P}$.
Formally, ${D} = \{\ {A_D \cup R_D}~|~{A_D}  \subseteq A_P, R_D \subseteq R_P\}$.
We assume that each action should result from one decision: $\forall a \in {A}, \forall d1, d2 \in {D}~|~a \in d1 \wedge a \in d2 \rightarrow d1 = d2$.

The temporal function $E_{A_P}$ is extended to decision in order to represent the execution time: $E_{A_P}: (A \cup D) \rightarrow I$.
For decision, the lower bound of the interval correspond to the lowest bound of the action execution intervals.
Following the same principle, the upper bound of the interval correspond to the uppermost bound of the action execution intervals.
Formally, $\forall d \in D \rightarrow E_{A_P}(d) = [l,u)$, where $l = \displaystyle \min_{a \in A_d} \{E_{A_P}(a)[start]\}$ and $u = \displaystyle \max_{a \in A_d} \{E_{A_P}(a)[end]\}$.

\paragraph{Sum up}
Knowledge of an adaptive system can be formalism with a temporal graph such as $K = (N, E, V^T, Z^T, R_P, A_P, D)$, wherein:
\begin{itemize}
	\item $N$ is a set of nodes to represent the different information (context, actions and requirements)
	\item $E$ is a set of edges with connect the different nodes,
	\item $V^T$ is a temporal relation which defines the temporal validity of each elements,
	\item $Z^T$ is a relation to track the history of each knowledge elements,
	\item $R_P$ is a relation that define the different requirements processes,
	\item $A_P$ is a relation that define the different action processes,
	\item $D$ is a set of action processes that result from a same decision.
\end{itemize}

In the next section, we exemplify this formalism over our case study.


\subsection{Application on the use case}

In this section we apply the formalism described on the use case presented in Section~\ref{sec:tkm:intro:uc}.

\paragraph{Description of \pmb{$N_{SG}$}}

$N_{SG}$ is divided into three subset: $C_{SG}$, $R_{SG}$ and $A_{SG}$.
$R_{SG}$ contains one node, $R_1$ in Figure~\ref{fig:tkm:contextFormExample}, which represents the requirement of this example: $R_{SG} = \{R_1\}$
Two nodes, $A_1$ and $A_2$, belong to $A_{SG}$: $A_{SG} = \{ A_1, A_2\}$.
They represent represent the two actions of this example, respectively decreasing and increasing amps limits.
Regarding the context $C_{SG}$, there is three nodes to represent the three smart meters ($M_1$, $M_2$, and $M_3$), one for the substation ($S_1$), two for the fuses ($F_1$ and $F_2$), one for the dead-end cabinet ($D_{C_1}$) and one node per consumption value received ($V_i$): $C_{SG} = \{M_1, M_2, M_3, S_1, F_1, F_2, D_{C_1}\} \cup \{ V_i | i \in [1..9]\}$.

According to the scenario, all nodes are created at $t_0$ and are never modified, except for nodes to store consumption values.
Therefore, their validity period starts at $t_0$ and never ends: $\forall n \in A_{SG} \cup R_{SG} \cup \{M_1, M_2, M_3, S_1, F_1, F_2, D_{C_1}\}, V^T_{SG}(n) = [t_0, t_\infty)$.
Considering the consumption values, all the nodes represent the history of the values for the three smart meters.
In other words, there is three knowledge element: the consumption measured for each meter.
Let $C_i$ notes the consumption measured by the smart meter $M_i$.
As shown in Figure~\ref{fig:tkm:contextFormExample}, there is:
\begin{itemize}
	\item $C_1$ of $M_1$ is represented by $\{V_1, V_4, V_7\}$,
	\item $C_2$ of $M_2$ is represented by $\{V_2, V_5, V_8\}$,
	\item $C_3$ of $M_3$ is represented by $\{V_3, V_5, V_9\}$.
\end{itemize}
Taking $C_2$ as example, $V_2$ is the initial consumption value, replaced by $V_5$ at $t_1$, itself replaced by $V_8$ at $t_2$. 
Applying the $V_{SG}^T$ on these different values, results are thus:
\begin{itemize}
	\item $V_{SG}^T(V_2) = [t_0, t_1)$,
	\item $V_{SG}^T(V_5) = [t_1, t_2)$,
	\item $V_{SG}^T(V_8) = [t_2, t_\infty)$.
\end{itemize}
These validity periods are shown in Figure~\ref{fig:tkm:validityC2}.
As meters send the new consumption values at the same time, this example can be also applied to $C_1$ and $C_3$.

%\begin{figure}
%	\centering
%	\includegraphics[width=0.5\linewidth]{img/chapt-tkm/validitySchemaC2}
%	\caption{Validity periods of the consumption values of one smart meters}
%	\label{fig:tkm:validityC2}
%\end{figure}

\begin{figure}
	\centering
	\subfloat[Consumption values] {
		\includegraphics[width=0.4\linewidth]{img/chapt-tkm/formalism/validitySchemaC2}
		\label{fig:tkm:validityC2}
	}
	\hfil
	\subfloat[Edges]{
		\includegraphics[width=0.4\linewidth]{img/chapt-tkm/formalism/validitySchemaC2Edges}
		\label{fig:tkm:validityC2Edges}
	}
	\caption{Validity periods of the consumptions values and their edges to the smart meter $M_2$}
\end{figure}

%More generally, the $V_{SG}^T$ returns:
%\begin{condItemize}
%	\item $\forall i \in [1..3], V^T_{SG}(V_i) = [t_0, t_1)$,
%	\item $\forall i \in [4..6], V^T_{SG}(V_i) = [t_1, t_2)$,
%	\item $\forall i \in [7..9], V^T_{SG}(V_i) = [t_2, t_\infty)$.
%\end{condItemize}
From these validity period, the $Z^T_{SG}$ can be used to navigate to the different values over time.
Let's continue with the same example, $C_2$.
In order to get the evolution of the consumption value $C_2$, given the initial one, one will use the $Z^T_{SG}$ relation:
\begin{itemize}
	\item $Z^T_{SG}(V_2, t_{s1}) = V_2$, where $t_0 \leqslant t_{s1} < t_1$
	\item $Z^T_{SG}(V_2, t_{s2}) = V_5$, where $t_1 \leqslant t_{s2} < t_2$
	\item $Z^T_{SG}(V_2, t_{s3}) = V_8$, where $t_2 \leqslant t_{s3} < t_\infty$.
\end{itemize}
%More generally, the $Z^T_{SG}$ returns:
%\begin{condItemize}
%	\item for $0 \leqslant t_s < t_1$:
%		\begin{condItemize}
%			\vspace{-0.5em}
%			\item $\forall i \in \{1, 4, 7\}, Z^T_{SG}(V_i, t_s) = V_1$,
%			\item $\forall i \in \{2, 5, 8\}, Z^T_{SG}(V_i, t_s) = V_2$,
%			\item $\forall i \in \{3, 6, 9\}, Z^T_{SG}(V_i, t_s) = V_3$.
%		\end{condItemize}
%	\vspace{-0.5em}
%	\item for $t_1 \leqslant t_s < t_2$:
%		\begin{condItemize}
%			\vspace{-0.5em}
%			\item $\forall i \in \{1, 4, 7\}, Z^T_{SG}(V_i, t_s) = V_4$,
%			\item $\forall i \in \{2, 5, 8\}, Z^T_{SG}(V_i, t_s) = V_5$,
%			\item $\forall i \in \{3, 6, 9\}, Z^T_{SG}(V_i, t_s) = V_6$.
%		\end{condItemize}
%	\vspace{-0.5em}
%	\item for $t_2 \leqslant t_s < t_\infty$:
%		\begin{condItemize}
%			\vspace{-0.5em}
%			\item $\forall i \in \{1, 4, 7\}, Z^T_{SG}(V_i, t_s) = V_7$,
%			\item $\forall i \in \{2, 5, 8\}, Z^T_{SG}(V_i, t_s) = V_8$,
%			\item $\forall i \in \{3, 6, 9\}, Z^T_{SG}(V_i, t_s) = V_9$.
%		\end{condItemize}
%\end{condItemize}



\paragraph{Description of \pmb{$E_{SG}$}}

In this example, edges are used to store the relationships between the different context elements.
For example, the edge between the substation $S_1$ and the fuse $F_1$ allow to represent the fact that the fuse is physically inside the substation.
Another example, edges between the cable $C_1$ and the meters $M_1$, $M_2$ and $M_3$ represent the fact that these meters are connected to the smart grid through this cable.

One may consider that relations (validity, $Z^T$, decisions, action processes and requirements processes) will be stored as edges.
But, this decision is let to the implementation part of this formalism.

In our model, only consumption values ($V_i$ nodes) are modified.
Plus, since the scenario do not imply other edges modifications, only those between meters and values are modified.
The edge set contains thus sixteen edges: $E_{SG} = \{E_i \mid i \in [1..16] \}$.

By definition, the unmodified edges have a validity period starting from $t_0$ and never ends: $\forall i \in [1..7], V^T_{SG}(E_i) = [t_0, t\infty)$.
The history of the three knowledge elements that represent consumption values do not only impact the nodes which represent the values but also the edges between those nodes and the meters ones:
\begin{itemize}
	\item $C_1$ impacts edges between $M_1$ and $V_1$, $V_4$, and $V_7$, \ie $\{E_8, E_{11}, E_{14}\}$,
	\item $C_2$ impacts edges between $M_2$ and $V_2$, $V_5$, and $V_8$, \ie $\{E_9, E_{12}, E_{15}\}$,
	\item $C_3$ impacts edges between $M_3$ and $V_3$, $V_6$, and $V_9$, \ie $\{E_{10}, E_{13}, E_{16}\}$.
\end{itemize}

Continuing with $C_2$ as example, the initial edge value is $E_8$ from $t_0$, which is replaced by $E_{11}$ from $t_1$, itself replaced by $E_{14}$ from $t_2$.
The validity relation, applied on these edges, thus returns:
\begin{itemize}
	\item $V^T_{SG}(E_8) = [t_0, t_1) = V^T_{SG}(V_2)$,
	\item $V^T_{SG}(E_{11}) = [t_1, t_2) = V^T_{SG}(V_5)$,
	\item $V^T_{SG}(E_{14}) = [t_2, t_\infty) = V^T_{SG}(V_8)$,
\end{itemize}

These validity periods are depicted in Figure~\ref{fig:tkm:validityC2Edges}.
As they are driven by those of consumption values ($V_2$, $V_5$, and $V_8$), they are equals.

As for nodes, the $Z^T_{SG}$ relation can navigate over time through these values.
For example, to get the history of the edges between the consumption value $C_2$ and the meter represented by $M_2$, one can apply the $Z^T_{SG}$ relation as following:
\begin{itemize}
	\item $Z^T_{SG}(E_8, t_{s1}) = E_8$, where $t_0 \leqslant t_{s1} < t_1$,
	\item $Z^T_{SG}(E_8, t_{s2}) = E_8$, where $t_1 \leqslant t_{s1} < t_2$,
	\item $Z^T_{SG}(E_8, t_{s3}) = E_8$, where $t_2 \leqslant t_{s1} < t_\infty$.
\end{itemize}


\paragraph{Description of \pmb{$R_{P_{SG}}$}}
The requirement calls for minimizing overloads.
It means that when the system detects at least one overload, for example in cables, it will take counter actions.
As the system has prediction capabilities, it will not only check is there is one at the current time $t$ but also if one will come in the next hour.
The pattern will be defined as follow: $P_{[t, t+15min]}$.
To determine if there is an overload, the system needs to know: the current and future consumption, the current and future topology.
The last one is used to compute the loads from the consumption (cf. Section~\ref{sec:intro:use-case}).

Let's consider that time points are regular and there is one every 15 minutes and that current time is $t_0$.
The pattern, $P_{[t_0, t_1]}$, will thus contain all nodes that are valid between $t_0$ and $t_1$ (included):
\begin{itemize}
	\item all topology nodes between: $\{S_1, C_1, F_1, F_2, D_{C_1}, M_1, M_2, M_3\}$
	\item all consumption values between: $\{V_i \mid i \in [1..6]\}$,
	\item all edges that connected these nodes: $\{E_i \mid i \in [1..13]\}$
\end{itemize}

From these values, the loads is computed and the system checks that none will exceed the capacity of the infrastructure (cables, substations, cabinets).

\paragraph{Description of \pmb{$A_{P_{SG}}$}}
Now, let us assume that the execution of $R_{P_{SG}}$ detects an overload on the cable ($C_1$) at $t_0$.
The system decides to reduce the amps limits, and thus the load, on the three meters.
The action $A_1$ (decreasing amps limits) is thus executed three times: one time per meter.
For each of these action, the input context will correspond to the pattern used by the requirement relation: $P_{[t_0, t_1]}$.
The output context will contain the predicted values after the actions have been executed.
Here, the actions are executed in parallel and their execution time is in seconds.
So the impact will be visible from $t_1$.
So the output pattern contain the three values at $t_1$:  $P_{[t_1, t_1]}$.
In summary:
\begin{itemize}
	\item Action 1: $A_{P_1}: P_{[t_0, t_1]} \rightarrow P_{[t_1, t_1]}$,
	\item Action 1: $A_{P_2}: P_{[t_0, t_1]} \rightarrow P_{[t_1, t_1]}$,
	\item Action 1: $A_{P_3}: P_{[t_0, t_1]} \rightarrow P_{[t_1, t_1]}$.
\end{itemize}


\paragraph{Description of \pmb{$D_{SG}$}}
Following the scenario, there is one decision, $D_{SG_1}$, which try to achieve the requirement $R_1$ by executing the actions $A_1$: $A_{P_1}$, $A_{P_2}$, and $A_{P_3}$.
Then, here the decision is equals to: $D_{SG_1} = \{R_1, A_{P_1}, A_{P_2}, A_{P_3}\}$.

\paragraph{Summrarize}
Through this section, I explifyed how the formalism can be used to define an adaptation decision on a smart grid system.
As the decision contains information about the circumstances and the impact, one may use it to debug the process and/or try to explain the behavior of such systems.
 \section{Modeling the knowledge}
\label{sec:tkm:mm}
 
 In order to simplify the diagnosis of adaptive systems, this thesis proposes a novel \gls{metamodel} that combines, what we call, design elements and runtime elements.
Design elements abstract the different elements involved in \gls{knowledge} information to assist the specification of the adaptation process.
Runtime elements instead, represent the data collected by the adaptation process during its execution.
In order to maintain the consistency between previous design elements and newly created ones, instances of design elements (\eg actions) can be either added or removed.
Modifying these elements would consist in removing existing elements and creating new ones.
Combining design elements and runtime elements in the same model helps not only to acquire the evolution of system but also the evolution of its structure and specification (e.g. evolution of the requirements of the system).
Design time elements are depicted in gray in the Figures~\ref{fig:knowledge-mm}--~\ref{fig:action-mm}.
Note that, this thesis does not address how runtime information is collected.

For the sake of modularity, the \gls{metamodel} has been split into four packages: Knowledge, Context, Requirement and Action.
All the classes of these packages have a common parent class that adds the temporality dimention: \textit{TimedElement} class.
Before describing the Knowledge (core) package, we detail this element.
Then, we introduce in more details the other three packages used by the Knowledge package: Context, Requirement, and Action. 
In below sections, we use "\textit{Package::Class}" notation to refer to the provenance of a class.
If the package is omitted, then the provenance package is this one described by the figure or text.

\subsection{Parent element: \textit{TimedElement} class}
we assume that all the classes in the different packages extend a \textit{TimedElement} class. 
This class contains three methods: \textit{startTime}, \textit{endTime}, and \textit{modificationsTime}.
The first two methods allow accessing the validity interval bounds defined by the previously discussed $V^T$ relation.
The last method resolves all the timestamps at which an element has been modified: its history. 
This method is the implementation of the relation $Z^T$ described in our formalism (cf. Section~\ref{sec:tkm:k-formalism:formalism}).


\subsection{Knowledge metamodel}

\begin{figure*}[t]
	\begin{center}
	\includegraphics[width=.9\linewidth]{img/chapt-tkm/mm/knowledge-mm}
	\caption{Excerpt of the knowledge metamodel}
	\label{fig:knowledge-mm}
	\end{center} 
\end{figure*}

In order to enable interactive diagnosis of adaptive systems, traceability links between the decisions made and their circumstances should be organized in a well-structured representation. % todo still correct according to the global blabla
In what follows, we introduce how the \gls{knowledge} \gls{metamodel} helps to describe \glspl{decision}, which are linked to their  goals and their context (input and impact). 
Figure~\ref{fig:knowledge-mm} depicts this \gls{metamodel}.

Knowledge package is composed of a \textit{\gls{context}}, a set of \textit{\glspl{requirement}}, a set of \textit{strategies}, and a set of \textit{\glspl{decision}}.
A \gls{decision} can be seen as the output of the Analyze and Plan steps in the \gls{mapek} loop.

Decisions comprise target \textit{goals} and trigger the execution of one \textit{tactic} or more.  
A decision has an \textit{input} context and an \textit{impacted} context.
The context impacted by a decision  (\textit{Decision.impacted}) is a derived relationship computed by aggregating the impacts of all actions belonging to a decision (see Fig.~\ref{fig:action-mm}).
Likewise, the \textit{input} relationship is derived and can be computed similarly. 
In the smart grid example, a decision can be formulated (in plain English) as follows: since the district D is almost overloaded (\textit{input context}), we reduce the amps limit of greedy consumers using the ``\textit{reduce amps limit}" \textit{action} in order to reduce the load on the cable of the district (\textit{impact}) and satisfy the ``\textit{no overload}" policy (\textit{requirement}).

As all the elements inherit from the \textit{TimedElement}, we can capture the time at which a given decision and its subsequent actions were executed, and when their impact materialized, \ie measured.
Thanks to this metamodel representation, a developer can apprehend the possible causes behind malicious behavior by navigating from the context values to the decisions that have impacted its value (\textit{Property.expected.impact}) and the goals it was trying to reach (\textit{Decision.goals}).
In Section~\todo{add reference}, we present an example of interactive diagnosis queries applied to the smart grid use case.

\subsection{Context metamodel}
\begin{figure*}
  \begin{center}
      \includegraphics[width=.8\linewidth]{img/chapt-tkm/mm/contextModel}
      \caption{Excerpt of the context metamodel}
      \label{fig:context-model}
  \end{center}	
\end{figure*}

Context models structure context information acquired at runtime. 
For example, in a smart-grid system, the context model would contain information about smart-grid users (address, names, etc.) resource consumption, etc.

An excerpt of the context model is depicted in Figure~\ref{fig:context-model}. 
we propose to represent the context as a set of structures (\textit{Context.structures}) and global attributes (\textit{Context.globals}).
A structure can be viewed as a C-structure with a set of properties (\textit{Property}): attributes (\textit{Attribute}) or relationships (\textit{Relation}).
A structure may contain other nested structures (\textit{Structure.inner}).
Structures and properties have values.
They correspond to the nodes described in the formalization section (\cf Section~\ref{sec:tkm:k-formalism:formalism}).
The connection feature described in Section~\ref{sec:adaptation-req} is represented thanks to three recursive relationships on the Property class: \textit{consistentWith}, \textit{computedUsing} and \textit{influence}.
Additionally, each property has a source (\textit{Source}) and an uncertainty (\textit{Uncertainty}).
It is up to the stakeholder to extend data with the appropriate source: measured, computed, provided by a user, or by another system (\eg weather information coming from a public API).
Similarly, the uncertainty class can be extended to represent the different kinds of uncertainties. Finally, a property can be either historic or static.

\subsection{Requirement metamodel}

\begin{figure}
	\centering
	\includegraphics[width=0.7\linewidth]{img/chapt-tkm/mm/requirementModel}
	\caption{Requirement metamodel}
	\label{fig:requirement-model}
\end{figure}

As different solutions to model system requirements exist (\eg KAOS~\cite{dardenne1993goal}, i*~\cite{yu2011modelling} or Tropos~\cite{DBLP:journals/aamas/BrescianiPGGM04}), in this metamodel, we abstract their shared concepts.
The requirement model, depicted in Figure~\ref{fig:requirement-model}, represents the \textit{requirement} as a set of \textit{goals}.
Each goal has a \textit{nature} and a textual specification.
The nature of the goals adheres to the four categories of requirements presented in Section~\ref{sec:adaptation-req}.
One may use one of the existing requirements modeling languages (\eg RELAX) to define the semantics of the requirements. 
Since the requirement model is composed solely of design elements, we may rely on static analysis techniques to infer the requirement model from existing specifications.
The work of Egyed~\cite{egyed01} is one solution among others.
This work is out of the scope of the paper and envisaged for future work. 

In the guidance example, the requirement model may contain a \textbf{balanced resource distribution} requirement.
It can be split into different goals: (i) \textit{minimizing overloads}, (ii) \textit{minimizing production lack}, (iii) \textit{minimizing production loss}.

\subsection{Action metamodel}
\label{sec:action-mm}

\begin{figure}
	\centering
	\includegraphics[width=0.8\linewidth]{img/chapt-tkm/mm/actionModel}
	\caption{Excerpt of the action metamodel}
	\label{fig:action-mm}
\end{figure}

Similar to the requirements metamodel, the actions metamodel also abstracts main concepts shared among existing solutions to describe adaptation processes and how they are linked to the context. 
Figure~\ref{fig:action-mm} depicts an excerpt of the action metamodel.
we define a strategy as a set of tactics (\textit{Strategy}).
A tactic contains a set of actions (\textit{Action}).
A tactic is executed under a precondition represented as a temporal query (\textit{TemporalQuery}) and uses different data from the context as input.
In future work, we will investigate the use of preconditions to schedule the executions order of the actions, similarly to existing formalisms such as Stitch~\cite{DBLP:journals/jss/ChengG12}.
The query can be as complex as needed and can navigate through the whole knowledge model.
Actions have impacts on certain properties, represented by the \textit{impacted} reference. 

The different executions are represented thanks to the \textit{Execution} class. Each execution has a status to track its progress and links to the impacted context values(\textit{Execution.impactedValues}).
Similarly, input values are represented thanks to the \textit{Execution.inputValues} relationship.
An execution has \textit{start} and \textit{end} time. Not to confuse with the \textit{startTime} and \textit{endTime} of the validity relation $V^T$.
Whilst the former corresponds to the time range in which a value is valid, the \textit{start} and \textit{stop} time in the class execution correspond to the time range in which an action or a tactic was being executed.
The start and stop attributes correspond to the relationL $E_{A_E}$ (see Section~\ref{sec:formalism}). These values can be derived based on the validity relation.
They correspond to the time range in which the status of the execution is "\textit{RUNNING}".
Formally, for every execution node $e$, $E_{A_E}(e)~=~(V(e)~|~e.status~=~$"RUNNING"$)$.


Similarly to requirement models, it is possible to automatically infer design elements of action models by statically analyzing actions specification.
Since acquiring information about tactics and actions executions happens at runtime, one way to achieve this is by intercepting calls to actions executions and updating the appropriate action model elements accordingly.
This is out of the scope of this paper and planned for future work.


 
 






