\section{Introduction}
 
%should define: decision, action, context, knowledge

Adaptive systems have the capabilities of adjusting their behavior to dynamic changes encountered in their environments~\cite{DBLP:conf/dagstuhl/BrunSGGKLMPS09}. 
To do so, they make adaptation decisions, in the form of actions, based on high-level policies. 
These adaptations are often relevant to achieving a goal or an objective while satisfying a set of constraints~\cite{DBLP:journals/eswa/Macias-EscrivaHTH13}.
One of the successful approaches used to achieve this adaptability is the closed control loop paradigm, in particular, the MAPE-K loop~\cite{kephart2003vision}. %~\cite{kephart2003vision, ibm2005architectural}. 
It consists of monitoring the well-behaving of the system with regards to a set of requirements, either functional or non-functional. 
Once the system deviates from its normal behavior, the control loop selects and executes a set of actions (adaptation tactic) to remediate the system behavior.\looseness=-1

Faced with growingly complex and large-scale software systems (e.g. smart grid systems), we can all agree that the presence of residual defects becomes unavoidable~\cite{DBLP:conf/icse/BarbosaLMJ17}. %~\cite{DBLP:conf/icse/BarbosaLMJ17, DBLP:conf/icse/MongielloPS15, DBLP:conf/icse/HassanBB15}. 
Even with a meticulous verification or validation process, it is very likely to run into an unexpected behavior that was not foreseen at design time. Alone, existing formal modeling and verification approaches may not be sufficient to anticipate these failures~\cite{DBLP:conf/icse/TaharaOH17}. 
As such, complementary techniques need to be proposed to locate the anomalous behavior and its origin in order to handle it in a safe way. \looseness=-1 

As there might be many probable causes behind an abnormal behavior, developers usually perform a set of diagnosis routines to narrow down the scope or origin of the failure. One way to do so is by investigating the satisfaction of its requirements and the decisions that led to this system state, as well as their timing~\cite{DBLP:conf/iceccs/BencomoWSW12}.  
In this perspective, developers may set up a set of systematic questions that would help them understand why and how the system is behaving in such a way. \looseness=-1
%These questions may comprise: (i) what goal(s) the system was trying to reach by executing a tactic $a$?, (ii) what were the circumstances used by a decision $d$ and its expected impact on the context?, (iii) what decision(s) influenced the system's context at a time $t$?
%\begin{itemize}
%   \item what goal(s) the system was trying to reach by executing a tactic $a$? 
%   \item what were the circumstances used by a decision $d$ and its expected impact on the context?
%   \item what decision(s) influenced the system's context at a time $t$? 
%\end{itemize}

Bencomo~\etal~\cite{DBLP:conf/iceccs/BencomoWSW12} argue that comprehensive explanation about the system behavior contributes drastically to the quality of the diagnosis, and eases the task of troubleshooting the system behavior. To enable this, we believe that adaptive software systems should be equipped with traceability management facilities to link the decisions made to their \textbf{(i) circumstances, that is to say, the history of the system states and the targeted requirements, and (ii) the performed actions with their impact(s) on the system}.
In particular, an \textbf{adaptive system should keep a trace of the relevant historical events}.
Additionally, it should be able to \textbf{trace the goals intended to be achieved by the system to the adaptations and the decisions that have been made, and vice versa}. 
Finally, in order to enable developers to interact with the system in a clear and understandable way, appropriate abstraction to \textbf{enable the navigation of the traces and their history should also be provided}. %Unfortunately, suitable solutions to support these features are under-investigated. 

Existing approaches~\cite{hassel13,heinrich14,ehlers11,DBLP:conf/icse/MendoncaAR14,DBLP:conf/icse/CasanovaGSA14,DBLP:conf/icse/IftikharW14a} are accompanied by built-in monitoring rules and do not allow to interact with the underlying system in a simple way. 
Moreover, they do not keep track of historical changes as well as causal relationships linking requirements to their corresponding adaptations. Only flat execution logs are stored. 

In this paper, we propose a framework to structure and store the state and behavior of a running adaptive system, together with a high-level API to efficiently perform diagnosis routines. 
Our framework relies on a temporal model-based solution that efficiently abstracts decisions and their corresponding circumstances.
Specifically, based on existing approaches for modeling and monitoring adaptation processes, we identify a set of properties that characterize context, requirements, and actions in self-adaptive systems.    
Then, we formalize the common core concepts implied in adaptation processes, also referred to as knowledge, by means of temporal graphs and a set of relations that trace decisions impact to circumstances. Finally, thanks to exposing common interfaces in adaptive processes, existing approaches in requirements and goal modeling engineering can be easily integrated into our framework. 

%\begin{itemize}
%   \item Based on existing approaches for modeling a monitoring adaptation process, we identify a set of properties that characterize context, requirements, and action in self-adaptive systems.    
%  \item Based on these properties, we formalize the common core concepts implied in adaptation processes, also referred to as knowledge, by means of temporal graphs and a set of relations that trace decisions impact to circumstances.
%   \item Based on the properties and the formalism, we propose a data model to design and store the knowledge, and a high-level API specify and perform diagnosis algorithms.
%Thanks to exposing common interfaces in adaptive processes, existing approaches in requirements and goal modeling engineering can be easily integrated into our framework. 
%\end{itemize}

%We demonstrate the applicability of our approach by applying it to a smart-grid case study and showing how developers can design algorithms to answer the above diagnosis questions. 
%We demonstrate the applicability of our approach by showing how developers can design algorithms to answer the above diagnosis questions.
%We validate the scalability of our implementation by running it on a randomly generated temporal graph with increasing size. 
%The evaluation results show that we can perform complex queries traversing more than 100,000 in less than 2.5 seconds, with an acceptable memory footprint.

The rest of the paper is structured as follows. 
We first describe a guidance example in Section~\ref{sec:example}, based on the smart grid system at Luxembourg.
In Section~\ref{sec:adaptation-process}, we summarize core concepts manipulated in adaptation processes and their characteristics.
Then, we provide a formal definition of these concepts in Section~\ref{sec:formalism}.
Later, we describe the proposed data model in Section~\ref{sec:knowledge-model}.
%In Section~\ref{sec:eval}, we first describe how a stakeholder can use our approach and then we validate our approach and evaluate its performance.
In Section~\ref{sec:eval}, we demonstrate the applicability of our approach by applying it to the smart grid example.
% Then, we validate the scalability of our default implementation by running it on a randomly generated temporal graph with increasing size. 
% The evaluation results show that we can perform queries traversing more than 100\,000, roughly 5 days history of the Luxembourg smart grid, in less than 2.5 seconds, with an acceptable memory footprint.
Before concluding the paper in Section~\ref{sec:conclusion}, we introduce some related work in Section~\ref{sec:relwork}.