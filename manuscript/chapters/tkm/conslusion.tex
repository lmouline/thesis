\section{Conclusion}
\label{sec:tkm:conclusion}

Adaptive systems are prone to faults  given their evolving complexity.
To enable interactive diagnosis over these systems, we proposed a temporal data model to abstract and store knowledge elements. We also provided a high-level API to specify and perform diagnosis algorithms.
Thanks to this structure, a stakeholder can abstract and store decisions made by the adaptation process and link them to their circumstances --targeted requirements and used context-- as well as their impacts.
In our evaluation, we showed that our solution can efficiently handle up to 100\,000 elements, in a single machine.
This size is comparable to 5 days history of one district in the Luxembourg smart grid.


Throughout this work, we assumed that designers are able to link actions with their expected impacts at design time. However, this is not always true. Some impacts  cannot be known in advance.
In this perspective, in addition to the future plans already mentioned throughout the chapter, we will investigate techniques to identify unknown impacts on the context model, for instance, by studying the use of machine learning  techniques.
In order to improve the accuracy and correctness of diagnosis routines, another aspect to be considered for future work is handling uncertainty in self-adaptive systems.
Understanding the effect of uncertainty on the development of self-adaptive systems and their diagnosis is still an open question.
We plan to explore this research direction by answering the following questions: How to represent and express uncertainty in self-adaptive systems at design time?
How to efficiently interrogate data with uncertainty in self-adaptive systems,  for instance, for troubleshooting purpose?