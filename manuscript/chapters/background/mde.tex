\section[Model-Driven Engineering]{\glsdesc{mde}}

Abstraction, also called modelling, is the heart of all scientific discipline, including computer science~\cite{DBLP:journals/cacm/Kramer07}.
One will abstract a system, computer or not, a problem, or a solution to reason about it for a specific purpose.
Abstraction reduces the scope to relevant parts, removing the superfluous that complexity the understanding.
For example, a climatologist models all elements that impact the global climate (wind, ocean current, temperature, ...), ignoring local information, like the temperature under a forest, or global one, like the solar system.
In contrast, astronomers model the solar system, ignoring all information regarding the Earth climate.

In computer science, different modelling formalisms have been proposed. 
We can cite the entity-relationship~\cite{DBLP:journals/tods/Chen76} that is used to describe data model for relational databases.
In the web community, they use the ontology~\cite{DBLP:journals/ijmms/Gruber95} formalism to define the semantic web\footnote{Semantic web is defined an en extension of the Web to enable the processing by machines.}~\cite{berners2001semantic}.
The software engineering community uses the \gls{uml}~\cite{omg2017umlspec} to formalism software system structure or behaviour.

Extending this need for abstraction to all software engineering activities, practitioners have proposed the \gls{mde} methodology~\cite{DBLP:journals/computer/Schmidt06,DBLP:conf/ifm/Kent02}.
This methodology advocates the use of models, or abstractions, as primary software artefacts~\cite{DBLP:journals/software/WhittleHR14}.
The contributions proposed in this thesis are in line with this vision.
In the following sections, we give an overview of the \gls{mde} methodology and how we will use it.


\subsection{\gls{mde}: principles and vision}
%- vision: use models for all software engineering activities: design, maintenance, software evolution, ...~\cite{DBLP:journals/computer/Schmidt06, DBLP:series/synthesis/2017Brambilla, DBLP:conf/icse/HutchinsonRW11, DBLP:conf/uml/BakerLW05, DBLP:conf/icse/HutchinsonWRK11}
%- MDE eased software evolution \cite{DBLP:journals/software/AtkinsonK03a}
%- idea: reduce gap between the problem space and the solution space by providing two mechanism: DSML (language specific to a domain (type system, semantics, concepts) and transformation engine \cite{DBLP:journals/computer/Schmidt06}
%- goal: reduce accidental complexity in software development~\cite{DBLP:conf/icse/FranceR07, DBLP:journals/computer/Schmidt06}

%- separation of concern principle: use of multiple domain-specific languages that provide appropriate abstractions
%- this principle help stakeholders to focus on their problem without loosing the big picture
%- 2nd principle: reduce gap between high and low level models by generating artifacts (components, documentaion, test cases, source code, ....

%- pros \cite{DBLP:journals/computer/Schmidt06, DBLP:conf/ifm/Kent02, DBLP:conf/uml/BakerLW05}:
%    - abstraction from specific techno., which improves portability and interoperability
%    - automated code generation and transformation, that increase productivity and efficiency
%    - model checking and verification that improve quality
%    - separation of application and infrastructure code, reusability, readability, and cost-effectiveness
%- cons \cite{DBLP:conf/ifm/Kent02, DBLP:conf/uml/BakerLW05, DBLP:conf/models/WhittleHRBH13, DBLP:conf/icse/HutchinsonRW11}:
%    - high initial effort for defining DSMLs
%    - scalability
%    - poor tool support
%    - organisational challenges 
%    - negative impact of abstraction

%- 3 fundamentals concepts: metamodels, models, and model transformations
%- metamodels: represent how a model is structured, define the different concept and how their relationships
- models: conform to metamodel, can be thought as an instance of metamodel
- model transformation: operation part of MDE that refine models and generate artifacts (source code, ...)


\paragraph{Global overview}    
Software systems tend to be more and more complex.
To tame this complexity~\cite{DBLP:conf/icse/FranceR07, DBLP:journals/computer/Schmidt06}, the \gls{mde} methodology suggests  to use models for all the steps of software development and maintenance~\cite{DBLP:journals/computer/Schmidt06, DBLP:series/synthesis/2017Brambilla, DBLP:conf/icse/HutchinsonRW11, DBLP:conf/uml/BakerLW05, DBLP:conf/icse/HutchinsonWRK11, DBLP:journals/software/AtkinsonK03a}: design, evolution, validation, etc.
The core idea of this approach is two reduce the gap between the the problem and the solution space~\cite{DBLP:journals/computer/Schmidt06}.
Two main mechanisms have been defined: Domain Specific Modelling Languages (DSML) and model transformation.
The former is based on the separation of concern principle.
Each concern\footnote{The definition of concern is intentionally left undefined as it is domain specific.} should be addressed with a specific language, which manipulates concepts, has a type system, and a semantics dedicated to this concern.
These languages allow to create and manipulate models, specific for a domain.
The latter allows engineers to automatically generate software artefacts, such as documentation, source code, or test cases.
Using these mechanisms, stakeholders can focus on their own problem keeping in ming the big picture.
A well known examples is the Structured Query Language \linebreak (SQL)~\cite{SQL:Spec}
Using this language, engineers can query a relational database, the data model being the model.
They don't have to consider indexes (hidden behind the concept of, for example, primary keys) or all the mechanisms to persist and retrieve data from the disk.

\paragraph{Advantages and disadvantages}
Defenders of the \gls{mde} approach mainly highlight the benefits of abstraction in software engineering\cite{DBLP:journals/computer/Schmidt06, DBLP:conf/ifm/Kent02, DBLP:conf/uml/BakerLW05}.
First, using a same model, engineers can target different specific platforms.
For example, the ThingML~\cite{DBLP:conf/models/HarrandFMH16} language allow to specify the behaviour of system through state machines.
A same ThingML can be deployed on different platform such a Arduino, Raspberry Pi.
Second, thanks to the transformation engine, the productivity and efficiency of developers is improved.
Third, models allow engineers to implement verification and validation techniques, like model checking~\cite{DBLP:books/daglib/0020348}, which improve the software quality.
Finally, the models enable the separation of application and infrastructure code and the reusability of models.

However, the literature has also identified some drawbacks of the \gls{mde} approach~\cite{DBLP:conf/ifm/Kent02, DBLP:conf/uml/BakerLW05, DBLP:conf/models/WhittleHRBH13, DBLP:conf/icse/HutchinsonRW11}.
First, it requires a big initial effort when the DSML needs to be defined.
Second, current approaches do not allow the definition of very large models.
This drawback can be mitigated with new approaches such as NeoEMF~\cite{DBLP:conf/ecmdafa/BenelallamGSTL14, DBLP:journals/scp/DanielSBTVGC17}, which enable the storage of large models,  and Mogwaï~\cite{DBLP:conf/rcis/DanielSC16}, a query engine for large models.
Third, this approach suffers of a poor tooling support, as they should be reimplemented for each model.
As for the second drawback, recent contributions try to remove this limitation.
For example, we can cite the work of Bousse \etal \cite{DBLP:journals/jss/BousseLCWB18} that define a generic omniscient debugger\footnote{Debugger, generally, allow to execute step by step a programs, that is, forward. An omniscient debugger is also able to go backward: to navigate back in the previous states of a program~\cite{DBLP:journals/corr/cs-SE-0310016}.} for DSML.
Third, introducing \gls{mde} in a software development team introduces organisational challenges.
It changes the way developers interact and work together.
Finally, abstraction is a two edges sword.
Indeed, reasoning at an abstract level may be more complex as some prefer to work with concrete examples and based on simulation.

\paragraph{Fundamentals concepts}
\gls{mde} is based on three fundamentals concepts: metamodel, model, and model transformation.
Metamodels represent how models are structured.
They define the different concept and their relationships that can be used in models.
They are also seen as models of models.
Metamodels are then instantiated into models.
And model transformations, which are the operational part of \gls{mde}, process these models.

In this thesis we do not use 


\subsection{Metamodel, model}

\paragraph{Metamodel}
 - "A  meta-model  is  the  explicit  specification  of  an  abstraction (a simplification)" \cite{DBLP:conf/kbse/BezivinG01}
  
 - defines the set of concepts, relationships, and semantics tules regulating how models can be denoted in a particular language definition
        - materialize the knowledge of a given domain (concepts, their relations) \cite{DBLP:conf/iceccs/BezivinJT05}
- metamodel: model of a model
- uses also to define the syntax of DSL \cite{DBLP:journals/corr/ClarkSW15}
- contains metaclass with properties (attributes or references), inheritence --> basically a class diagram

- MetaObject Facicilty (MOF) (6.) is the standard metamodeling architecture proposedby the OMG, nuilt around a set of modeling standard
        - OCL (7.): constraints language
        - XML Metadata Interchange (XMI) (8.): data format to store MOF-based models

\paragraph{Model}
- conform to a metamodel / instanciation of a metamodel
	- satisfies all rules of the metamodel
        - each object in the model is an instance of a meta-class and statisfies all semantics rules
- models are used to perform verification and validation techniques such as simulation, model checking.
- 2 broad class of models~\cite{DBLP:conf/icse/FranceR07}
    - development models: abstraction above code level (e.g., requirements, architectural, implementation, and deployment)
    - runtime models: abstraction of a runtime phenomena   , views of some aspect of an executing system (models@run.time)
- (partial) representation of a system/domain that captures some of its characteristics into an abstraction that can be easily understood and manipulated by designers
    - Models are defined using a formal or semi-formal langiage such as UML
    
\paragraph{}
- models / metamodels are organized in multiple levels (metamodeling stack) related by the conformance relationship
 - 4 layers of modeling by MOF
        - M3: represents te language used by MOF to define M2 models, define itself
        - M2: UML metamodel 
        - M3: user model expressed in UML
        - M0: runtime layer (instances)
        
\subsection{Models@run.time}
-MDE paradigm \cite{DBLP:journals/computer/MorinBJFS09, DBLP:journals/computer/BlairBF09}
- uses models both at design and runtime to support self-adaptive systems
- model-based abstraction causally connected to their own current state
- model reflects the current state of the system
- adaptation can be made at the model level thanks to the causally connection
- model of the problem space, not the solution space

- design time: models support the design and implementation of the system
- the same (or similar) models are then embedded at runtime in order to support reasoning process
- models use as context model, and used in reasoning activities

- often mentioned in the context of architectural and variability models \cite{DBLP:journals/computer/MorinBJFS09, DBLP:journals/computer/GeorgasHT09, DBLP:journals/computer/CetinaGFP09}

\subsection{Tooling}
- def: development platforms that offer technical solutions to create, manipulate, and persist (meta) models
- provide low level programming interface to interact with models
- graphical tools to ease (meta) model definition, constraint specification, or model element creations
- embeds code generator to create a set of software artefacts from models

- Standard tool: EMF~\cite{steinberg2008emf}
- de-facto standard baseline framework to build DSL and modeling tools within the Eclipse ecosystem
- embeds own metamodeling language ECore
        - subset of UML
        - reference implementation of E-MOF, subset of MOF that correspond to facilities found in OO languages
    - Ecore model = model of the classes of a software application, used to generate Java code that allows manipulatingconceptual elements  at the application level
    - different way to define models
        - annotated Java, UML, XML, graphical/textual modelling tools
- Specification of an application's data (object attributes, relationships between objects, operation available on each object, simple constraint)
            - subset of class diagram of UML
        
- Pointed by Fouquet \etal \cite{DBLP:journals/corr/FrancoisNMDBPJ14, DBLP:conf/models/FouquetNMDBPJ12}, EMF has some limitation to adress m@r.t requirement
- low memory footprint
- small dependencies size
- thread safety
- efficient model (un)marshalling and cloning
- connecting m@r.t to classical design tools 

- This work has been extended with the GreyCat Modeling Environment~\footnote{https://github.com/datathings/greycat/tree/master/modeling}
- Based on the GreyCat temporal graph database~\cite{DBLP:journals/is/HartmannFMRT19}
- Designed for m@r.t \cite{DBLP:conf/seke/0001FNMKT14, DBLP:conf/models/Moawad0FNKT15,} + high scalable \cite{DBLP:phd/basesearch/Hartmann16, DBLP:conf/seke/0001FJRT17}
- also introduces machine learning \cite{DBLP:journals/sosym/0001MFT19}
- GCM: allow to define model with time as first-class concepts
	- all elements are automaticcaly augmented with a temporal dimension
- from the meta-model definition, generates a Java/Javascript API to manipulate the model (here, the temporal graph)

\subsection{Modelling in the context of this thesis}

- models@run.time designed for adaptive systems
- in this thesis we focus on the definition of this model

- thesis is based on the work of Hartmann et al.,
- it propose an approach to add time as a first-class concept for m@r.t
- we will use the GreyCat Modelling Environment (GCM)

