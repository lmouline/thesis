\section{Adaptive systems}

\subsection{From autonomic computing to }

Before formalising and modelling decisions and their circumstances, we abstract common concepts implied in an adaptation process. We refer to these concepts as the knowledge.

\subsubsection{General concepts of adaptation process}

Similar to the definition provided by Kephart~\cite{DBLP:journals/computer/KephartC03}, IBM  defines adaptive systems as ``a computing environment with the ability to manage itself and \textbf{dynamically adapt} to change in accordance with \textbf{business policies and objectives}. [These systems] can perform such activities based on \textbf{situations they observe or sense in the IT environment} [...]"~\cite{computing2006architectural}.

Based on this definition, we can identify three principal concepts involved in adaptation processes.
The first concept is  \textit{actions}. They are executed in order to perform a dynamic adaptation through actuators.
The second concept is \textbf{business policies and objectives}, which is also referred to as the \textbf{system requirements} in the domain of (self-)adaptive systems.
The last concept is the observed or sensed \textbf{situation}, also known as the \textbf{context}.
The following subsections provide more details about these concepts.

\subsubsection{Context}

In this thesis, we use the widely accepted definition of context provided by \linebreak Dey~\cite{DBLP:journals/puc/Dey01}: \textquote{Context is \textbf{any information that can be used to characterize} the situation of an entity. An entity is a person, place, or object that is considered relevant to the interaction between a user and [the system], including the user and [the system] themselves}.
In this section, we list the characteristics of this information based on several works found in the literature~\cite{DBLP:conf/pervasive/HenricksenIR02, DBLP:conf/seke/0001FNMKT14, DBLP:journals/percom/BettiniBHINRR10, DBLP:journals/comsur/PereraZCG14}.
We use them to drive our design choices of our Knowledge meta-model (cf. Section~\ref{sec:tkm:mm:knoeldge}).

\paragraph{Volatility}
Data can be either \textbf{static} or \textbf{dynamic}.
Static data, also called frozen, are data that will not be modified, over time, after their creation~\cite{DBLP:conf/pervasive/HenricksenIR02, DBLP:journals/comsur/MakrisSS13, DBLP:journals/percom/BettiniBHINRR10}.
For example, the location of a machine, the first name or birth date of a user can be identified as static data. 
Dynamic data, also referred to as volatile data, are data that will be modified over time.

\paragraph{Temporality}
In dynamic data, sometimes we may be interested not only in storing the latest value, but also the previous ones~\cite{DBLP:conf/seke/0001FNMKT14, DBLP:conf/pervasive/HenricksenIR02}. 
We refer to these data as \textbf{historical} data.
Temporal data is not only about past values, but also future ones. 
Two kinds of future values can be identified, \textbf{predicted} and \textbf{planned}.  
Thanks to machine learning or statistical methods, dynamic data values can be \textbf{predicted}. 
\textbf{Planned} data are set by a system or a human to specify planned modification on the data.

\paragraph{Uncertainty}
One of the recurrent problems facing context-aware applications is the data uncertainty~\cite{DBLP:conf/dagstuhl/LemosGMSALSTVVWBBBBCDDEGGGGIKKLMMMMMNPPSSSSTWW10, DBLP:conf/pervasive/HenricksenIR02, DBLP:journals/comsur/MakrisSS13, DBLP:journals/percom/BettiniBHINRR10}.
Uncertain data are not likely to represent the reality. They contain a noise that makes it deviate from its original value.
This noise is mainly due to the inaccuracy and imprecision of sensors.
Another source of uncertainty is the behaviour of the environment, which can be unpredictable.
All the computations that use uncertain data are also uncertain by propagation.

\paragraph{Source}
According to the literature, data sources are grouped into two main categories, either sensed (measured) data or computed (derived) \linebreak data~\cite{DBLP:journals/comsur/PereraZCG14}.
We refine this with an additional category called profiled.
Profiled data may be set either by a user (\textbf{profiled by a human}) or by an external system (\textbf{profiled by an external}).

\paragraph{Connection}
Context data entities are usually linked using three kinds of connections: conceptual, computational, and consistency~\cite{DBLP:conf/pervasive/HenricksenIR02, DBLP:journals/percom/BettiniBHINRR10}.
The conceptual connection relates to  (direct) relationships between entities in the real world (e.g. smart meter and concentrator).
The computational connection is set up when the state of an entity can be linked to another one by a computation process (derived, predicted). 
Finally, the consistency connection relates entities that should have consistent values. For instance, temperature sensors belonging to the same geographical area.

\subsubsection{Requirement}
\label{sec:adaptation-req}

Adaptation processes aim at modifying the system state to reach an optimal one.
All along this process, the system should respect the \textbf{system requirements} established ahead. 
Through this paper, we use the definition provided by IEEE~\cite{iso2017systems}: ``(1) Statement that translates or expresses a need and its associated \textbf{constraints} and \textbf{conditions}, (2) \textbf{Condition or capability that must be met or possessed} by a system [...] to satisfy an agreement, standard, specification, or other formally imposed documents".\looseness=-1

Although in the literature, requirements are categorized as functional or non-func\-tional, in this paper we use a more elaborate taxonomy introduced by Glinz~\cite{DBLP:conf/re/Glinz07}.
It classifies requirements in four categories: functional, performance, specific quality, and constraint.
All these categories share a common feature: they are all temporal.
During the life-cycle of an adaptive system, the developer can update, add or remove some requirements~\cite{DBLP:conf/icse/ChengA07, pandey2010effective}.

\subsubsection{Action}
In the IEEE Standards~\cite{iso2017systems}, an action is defined as: \textquote{\textbf{process of transformation} that \textbf{operates upon data} or other types of inputs to create data, produce outputs, or \textbf{change the state} or condition of the subject software}.

Back to adaptive systems, we can define an action as a process that, given the context and requirements as input, adjusts the system behaviour.
This modification will then create new data that correspond to an output context. In the remainder of this paper, we refer to output context as impacted context, or simply impact(s).
Whereas requirements are used to add preconditions to the actions, context information is used to drive the modifications.
Actions execution have a start time and a finish time. They can either succeed, fail, or be canceled by an internal or external actor.