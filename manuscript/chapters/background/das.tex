\section[Adaptive systems]{\Glspl{adptSyst}}
\label{sec:back:adapt-syst}

- \glspl{sadapt} defined in response to the increasing complexity of software systems and their environment~\cite{computing2006architectural, DBLP:journals/computer/KephartC03}
	- more and more complex to install, configure, tune, and maintain these systems
	- evolve in uncertain environment~\cite{DBLP:conf/dagstuhl/EsfahaniM10}
- a \gls{adptSyst} can be automatically adapted to achieve its goals based on high-level objectives in response to changes in the environment of the system itself~\cite{DBLP:conf/dagstuhl/ChengLGIMABBBCSDFGGGKKKLMMMPSTTWW09}
- \glspl{sadapt} is an \gls{adptSyst} that adapts itself, autonomously, with minimal interference~\cite{DBLP:conf/dagstuhl/BrunSGGKLMPS09}


\subsection{Principles and vision}

- 2 principles for \glspl{sadapt}~\cite{DBLP:books/sp/19/Weyns19}
	- internal principle, base on the \textquote{discipline split}~\cite{DBLP:conf/icse/AnderssonLMW09}
		- part that deals with the domain concern (concerns for which the system is built), the business part
		- part that deals with the adaptation concern, interacts with the first one
	- external principle
		- \glspl{sadapt} handle changes and uncertainties in their environment, the system, and the goal autonomously, without minimal human interference	
- SAS principles can be employed in any software layer (low level, middleware, ...)
- optimal adaptation should be performed without interfering with the SAS activity~\cite{DBLP:journals/tse/KramerM90}

- distinguish between 4 families, often referred to as the self-* features~\cite{computing2006architectural}
	- self-healing: \textquote{To discover, diagnose, and act to prevent disruptions}~\cite{computing2006architectural}
	- self-optimisation: \textquote{To tune resources and balance workloads to maximize the use of information technology resources.}~\cite{computing2006architectural}
	- self-configuration: \textquote{To adapt to dynamically changing environments.}~\cite{computing2006architectural}
	- self-protection: \textquote{To anticipate, detect, identify, and protect against threats.}~\cite{computing2006architectural}
- we can also consider self-organisation~\cite{dempster1998self}
	- "Self-organisation is a dynamical and adaptive process where systems acquire and maintain structure themselves, without external control."~\cite{DBLP:conf/atal/WolfH04}
	- component apply local rules to adapt their interactions and cooperatively realise adaption
	- but have emergent behaviour
	
- \glspl{sadapt} are composed of 4 elements~\cite{DBLP:books/sp/19/Weyns19}
	- environment:
    	- part of the external world with which the SAS interacts and in which the effects of the system will be observed and evaluated \cite{DBLP:journals/ansoft/Jackson97}
    	- can contain both virtual or physical entities
    	- difference between the SAS and the environment: the SAS can be controlled by the developer whereas he cannot control the environment
	- managed system:
    	- the system that reaslises the domain functionality (domain concern)
    	- it evolves in en environment
    	- should contain sensors and actuators to be monitored and adapted
    	- different names in the litterarture:
        	- managed element \cite{DBLP:journals/computer/KephartC03}
        	- system layer \cite{DBLP:journals/computer/GarlanCHSS04}
        	- core function \cite{DBLP:journals/taas/SalehieT09}
        	- base-level subsystem \cite{DBLP:journals/taas/WeynsMA12}
        	- controllable plant \cite{DBLP:conf/icse/FilieriHM14}
	- adaptation goals: 
    	- concerns of the managing system over the managed systems
    	- related to software qualities of the managed system \cite{DBLP:conf/ecsa/WeynsA13}
    	- 4 kinds of goals \cite{DBLP:journals/computer/KephartC03}: configuration, optimisation, healing, protection
    	- can evolve over time
    	- goals expressed in terms of the uncertainty they have to deal with
        	- probabilistic temporal logics \cite{DBLP:journals/tse/CalinescuGKMT11}
        	- fuzzy goals, expressed by fuzzy constraints \cite{DBLP:conf/re/BaresiPS10}
	- managing system:
    	- adapt the managed system according to the adaptation goals
    	- monitor the environment and the managed systems
    	- different names:
        	- autonomic manager \cite{DBLP:journals/computer/KephartC03}
        	- architecture layer \cite{DBLP:journals/computer/GarlanCHSS04}
        	- adaptation engine \cite{DBLP:journals/taas/SalehieT09}
        	- reflective subsystem \cite{DBLP:journals/taas/WeynsMA12}
        	- controller \cite{DBLP:conf/icse/FilieriHM14}

- different mechanism can be used to implement a SAS~\cite{DBLP:journals/computer/GarlanCHSS04}
	- traditional one, like exceptions or fault-tolerant protocols
	- feedback control loop, taken from control control theory~\cite{DBLP:conf/dagstuhl/BrunSGGKLMPS09}
- the common implementation of the feedback control loop is the MAPE-K model \cite{DBLP:journals/computer/KephartC03, computing2006architectural}
	- Monitor, that continuously monitor the managed element and its environment, it updates the Knowledge
    - Analyse, from the Knowledge, detects if there is a need for adaptation regarding the adaptation goals
    - Plan, that define the set of adaptation actions to realise the goals
    - Execute, that executes the plan

\subsection[Models@run.time]{\Gls{m@rt}}

- 6 Waves of engineering \glspl{sadapt}
	- automating tasks
		- focus on the automation of recurrent and error-prone management tasks~\cite{DBLP:journals/computer/KephartC03}: installing, configuring, operating, optimising, and maintaining heterogeneous systems 
	- architecture-based adaptation
		- architecture as fundation to support systematic runtime change and manage the complexity of engineering SAS
	- \Gls{m@rt}~\cite{DBLP:journals/computer/BlairBF09}
		- "causally connected self-representation of the associated system that emphasises the structure, behaviour, or goals of the system from a problem space perspective"
	- Goal-driven Adaptation
		- two sides: how to specify the requirements of a system that is exposed to uncertainties and what are the requirement of the solution, found by a feedback control loop, is intended to solve?
	- Guarantees Under Uncertainties
		- uncertainty is key driver for self-adaptation, 
		- Guarantees for a managing system include guarantees for the adaptation goals (qualities) and the functional corectness of the adaptation components themselves
	- Control-Based Adaptation
		- use of the control theory for \glspl{sadapt}
		
		
- models@run.time: "causally connected self-representation of the associated system that emphasises the structure, behaviour, or goals of the system from a problem space perspective" [7]
    - stucture the infromation received by the SAS
    - causally connection:
        - modification of the system are reflected in the model
        - modification of the model triggers modification of the system
    - abstraction of the system and its goals
    - 4 dimensions that can be used for classification:
        - structural vs behavioural
        - procedural (how: organisation or execution of the system) vs declarative (what: purpose of the adaptation: goal/requirement modeling)
        - functional (functions of the SAS) vs non-functional (quality properties of the SAS)
        - formal vs non-formal
    - \cite{DBLP:journals/computer/MorinBJFS09} defines models@run.time as a set of configuration points, that can be selected at runtime in response to changes

    - model-oriented architecture
        - 3 layers
            - Online Model Space: platform independent layer, manipulates model
            - Causal Connection: platform-specific, "drivers", link between the 2 layers: model and runtime spaces
            - Business Application: contains application logic, is equipped with sensors that track runtime events, equipped with sensors and "factories services" (actuators)
        - 4 types of runtime models
            - feature model, describes the configuration/variability point
            - context model, specifies relevant variables of the environment
            - reasoning model, associates features with particular context, example: Event-Condition-Action rules
            - architecture model, specifies the component composition of the application
        - 5 elements
            - Event processor (implement the Monitor stage): update the context model with the received information
            - Goal-Based Reasoner (implements the Analyse stage): uses the feature and reasoning model to select the new features that should be executed by the SAS, according to goals
            - Model Weaver (implements part of the Plan stage): from the selected feature, compose a new architecture model
            - Configuration checker (implements part of the Plan stage): check the configuration at runtime
            - Configuration manager (implements part of the Plan stage + the Execute stage): define the right sequence of actions to reach the proposed architecture model, execute them 

Sum-up:
    - causally connection between the the DAS and the model: model represents the up-to date state of the system, modification on the model will trigger modification on the system
    - models allow to tame the complexity and the huge amount of information
    - with goal first class citizen, models enable anaysis of the behaviour of SAS at runtime, supporting decision making
    - 4 dimensions of runtime models: structural or behavioural, procedural or declarative, functional or non-functional, formal or non-formal
    - SAS can be viewed as a set of possible configuration that should be selected/modified at runtime 

\subsection{Context}

\subsection[Adaptive systems in the context of this thesis]{\Glspl{adptSyst} in this thesis}

- following the \gls{mapek} loop, the adaptation is based on a shared knowledge
- this knowledge is uncertain and time related
- in this thesis, we propose a meta-model to represent decisions over time, following the \gls{m@rt} paradigm
- a language to handle uncertainty







- 0  DBLP:books/sp/19/Weyns19
- 2  DBLP:conf/icse/AnderssonLMW09
- 5  DBLP:conf/re/BaresiPS10
- 10 DBLP:conf/dagstuhl/BrunSGGKLMPS09
- 13 DBLP:journals/tse/CalinescuGKMT11
- 17 DBLP:conf/dagstuhl/ChengLGIMABBBCSDFGGGKKKLMMMPSTTWW09
- 23 DBLP:conf/atal/WolfH04
- 27 DBLP:conf/dagstuhl/EsfahaniM10
- 28 DBLP:conf/icse/FilieriHM14
- 30 DBLP:journals/computer/GarlanCHSS04
- 36 computing2006architectural
- 38 DBLP:journals/ansoft/Jackson97
- 39 DBLP:journals/computer/KephartC03
- 40 DBLP:journals/tse/KramerM90
- 44 DBLP:journals/computer/MorinBJFS09
- 52 DBLP:journals/taas/SalehieT09
- 59 DBLP:conf/ecsa/WeynsA13
- 61 DBLP:journals/taas/WeynsMA12